\chapter{Futuros trabalhos}

\indent Assim como muitos projetos, o atual se apresenta como um recorte limitado, tendo potencial para expensão e refinamento. Espera-se que, frente a essa nova dinâmica epidemiológica da dengue no Estado de \acrlong{SC}, projetos continuem a linha de pesquisa, assim como esse também o foi.

\section{Sugestões para trabalhos futuros}

\indent Para aprimorar o trabalho atual, foram listadas sugestões para implementar em trabalhos futuros, segue abaixo:

\begin{itemize}
    \item Utilização de pontos urbanos centrais das cidades;
    \item Utilização de série histórica de temperaturas (mínima, média e máxima) do \acrfull{Bdmep/Inmet} para cobrir a máscara do \ingles{\acrshort{SAMeT}} nos municípios de Balneário Camboriú, Bombinhas e Porto Belo;
    \item Inclusão de dados sociais para refinamento do modelo;
    \item Inclusão de cálculos epidemiológicos a partir de dados sociais;
    \item Inclusão de casos suspeitos e inconclusivos de dengue para adequação com a nova metodologia utilizada pela \acrshort{Dive}/\acrshort{SC};
    \item Utilização de outras bases de dados secundários para pesquisa em saúde;
    \item Utilização de dados do \ingles{\acrfull{CFS}} do \ingles{\acrshort{NCEP}/\acrshort{NOAA}};
    \item Utilização dos primeiros registros entomo-epidemiológicos para avaliar a distribuição temporal no território catarinense;
    \item Derivação dos dados, tanto climatológicos quanto entomo-epidemiológicos, para síntese de índices e taxas;
    \item Adaptação da modelagem para modelo dinâmico a partir de cálculos epidemiológicos e história natural da doença;
    \item Adaptação dos dados conforme Níveis de Emergência do Plano de Contingência de Dengue em \citeonline{contingenciaSCdengue};
    \item Ajustes de hiperparâmetros durante a modelagem;
    \item Ajustes de modelagem a partir de outras metodologias;
    \item Ajustes de validação de modelagem a partir de outras métricas;
    \item Análise pormenorizada das séries de anomalias estacionárias entomo-epidemiológicas e climatológicas;
    \item Análise de modelagem a partir de novos cenários presentes (vacinação e presença de \latim{Aedes-Wolbachia}).
\end{itemize}
