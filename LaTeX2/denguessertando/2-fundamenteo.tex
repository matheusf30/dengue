\chapter{Fundamentação Teórica}

Esta seção apresenta uma breve contextualização sobre questões sanitárias e climatológicas. Foi elaborada através de  fichamento do material científico acessado: livros, artigos, \ingles{sites}, boletins e informes oficiais, dentre outros. Essa revisão bibliográfica servirá como base para conciliar diferentes áreas do conhecimento, assim como facilitar o entendimento de terminologias abordadas no presente estudo.

\section{Caracterização do Clima e Ambiente}

\subsection{Clima Catarinense}

\indent "Clima é a sucessão habitual dos diversos tipos de tempo que compõem o cenário atmosférico de uma região ao longo de um período de pelo menos trinta anos" (\acrlong{SC}, \citeyear{AtlasSCnatureza}, pg-75 ).

\indent Segundo a classificação de \citeonline{reboita2010},  o Estado de \acrlong{SC} se encontra no setor R4 de precipitação, sendo similar ao restante do sul do Brasil, ao Paraguai e ao Uruguai; setor onde o total pluviométrico é elevado e bem distribuído ao longo do ano. Esse comportamento se deve a atuação de sistemas meteorológicos na região. A presença de vórtices ciclônicos e cavados em altos níveis sobre a costa oeste da América do Sul favorece o desenvolvimento de ciclones e \acrlong{FF} nesse setor (R4). Dessa forma, condições ciclogenéticas, e/ou frontogenéticas, se desenvolvem nesse próprio setor.

\indent De acordo com  \citeonline{reboita2010}, os Sistemas Frontais, que se deslocam do sul em sentido Noroeste-Nordeste, passando pela Argentina e adentrando o Brasil, causam precipitação atuando diretamente na região ou fornecendo condições para o desenvolvimento
de \acrfull{LI} pré-frontais.

\indent As condições favoráveis à ocorrência de tempestades aumentam quando o processo convectivo é acoplado a um ou mais sistemas de instabilidade, como vórtices ciclônicos, baixas térmicas, \acrfull{JBN} ou a passagem de \acrlong{FF} pelo litoral sul do Brasil (\acrlong{SC}, \citeyear{AtlasSCnatureza}).

DESCREVER \acrlong{CCM};

DESCREVER Sistema Clicônico em Médios Níveis (vírgula invertida);

DESCREVER \acrlong{ZCAS};

DESCREVER  sistemas de circulação locais (brisas).

\indent De acordo com o Atlas Geográfico de \acrlong{SC} (\citeyear{AtlasSCnatureza}), os maiores índices de precipitação anuais são observados na região Oeste, Grande Florianópolis e Litoral Norte. Entretanto, no Litoral Sul, entre Araranguá e Laguna, os totais anuais desses índices são os menores. O efeito sazonal da dinâmica dos sistemas atmosféricos, sobretudo pelas altas pressões, e a influência da orografia, principalmente por planícies e planaltos, resultam em variações na distribuição espacial de elementos climáticos, como temperatura e precipitação, que se observa menores temperaturas médias em regiões de maiores altitudes e tendência com maiores acumulados de precipitação próximos às serras. Em consonância com \citeonline{Oliveira2022Sistema}, a definição do sistema de anticiclone é tida como centros de alta pressão atmosférica.

\begin{citacao}
"...A maior influência ocorre pelo anticiclone Migratório Polar, centro de ação da Massa de Ar Polar, úmida quando a trajetória é marítima (mPa) e seca quando a trajetória é continental, sempre fria (mPc); pelo anticiclone do Atlântico, centro de ação da Massa Tropical Atlântica (mTa), quente e úmida e pela depressão do Chaco, que é o centro de ação da Massa Tropical Continental (mTc), quente e seca (\acrlong{SC}, \citeyear{AtlasSCnatureza}, pg-77)." 
\end{citacao}

\indent "Os elementos que constituem as condições momentâneas de tempo passam a ser denominados elementos climáticos quando utilizados para fins de estudos relacionados ao clima" (\acrlong{SC}, \citeyear{AtlasSCnatureza}, pg-75).


\subsection{Fatores Abióticos}

O Estado catarinense está situado no sul do Brasil, tendo limites com o Estado do Paraná ao norte e, ao sul, com o Estado do Rio Grande do Sul. Esses três Estados  formam a região Sul do Brasil. \acrlong{SC} também tem limites a leste com o oceano Atlântico e a oeste com a República Argentina.  O Estado ocupa aproximadamente 1\% da território nacional e  cerca de 16\% da área total da Região Sul (\acrlong{SC}, \citeyear{AtlasSCterritorio}).

\indent É possível notar diferentes paisagens em \acrlong{SC}; essas, criadas por formas de relevos variadas. Sob esses aspectos, pode-se observar os principais compartimentos de relevo no Estado catarinense: Planície (Costeira), Serras (do Mar, do Tabuleiro-Itajaí e Geral), Planaltos (de São Bento do Sul, de Lages, dos Campos Gerais e Dissecado Rio Iguaçu-Rio Uruguai), Patamares (de Mafra, do Alto Rio Itajaí e da Serra Geral) e Depressão (da Zona Carbonífera) (\acrlong{SC}, \citeyear{AtlasSCnatureza}).

\indent DESCREVER Paisagens;

\indent DESREVER Hipsometria;

\subsection{Fatores Bióticos}

\indent DESCREVER Fitofisionomia;

\subsection{Contexto Histórico}

\indent DESCREVER Território;

\indent DESCREVER Colonização territorial.

\section{Contextualização sobre Saúde}

\subsection{Saúde para Ciências Exatas e da Terra}

\begin{center}
\textcolor{red}{CONSIDERAÇÕES AQUI!}\\
\textcolor{red}{JOÃO: Geomedicina (Geografia médica, Geografia da saúde, medicina social, espaço).}\\
\end{center}

\indent Sendo o \acrfull{PCAM} vinculado a Ciências Exatas e da Terra, particularmente dentro das GeoCiências, de acordo com sistematização das áreas de conhecimento da \acrfull{CAPES} (\citeyear{CAPES_Tabela_Conhecimento}) e publicizado pelo \acrfull{CNPq} \citeyear{CNPq_Tabela_Conhecimento}), deve-se atentar para nomenclaturas e abordagens sobre saúde.

\indent Segundo \citeonline{Geomedicine2012Davenhall}, a Geomedicina é definida como uma nova área de inteligência médica que utiliza dados e infraestrutura espaciais para benefício da prória saúde humana; para \citeonline{Geomedicine1990JulLag}, é uma área da ciência que observa influências e relações do meio ambiente com a distribuição espacial de agravos em saúde de homens e animais.

\indent Geografia médica.

\indent Geografia da saúde.

\indent Medicina social/espaço. RENE AREIA, Associação de Medicina Social da Bélgica "enfatizou a importância do social, político e cultural na origem e persistência de doenças epidêmicas".

\begin{center}
\textcolor{red}{CONSIDERAÇÕES AQUI!}\\
\indent \textcolor{red}{JOÃO: Tópico para Saúde Coletiva/Pública e Saúde Única\\(SEPARAR EM TÓPICOS DISTINTOS)}\\
\end{center}

%\subsection{Saúde: Coletiva e Única}

\subsection{Saúde Coletiva}

\indent Adotando o conceito de saúde da Carta Magna para Saúde Mundial, desde planejamento de criação até a execução atual da \citeonline{OMS2022S1} (\acrshort{OMS}), tem-se saúde como um estado de completo bem-estar físico, mental e social, e não apenas como a ausência de doença ou enfermidade \cite{ParranHEALTH}

\indent The Manhattan Principles

\subsection{Saúde Única}

\indent EXPLICAR MAIS S1. EXPLICAR MAIS S1

%"É importante notar que a definição do preâmbulo sugere uma motivação idealista em favor da igualdade universal, que era novo, especialmente em muitos governos europeus, mesmo após a Segunda Guerra, e, além disso, o preâmbulo ligava saúde com termos como paz. Inspirados pelo postulados da medicina social, saúde pública não devia ser um produto isolado do resto da vida social, mas um processo intrínseco de desenvolvimento social", afirmou Marcos Cueto. --- %https://www.coc.fiocruz.br/index.php/pt/todas-as-noticias/319-saude-internacional-e-as-origens-da-oms\\
%DECRETO Nº 26.042, DE 17 DE DEZEMBRO DE 1948 --- https://www2.camara.leg.br/legin/fed/decret/1940-1949/decreto-26042-17-dezembro-1948-455751-publicacaooriginal-1-pe.html\\
%https://edisciplinas.usp.br/pluginfile.php/5733496/mod_resource/content/0/Constitui%C3%A7%C3%A3o%20da%20Organiza%C3%A7%C3%A3o%20Mundial%20da%20Sa%C3%BAde%20%28WHO%29%20-%201946%20-%20OMS.pdf\\

\indent Ainda para a \acrshort{OMS} (\citeyear{OMS2022S1}), saúde única é uma abordagem na interface humano-animal-ambiental para se alcançar melhores resultados em saúde coletiva. Vários setores que se intercomunicam, especialmente para o controle de zoonoses, promoção de segurança alimentar e combate a resistência a antibióticos, tendo interdependência e ligação entre saúde humana e saúde animal por meio de saúde de ecossistemas; para a \citeonline{FAO2022} (\acrshort{FAO}), há sinergismo entre estas políticas e estratégias, enquanto é entendido pelo \citeonline{CFMVSaude} (\acrshort{CFMV}) como uma união indissociável.

\indent Para a \citeonline{ONUODS22} (\acrshort{ONU}), como objetivo de desenvolvimento sustentável, deve ser assegurado uma condição de vida saudável e promoção de bem-estar para todas e todos, em todas as idades; além de combater doenças veiculadas pela água e outras doenças transmissíveis. Também é tido de suma importância a integração das medidas da mudança do clima nas políticas, estratégias e planejamentos nacionais. Ainda segundo a \acrshort{ONU} (\citeyear{ONUODS22}) é necessário implementar medidas para evitar a introdução de espécies exóticas e reduzir significativamente o impacto dessas espécies invasoras.

\subsection{Doenças Zoonóticas}

\indent A \acrshort{OMS} (\citeyear{WHO2020Zoonoses}) considera zoonoses como doenças infecciosas trasmitidas, via ingestão hídrica ou sólida, pelo meio ambiente ou por contato direto, entre humanos e animais-não-humanos. Elas representam o principal problema em saúde coletiva no mundo. Seus agentes etiológicos podem ser de caráter bacteriano, viral, parasitário ou envolver agentes não convencionais.

\indent Pensando nos animais como potenciais reservatórios para a transmissão da dengue, \citeonline{Dengue_Animals_Gwee_2021} em uma revisão sistemática, constataram que aproximadamente 11\% dos animais testados (...) apresentaram positividade para DENV, tanto PCR quanto sorologia.


\subsection{Ecoepidemiologia}

\indent DESCREVER TRÍADE EPIDEMIOLÓGICA: Agente, Hospedeiro e Ambiente.

\indent A dengue é uma doença viral de grande importância na saúde coletiva e sua forma epidêmica tem se adaptado ao ambiente urbano durante centenas de anos. \cite{ArboviralTransmission}.

\citeonline{ArboviralTransmission} "engue viruses are very important human arboviral pathogens and use humans as reservoir hosts. Aedes aegypti and Aedes albopictus mosquitoes are the most common vectors in urban settings. It is thought that the human epidemic form of dengue virus evolved in the last 2000 years, and genetic analysis indicates that mutations have resulted in adaptation to the urban mosquito host."

\indent Assim como citado por \citeonline{Fiocruz2010Atlas}, o vírus da dengue (\acrfull{DENV}) faz parte da família \latim{Flaviviridae} e possui estrutura genética formada por ácido ribonucleico (\acrfull{RNA}). \citeonline{ArboviralTransmission} também comentam que muitas pandemias have been attributed to the ability of some RNA viruses to change their host range to include humans. 

\indent A dengue é uma arbovirose, ou seja, é uma doença vetorizada por artrópodes, logo tem os seguintes mosquitos como transmissores no ciclo urbano peridomiciliar: \latim{Aedes aegypti} e \latim{Aedes albopictus} \cite{ArboviralTransmission}.

\indent Conforme adotado pela \citeonline{SBPGlossario} (\acrshort{SBP}), vetor é um artrópode, molusco ou veículo que transmite um parasito entre dois hospedeiros, que albergam o parasito; sendo vetor biológico quando o agente etiológico apresenta parte do ciclo biológico, multiplicando-se ou se desenvolvendo, no próprio animal vetor.

\subsection{Situação Epidemiológica Atualizada da Dengue em \acrlong{SC}}

\begin{center}
\textcolor{red}{CONSIDERAÇÕES AQUI!}\\
\indent \textcolor{red}{JOÃO: Situação Epidemiológica atualizada da dengue (referenciar o ano de que está se falando)}\\
\end{center}

\indent As informações atualizadas e expostas estão em acordo com a \citeonline{Informe8DiveSE12/23}.

\indent Embora haja redução (10,7\%) na quantidade de focos identificados, se comparados ao mesmo período do ano passado (28.114 focos do mosquito Aedes aegypti em 2022 e 25.105 focos em 2023), há aumento (8,97\%) de municípios infestados, em relação ao mesmo período de 2022 (132 municípios infestados anteriormente e 145, em 2023). Por ora, há 4.769 casos confirmados, o que significa uma diminuição de 39,55\% em relação a 2022 (12.058 confirmações até a décima segunda semana epidemiológica daquele ano), porém continuam 11.124 casos suspeitos atualmente.

\indent Outro ponto importante são os casos autóctones (transmissão dentro de \acrlong{SC}), que teve registro de 4.078 casos, distribuídos em 59 municípios. Desses casos, 361 amostras foram processadas para pesquisa viral pelo \acrfull{LACEN} de \acrlong{SC}, sendo: 98,34\% delas (355/228) identificadas o sorotipo DENV1 e 1,66\%, DENV2.

\indent Desses autóctones, o município de Palhoça concentra 1.593 casos e detém a maior taxa de incidência de dengue atualmente (673,18 casos/100 mil habitantes), sendo o primeiro município catarinense a atingir o nível de epidemia em 2023, na décima semana epidemiológica \cite{Informe6DiveSE10/23}. Nessa décima segunda semana epidemiológica, dois municípios atingiram o nível de epidemia: Palhoça e Saudades.

\indent Cabe ressaltar que o município de Joinville, o mais populoso do Estado de \acrlong{SC}, obteve alta taxa de incidência de casos autóctones em 2022, com 3.628,15 casos/100 mil habitantes (totalizando 21.423 casos autóctones) \cite{Informe31DiveSE52/22}.

\section{Emergências Ambientais}

\begin{center}
\textcolor{red}{CONSIDERAÇÕES AQUI!}\\ 
\textcolor{red}{JOÃO: Regulamento Sanitário Internacional (2005) e Emergência de Saúde Pública de Importância Nacional (Decreto 7.616/2011).}
\end{center}

\indent \citeonline{Cubas2014Tratado} confirmam a importância para as sociedades contemporâneas sobre os impactos de mudança climática global, de emergências ambientais e de modificação antrópica dos ecossistemas naturais. Os efeitos desses impactos se estendem à integridade do meio ambiente e à saúde coletiva, além de afetar diretamente a economia. As avaliações sistemáticas de vulnerabilidade socioambiental e de saúde em relação às modificações de larga escala do meio ambiente são urgentes, em seu sentido biofísico. 

\begin{citacao}
"Alguns estudos demonstram como a dinâmica ecoepidemiológica destes agravos tem sido afetada por mudanças ambientais. [...] Inventários faunísticos e microbiológicos, cenários de clima e seus efeitos em ecossistemas, e a implantação de sistemas permanentes e eficazes de monitoramento bioclimático, são aspectos a serem considerados pelos gestores públicos e pelas comunidades científica e conservacionista" \cite[pg-2325]{Cubas2014Tratado}.
\end{citacao}

\begin{citacao}
“Infere-se que, em primeiro lugar, o poder de ação dos médicos-veterinários engloba a monitoração e análise dos indicadores epidemiológicos. Em um segundo momento, conclui-se que os médicos-veterinários, vinculados ou não ao Programa Nacional de Controle da Dengue (PNCD), têm uma importante responsabilidade na geração de propostas de prevenção da dengue, chikungunya e zika” \cite[pg-12]{Silva2016O}.
\end{citacao}

\indent Levando em consideração a saúde única, devemos adotar um caráter preventivo frente a catástrofes, não apenas  intervindo de forma mitigadora para interromper processos crônicos estabelecidos de degradação ambiental, recuperação de ecossistemas e manejo de populações comprometidas \cite{Cubas2014Tratado}.
%\indent Em se tratando de populações comprometidas, é importante salientar que pessoas impactadas em desastres optam por não sair de suas residências para irem aos abrigos sem seus animais de estimação. Por isso, a Defesa Civil do Estado do \acrlong{PR} conta com uma rede de atendimento a animais em situação de emergência, tendo participação do \acrlong{CRMV} do \acrlong{PR} (\acrshort{CRMV}/\acrshort{PR}), \acrfull{CEGRADE} alinhada com a Comissão Nacional de Desastres do \acrfull{CFMV}, Secretarias Estaduais e Municipais, além de voluntários \cite{CRMVPR22CEGRADE}.
\begin{center}
\textcolor{red}{CONSIDERAÇÕES AQUI!}\\ 
\end{center}




%As citações diretas com menos de três linhas “devem estar entre aspas e devem mostrar entre parênteses o ano e a página da obra consultada.” (AUTOR, ano, página). Já as citações com mais de três linhas devem ser recuadas da margem esquerda em 4 cm, tamanho da fonte 10, espaçamento simples e texto sem aspas (ABNT, 2002, p. 2).


%\subsubsection{Subtítulo Quaternário}

%\section{Subtítulo Secundário 3a}
\section{Descrição das Bases de Dados}

As informações epidemiológicas (casos de dengue) e entomológicas (focos de \latim{Aedes} sp.) foram obtidas através de dados oficiais do Ministério da Saúde / \acrfull{SVS} / \acrfull{Dive}, através de plataformas \ingles{on-line} (\acrshort{Sinan}-\acrshort{DataSUS} e \acrshort{Sinan}-\acrshort{Dive}) e disponibilizados oficialmente (\acrshort{Dive}).\\
\indent Também foram utilizados dados georreferenciados atualizados para o momento atual (\acrfull{IBGE}) e sobre elementos climáticos de:
\begin{itemize}
    \item Precipitação e temperatura - fornecidos pelo \ingles{\acrfull{GFS}};
    \item Precipitação - obtido pelo produto \ingles{\acrfull{MERGE}};
    \item Temperatura - obtido pelo produto \ingles{\acrfull{SAMeT}}.
\end{itemize}

\subsection{\acrfull{DataSUS}}

O \acrshort{DataSUS} está inserido atualmente na pasta da Secretaria Executiva do Ministério da Saúde e a \acrfull{Dive} é vinculada à Superintendência de Vigilância em Saúde, da Secretaria de Estado da Saúde de Santa Catarina. O \acrshort{DataSUS} disponibiliza informações, via plataforma TABNET, que subsidiam  análises objetivas da situação sanitária para tomada de decisões baseadas em evidências e elaboração de programas de ações para promoção da saúde. Atualmente, dados como condições de vida, acesso a serviços, qualidade da atenção à saúde, morbidade, incapacidade, além de fatores ambientais, passaram a ser métricas utilizadas na construção de Indicadores de Saúde, que se traduzem em informação relevante para a quantificação e a avaliação das informações em saúde. (MINISTÉRIO DA SAÚDE, \citeyear{TABNETMinisterio}).

\subsection{\acrfull{Sinan}}

O \acrfull{Sinan} foi implantado em 1993, porém apenas em 1998 o \acrfull{Cenepi} coordenou e organizou as três esferas do governo por meio da Portaria \acrshort{Funasa}/Ministério da Saúde n.º 073 de 9/3/98,  regulamentando  e tornando obrigatória a alimentação regular dessa base de dados nacional. A partir de 2003, com a criação do \acrfull{SVS}, essa secretaria passa a ser responsável pelo sistema. Apesar de o \acrshort{Sinan} ser alimentado por notificações  de casos de doenças e agravos que constam da lista nacional de doenças de notificação compulsória, é facultado aos estados e municípios incluir outros agravos de saúde importantes em sua região ( MINISTÉRIO DA SAÚDE, \citeyear{SINANWEB}; MINISTÉRIO DA SAÚDE, \citeyear{SINAN07Ministerio}).

\subsection{Vigilantos}
O Vigilantos é o sistema informatizado \ingles{on-line} desenvolvido em 2012 pelo Estado de \acrlong{SC}. Os municípios podem inserir dados relativos ao \latim{Aedes aegypti}, facilitando e agilizando o acesso de todos os níveis a essa informação. Além de permitir registro, é possível fazer análise das informações de vigilância e controle vetorial através de dois módulos: Módulo Focos e Módulo Programa de Controle da Dengue. \cite{Vigilantos}

\subsection{\acrfull{IBGE}}

O \acrshort{IBGE} tem como missão, resumidamente: identificar e analisar o território; realizar a contagem da população; mostrar como a economia evolui através do trabalho e da produção das pessoas, revelando ainda como elas vivem.  O Instituto é o principal provedor de dados e informações do Brasil, que atendem às necessidades da sociedade civil, assim como dos órgãos das esferas governamentais federal, estadual e municipal \cite{IBGE22}, \cite{IBGE23prev}.

% \subsection{\acrfull{Bdmep/Inmet}}

% O \acrshort{Bdmep/Inmet} reune dados meteorológicos diários em formato digital, sendo atualizados a cada 90 dias e disponibilizados via \ingles{internet}. Essas séries históricas foram coletadas das várias estações meteorológicas convencionais da rede de estações do próprio \citeonline{INMET22}, que está vinculado diretamente ao \acrfull{MAPA}.

\subsection{\ingles{\acrfull{GFS}}}

Segundo \citeonline{GFS}, o \ingles{\acrshort{GFS}} é a regressão da reanálise do modelo \ingles{\acrfull{MERRA2}}; e atualizado para a versão 16.0 em março de 2020. Os dados são disponibilizados quatro vezes ao dia, com até 16 dias de predição, tendo variações: dados horários para até as primeiras 120h (0,25º) e dados a cada três horas para até 16 dias (0,5º). \textcolor{red}{CITAR CFS? TROCAR GFS?!}

\subsection{\ingles{\acrfull{MERGE}}}

De acordo com \citeonline{Rozante2010MERGE} o produto \acrshort{MERGE} é a interpolação de dados do satélite \ingles{\acrfull{TRMM}} e a precipitação observada por estações de superfície. Esse produto é de alta resolução (0.1º), sendo gerados dados históricos  diários a partir de junho de 2000, e horários desde 2010;

\subsection{\ingles{\acrfull{SAMeT}}}
 O produto \acrshort{SAMeT} também é de alta resolução (0,05º), sendo a interpolação de estimativas do modelo \ingles{\acrfull{ERA5}} [\ingles{\acrfull{ECMWF}}] e regressão linear simples com \ingles{\acrfull{GTOPO30}}, além de observações de estações (convencionais e automáticas) de temperatura a 2 metros \cite{Rozante2021SAMeT};


