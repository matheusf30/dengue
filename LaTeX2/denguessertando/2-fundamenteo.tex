\chapter{Fundamentação Teórica}

Esta seção apresenta uma breve contextualização sobre questões sanitárias e climatológicas. Foi elaborada através de  fichamento do material científico acessado: livros, artigos, \ingles{sites}, boletins e informes oficiais, dentre outros. Essa revisão bibliográfica servirá como base para conciliar diferentes áreas do conhecimento, assim como facilitar o entendimento de terminologias abordadas no presente estudo.

\section{Caracterização do Clima e Ambiente}

\subsection{Climatologia e Sistemas Meteorológicos de Santa Catarina (SC)}

\indent "Clima é a sucessão habitual dos diversos tipos de tempo que compõem o cenário atmosférico de uma região ao longo de um período de pelo menos trinta anos" (\acrlong{SC}, \citeyear{AtlasSCnatureza}, pg-75 ).

\indent Segundo \citeonline{reboita2010}, \acrlong{SC} se encontra no setor R4 de precipitação, sendo similar ao restante do sul do Brasil, ao Paraguai e ao Uruguai. No entanto, utilizando a classificação de \citeonline{MERGEatual}, o Estado catarinense está compreendido no setor R1 de precipitação. Ambos os autores defendem um total pluviométrico elevado e bem distribuído ao longo do ano. Esse comportamento se deve a atuação de sistemas meteorológicos na região. A presença de vórtices ciclônicos e cavados em altos níveis sobre a costa oeste da América do Sul favorece o desenvolvimento de ciclones e \acrlong{FF} nesses setores (R1 \cite{MERGEatual} e R4 \cite{reboita2010}). Dessa forma, condições ciclogenéticas e frontogenéticas se desenvolvem sobre \acrlong{SC}.

\indent Os sistemas frontais, que se deslocam do sul em sentido Noroeste-Nordeste, passando pela Argentina e adentrando o Brasil, causam precipitação atuando diretamente na região ou fornecendo condições para o desenvolvimento de \acrfull{LI} pré-frontais \cite{reboita2010}.

\indent As condições favoráveis à ocorrência de tempestades aumentam quando o processo convectivo é acoplado a um ou mais sistemas de instabilidade, como vórtices ciclônicos, baixas térmicas, \acrfull{JBN} ou a passagem de \acrlong{FF} pelo litoral sul do Brasil (\acrlong{SC}, \citeyear{AtlasSCnatureza}).

DESCREVER \acrlong{CCM};

DESCREVER Sistema Clicônico em Médios Níveis (vírgula invertida);

DESCREVER \acrlong{ZCAS};

DESCREVER  sistemas de circulação locais (brisas).

\indent De acordo com o Atlas Geográfico de \acrlong{SC} (\citeyear{AtlasSCnatureza}), os maiores índices de precipitação anuais são observados na região Oeste, Grande Florianópolis e Litoral Norte. Entretanto, no Litoral Sul, entre Araranguá e Laguna, os totais anuais desses índices são os menores. O efeito sazonal da dinâmica dos sistemas atmosféricos, sobretudo pelas altas pressões, e a influência da orografia, principalmente por planícies e planaltos, resultam em variações na distribuição espacial de elementos climáticos, como temperatura e precipitação, que se observa menores temperaturas médias em regiões de maiores altitudes e tendência com maiores acumulados de precipitação próximos às serras. Em consonância com \citeonline{Oliveira2022Sistema}, a definição do sistema de anticiclone é tida como centros de alta pressão atmosférica.

\begin{citacao}
"...A maior influência ocorre pelo anticiclone Migratório Polar, centro de ação da Massa de Ar Polar, úmida quando a trajetória é marítima (mPa) e seca quando a trajetória é continental, sempre fria (mPc); pelo anticiclone do Atlântico, centro de ação da Massa Tropical Atlântica (mTa), quente e úmida e pela depressão do Chaco, que é o centro de ação da Massa Tropical Continental (mTc), quente e seca (\acrlong{SC}, \citeyear{AtlasSCnatureza}, pg-77)." 
\end{citacao}

\indent "Os elementos que constituem as condições momentâneas de tempo passam a ser denominados elementos climáticos quando utilizados para fins de estudos relacionados ao clima" (\acrlong{SC}, \citeyear{AtlasSCnatureza}, pg-75).

\subsection{Aspectos Bióticos e Abióticos}

\indent Cabe destacar que, antes da promulgação da Constituição Federal de 1988, o Estado brasileiro estabelece a \acrfull{PNMA}, que tem a preservação, melhoria e recuperação da qualidade ambiental proprícia à vida como objetivo principal, conforme a luz da lei. Essa mesma política define recursos ambientais, englobando fauna, flora, elementos da biosfera, solo, subsolo, mar territorial, águas interiores, estuários, águas subterrâneas, águas superficiais e atmosfera, além de definir meio ambiente como "o conjunto de condições, leis, influências e interações de ordem física, química e biológica, que permite, abriga e rege a vida em todas as suas formas" \cite{BRASIL1981LeiPNMA}. Logo, meio ambiente é um conceito amplo que, dependendo da doutrina, vai além dos aspectos naturais.  

\indent O Estado catarinense está situado no sul do Brasil, tendo limites com o Estado do Paraná ao norte e, ao sul, com o Estado do Rio Grande do Sul. Esses três Estados  formam a região Sul do Brasil. \acrlong{SC} também tem limites a leste com o oceano Atlântico e a oeste com a República Argentina.  O Estado ocupa aproximadamente 1\% da território nacional e  cerca de 16\% da área total da Região Sul (\acrlong{SC}, \citeyear{AtlasSCterritorio}).

\indent É possível notar diferentes paisagens em \acrlong{SC}; essas, criadas por formas de relevos variadas. Sob esses aspectos, pode-se observar os principais compartimentos de relevo no Estado catarinense: Planície (Costeira), Serras (do Mar, do Tabuleiro-Itajaí e Geral), Planaltos (de São Bento do Sul, de Lages, dos Campos Gerais e Dissecado Rio Iguaçu-Rio Uruguai), Patamares (de Mafra, do Alto Rio Itajaí e da Serra Geral) e Depressão (da Zona Carbonífera) (\acrlong{SC}, \citeyear{AtlasSCnatureza}).

\indent O relevo define a grande divisão regional de \acrlong{SC}: região do
litoral e encostas, e região do planalto. Somado a esse relevo singular, o clima explica o processo colonial tardio do Estado catarinense (\acrlong{SC}, \citeyear{AtlasSCpopulacao}).

\indent Como verificado no Atlas Geográfico de \acrlong{SC} (\citeyear{AtlasSCnatureza}), hipsometria do Estado catarinense é muito diversa, variando de mais de 1.800 metros ao zero, no nível do mar. Esse comportamento ocorreu pelo soerguimento epirogenético durante o período Terciário, que afetou toda a porção leste do continente Sul Americano, também por conta de processos erosivos e a diferentes resistências de rochas. A zona costeira de \acrlong{SC} varia de 0 a 200 metros, atingindo 400 metros em regiões próximas ao litoral, no baixo curso do rio Itajaí-Açú e no fundo do vale do rio Uruguai e seus afluentes. A maior parte da bacia hidrográfica do Itajaí se encontra entre 400 e 800 metros, assim como parte do vale do rio Uruguai e regiões do Meio Oeste catarinense. Altitudes entre 800 e 1.200 metros são mais comuns em \acrlong{SC}, sendo decorrentes de serras, escarpas, planaltos e patamares. Nessas altitudes se encontram os divisores de água do Estado catarinense. "Por causa destas altitudes, as temperaturas são mais amenas e o clima se torna mais propício para o cultivo de frutas de clima temperado" (\acrlong{SC}, \citeyear{AtlasSCnatureza}). "Altitudes maiores do que 1.200 m são pouco comuns em Santa Catarina, [...] as elevadas altitudes permitem, em condições meteorológicas favoráveis, a ocorrência de neve..." (\acrlong{SC}, \citeyear{AtlasSCnatureza}).

\indent  O Atlas Geográfico de \acrlong{SC} (\citeyear{AtlasSCnatureza}) também comenta que remanescentes de cobertura primária ainda são existentes, porém são pequenos grupamentos e estão em locais de difícil acesso. Parte dessa fragmentação, também em coberturas secundárias, ocorre devido ao avanço agropecuário em \acrlong{SC} e o  processo de urbanização, principalmente, na segunda metade do século XX. A cobertura vegetal catarinense é biodiversa, resultante de variedades de climas, relevo e solos. A vegetação litorânea é composta por restinga e mangue. As coberturas florestais ocorrem nas encostas de serras e vales de vertente atlântica (floresta ombrófila densa), nos planaltos (floresta ombrófila mista) e vales do rio Uruguai e afluentes (floresta estacional decidual subtropical). Em elevadas altitudes, acima de 1.500 mestros, a cobertura vegetal é composta por floresta nebular e campos de altitude. Também ocorre vegetação campestre numa altitude aproximada de 800 metros, tendo precipitação bem distribuída ao longo do ano e médias de temperatura inferiores a 15 C. "Um tipo de floresta muito típico são os Faxinais, que aparecem em altitudes entre 700 e 1.200 metros [...] são associações mistas de espécies da mata Pluvial com elementos da floresta de Araucária" (\acrlong{SC}, \citeyear{AtlasSCnatureza}).

\subsection{População Catarinense e Contexto Histórico}

\indent Como \citeonline{Lages2004Territorios} argumentam, "território é o espaço apropriado por um ator, sendo definido e delimitado por e a partir de relações de poder, em suas múltiplas dimensões". Os próprios autores ainda definem a noção de espaço, que representa uma abstração maior, tendo o território como produto da intervenção e do trabalho dos atores sobre determinado espaço. "Tal como a cultura, o território não é rígido, mas produto da dinâmica histórica das sociedades" (\acrlong{SC}, \citeyear{AtlasSCpopulacao}).

\indent Como ressaltado no Atlas Geográfico de \acrlong{SC} (\citeyear{AtlasSCpopulacao}) o território era ocupado por povos originários há milhares de anos, muito antes da colonização. São conhecidos três grupos indígenas, denominados: Guarani, Kaingang e Xokleng. "Os Xokleng no Alto Vale do Itajaí voltaram a se definir Laklãnõ recentemente". Com o processo histórico de colonização, houve redução  territorial desses povos, principalmente por uma política indigenista de aldeamento.

\indent O grupo Guarani possuia uma agricultura mais sofisticada e ocupava terras baixas, principalmente do litoral sul brasileiro e várzeas da bacia do rio da Prata. "Há mais de um milênio foi iniciada a ocupação Guarani na costa Atlântica sul, inclusive de Santa Catarina, cujas datações mais antigas estão entre os anos 930 e 980 em Imbituba". Os Laklãnõ (Xokleng) se deslocavam no território entre mata Atlântica e florestas de araucárias do planalto. Os Kaingang tinham as terras altas como território, isso por associação à mitologia do grande dilúvio e por predomínio da mata de araucária, do qual proporcionava a base alimentar (\acrlong{SC}, \citeyear{AtlasSCpopulacao}).

\begin{citacao}
"Os indígenas resistiram a esse processo de esbulho do seu patrimônio e de violência
física. O conceito de resistência deve ser amplo, não no sentido de luta armada, mas
de estratégias cotidianas de enfrentamento aos ditames da política indigenista. [...] Atualmente, é importante salientar que os povos indígenas habitantes do Brasil, incluindo os povos que vivem em \acrlong{SC}a, estão em crescimento demográfico, sendo uma característica desse processo o reconhecimento da sua própria identidade" (\acrlong{SC}, \citeyear{AtlasSCpopulacao}, pg-47).
\end{citacao}

\indent Assim descrito por \acrlong{SC} (\citeyear{AtlasSCpopulacao}), os estados da região sul brasileira, assim como o território catarinense, foram conquistados e colonizados de forma tardia, a partir do século XVII. "Certamente dentre os elementos referentes ao quadro natural, o clima é o mais ressaltado como um dos determinantes na estruturação tardia da formação meridional brasileira" (\acrlong{SC}, \citeyear{AtlasSCpopulacao}). Seria muito difícil, para o universo colonial marcado pela tropicalidade, oferecer aos interesses comerciais da Metrópole Portuguesa uma produção que atendesse, abaixo do Trópico de Capricórnio, ao predomínio do clima subtropical.

\indent Nesse mesmo século, XVII, fazia-se necessário um sistema defensivo litorâneo da ilha de \acrlong{SC}. Porém, a partir do século XVIII, houve a incorporação do litoral catarinense ao sistema econômico colonial português, o que levou a migração de açorianos e madeirenses para o território. Também iniciando nesse período, surgiram novos povoados na estrada das tropas para pouso e descanso, sendo anexados ao território catarinense no século XIX (\acrlong{SC}, \citeyear{AtlasSCterritorio}). "... A exploração das terras do planalto começa com os paulistas no século XVIII, instalando a pecuária extensiva nas manchas de campos naturais de Lages, Curitibanos e Campos Novos" (\acrlong{SC}, \citeyear{AtlasSCpopulacao}).

\begin{citacao}
"Santa Catarina vivencia seu processo de conquista e colonização, tardio em cerca
de cem anos frente ao Nordeste e Sudeste brasileiros, com povos fruto do crescimento demográfico da Colônia, caracterizando um processo migratório interno, no sentido norte-sul. Isto tanto no século XVII (Litoral), quanto ainda no XVIII (Planalto), mesmo século em que também se inicia a migração externa com populações vindas de além-mar, no caso da 2ª ocupação do litoral catarinense. A partir do século XIX tal processo imigratório se intensifica com a ocupação dos vales atlânticos e encostas pelos imigrantes europeus, excedente populacional  relativo a uma Europa em transformação frente ao avanço e desenvolvimento do modo de produção capitalista" (\acrlong{SC}, \citeyear{AtlasSCpopulacao}, pg-22). 
\end{citacao}

\indent Por meados do século XIX, algumas regiões européias, já em fase industrial, atravessam um período de crise econômica e turbulência política. Esse cenário de crise foi provocado pela expansão do capitalismo, o que levou ao aprofundamento e ampliação do processo de expropriação, responsável pelo próprio crescimento demográfico europeu. É nesse contexto que as transformações em áreas de capitalismo
tardio levam milhões de emigrantes a cruzar o oceano em busca de novas oportunidades. "...Foram criados, principalmente por alemães e italianos, vários núcleos rurais e urbanos em áreas correspondentes aos vales atlânticos e encostas dos rios Itajaí, Cachoeira, Cubatão, Tijucas, Tubarão, Urussanga e Araranguá" (\acrlong{SC}, \citeyear{AtlasSCpopulacao}).

\indent Também pela metade do século XIX, o Governo Imperial criou mais uma colônia militar, a do Chapecó. "Implantada em 1882 em área correspondente hoje ao município de Xanxerê". Além de fazer a guarda da fronteira, também protegia a porção mais a oeste do caminho das tropas, como as primeiras colônias nas encostas do planalto a leste. No início do século XX, instalaram-se pequenas propriedades rurais policultoras (Joaçaba, Chapecó e Concórdia), por conta da expansão demográfica do Rio Grande do Sul, principalmente com a chegada da ferrovia São Paulo-Rio Grande do Sul (1908-1910), no vale do rio do Peixe (\acrlong{SC}, \citeyear{AtlasSCpopulacao}).

\indent Também devemos tomar nota da população de origem africana que, assim como os povos indígenas, foi escravizada durante os períodos colonial e imperial. "Das origens, na África, houve predomínio da região Centro-Ocidental (Congo, Angola, Benguela), com alguma presença de africanos orientais (Moçambique) e ocidentais (Mina, Jejes)". Há registros que apontam para o crescimento gradual de africanos escravizados e indígenas administrados, desde o século XVII, com a fundação de São Francisco do Sul, Desterro e Laguna. Ainda assim, a captura e a escravização de indígenas estava condenada por parte das autoridades judiciais da capitania de São Paulo na década de 1720. Fato esse que intensificou o tráfico transatlântico de escravos africanos para os territórios portugueses na América nos séculos XVIII e XIX, marcando o início de uma virada demográfica. Parte desses afrodescendentes foi utilizada como mão-de-obra para contrução de fortificações no litoral, estações baleeiras, trabalho em engenhos de farinha de mandioca, açúcar, aguardente e na lida com o gado no planalto (\acrlong{SC}, \citeyear{AtlasSCpopulacao}). 

\begin{citacao}
"Desde o século XIX, a autoimagem de Santa Catarina é a de uma província ou um estado europeu, uma singularidade que nos distanciaria do restante do Brasil, marcado pela mestiçagem. A associação de “branco” com “europeu”, e deste com “civilização” e “progresso”, é um traço persistente do senso comum. Embora seja
inegável tanto a presença quanto a participação de descendentes de europeus na
constituição demográfica e econômica de Santa Catarina, o efeito colateral é a invisibilização da população de origem africana" (\acrlong{SC}, \citeyear{AtlasSCpopulacao}, pg-79).
\end{citacao}

\indent Não coincidentemente, o Brasil tem registro de dengue desde o século XIX, relatado em literatura por epidemias no Rio de Janeiro (1846) e em São Paulo (1852) \cite{Valle2015Dengue}.

\section{Contextualização sobre Saúde}

\subsection{Saúde para Ciências Exatas e da Terra}

\begin{center}
\textcolor{red}{CONSIDERAÇÕES AQUI!}\\
\textcolor{red}{JOÃO: Geomedicina (Geografia médica, Geografia da saúde, medicina social, espaço).}\\
\end{center}

\indent Sendo o \acrfull{PCAM} vinculado a Ciências Exatas e da Terra, particularmente dentro das GeoCiências, de acordo com sistematização das áreas de conhecimento da \acrfull{CAPES} (\citeyear{CAPES_Tabela_Conhecimento}) e publicizado pelo \acrfull{CNPq} \citeyear{CNPq_Tabela_Conhecimento}), deve-se atentar para nomenclaturas e abordagens sobre saúde.

\indent Segundo \citeonline{Geomedicine2012Davenhall}, a Geomedicina é definida como uma nova área de inteligência médica que utiliza dados e infraestrutura espaciais para benefício da prória saúde humana; para \citeonline{Geomedicine1990JulLag}, é uma área da ciência que observa influências e relações do meio ambiente com a distribuição espacial de agravos em saúde de homens e animais.

\indent DESCREVER Geografia médica.

\indent DESCREVER Geografia da saúde.

\indent Medicina social/espaço. RENE AREIA, Associação de Medicina Social da Bélgica "enfatizou a importância do social, político e cultural na origem e persistência de doenças epidêmicas".

\begin{center}
\textcolor{red}{CONSIDERAÇÕES AQUI!}\\
\indent \textcolor{red}{JOÃO: Tópico para Saúde Coletiva/Pública e Saúde Única\\(SEPARAR EM TÓPICOS DISTINTOS)}\\
\end{center}

%\subsection{Saúde: Coletiva e Única}

\subsection{Saúde Coletiva}

\indent Adotando o conceito de saúde da Carta Magna para Saúde Mundial, desde planejamento de criação até a execução atual da \citeonline{OMS2022S1} (\acrshort{OMS}), tem-se saúde como um estado de completo bem-estar físico, mental e social, e não apenas como a ausência de doença ou enfermidade \cite{ParranHEALTH}.

\indent BRASIL. Lei nº 8.080, de 19 de setembro de 1990. Lei Orgânica da Saúde. Dispõe sobre as condições  para a promoção, proteção  e  recuperação  da  saúde,  a  organização  e  o  funcionamento  dos  serviços correspondentes  e  dá  outras providências.

\indent Foucault - Medicina Social

\indent Abordagens articuladas á saúde: Foucault, Butler e Laqueur

\indent Maria Cecília Minayo (2001), reforma sanitária.

\indent Convergências epistemológicas entre a Bioética e saúde coletiva discutidas por Junges
e Zoboli (2012)

\indent Segundo Marsiglia (2013) o uso do termo “coletivo” no campo da saúde coletiva é de
fundamental importância, haja vista que

\indent The Manhattan Principles

\subsection{Saúde Única}

%"É importante notar que a definição do preâmbulo sugere uma motivação idealista em favor da igualdade universal, que era novo, especialmente em muitos governos europeus, mesmo após a Segunda Guerra, e, além disso, o preâmbulo ligava saúde com termos como paz. Inspirados pelo postulados da medicina social, saúde pública não devia ser um produto isolado do resto da vida social, mas um processo intrínseco de desenvolvimento social", afirmou Marcos Cueto. --- %https://www.coc.fiocruz.br/index.php/pt/todas-as-noticias/319-saude-internacional-e-as-origens-da-oms\\
%DECRETO Nº 26.042, DE 17 DE DEZEMBRO DE 1948 --- https://www2.camara.leg.br/legin/fed/decret/1940-1949/decreto-26042-17-dezembro-1948-455751-publicacaooriginal-1-pe.html\\
%https://edisciplinas.usp.br/pluginfile.php/5733496/mod_resource/content/0/Constitui%C3%A7%C3%A3o%20da%20Organiza%C3%A7%C3%A3o%20Mundial%20da%20Sa%C3%BAde%20%28WHO%29%20-%201946%20-%20OMS.pdf\\

\indent Ainda para a \acrshort{OMS} (\citeyear{OMS2022S1}), saúde única é uma abordagem na interface humano-animal-ambiental para se alcançar melhores resultados em saúde coletiva. Vários setores que se intercomunicam, especialmente para o controle de zoonoses, promoção de segurança alimentar e combate a resistência a antibióticos, tendo interdependência e ligação entre saúde humana e saúde animal por meio de saúde de ecossistemas; para a \citeonline{FAO2022} (\acrshort{FAO}), há sinergismo entre essas políticas e estratégias, enquanto é entendido pelo \citeonline{CFMVSaude} (\acrshort{CFMV}) como uma união indissociável.

\begin{citacao}
"Saúde Única é um enfoque colaborativo, multidisciplinar e multissetorial que pode abordar as ameaças à saúde na interface homem-animal-ambiente no âmbito subnacional, nacional e internacional, com o objetivo final de obter resultados de saúde ótimos reconhecendo as interconexões entre pessoas, animais, plantas e meio ambiente. Essa  interface, um aspecto definidor de Saúde Única, consiste num continuum de interações entre pessoas, animais e meio ambiente que permite a transmissão entre espécies de patógenos zoonóticos e emergentes" (\citeauthor{S1_OPAS_OMS}, \citeyear{S1_OPAS_OMS}, pg-2).
\end{citacao}

\indent Tripartite >>> Quadripartite

%https://www.who.int/news/item/01-12-2021-tripartite-and-unep-support-ohhlep-s-definition-of-one-health

\indent GOV.BR/UMA-SO-SAUDE

% https://www.gov.br/saude/pt-br/assuntos/saude-de-a-a-z/u/uma-so-saude

% https://www.gov.br/saude/pt-br/assuntos/saude-de-a-a-z/u/uma-so-saude/doencas-zoonoticas

% https://www.gov.br/saude/pt-br/assuntos/saude-de-a-a-z/u/uma-so-saude/biodiversidade

\indent No ano de 2024, esse mesmo Conselho emite a Portaria nº 115, que cria e nomeia a \acrfull{CNSU} do \acrshort{CFMV} . Essa Portaria considera  a necessidade de criar políticas que assegurem à população qualidade em saúde, sendo o médico-veterinário um  profissional importante na equipe multidisciplinar dessa área. Também é incluído, no conceito de Saúde Única, a conexão com saúde vegetal, além da abordagem integrada entre saúde humana, animal e ambiental. Ao \acrshort{CNSU} fica atribuído: analisar, sugerir, articular e propor políticas de Saúde aos órgãos competentes, além de demais competências e representações perante o \acrshort{CFMV} (\acrlong{CFMV}, \citeyear{CFMV2024PORTARIA}).

\indent Para a \citeonline{ONUODS22} (\acrshort{ONU}), como objetivo de desenvolvimento sustentável, deve ser assegurado uma condição de vida saudável e promoção de bem-estar para todas e todos, em todas as idades; além de combater doenças veiculadas pela água e outras doenças transmissíveis. Também é tido de suma importância a integração das medidas da mudança do clima nas políticas, estratégias e planejamentos nacionais. Ainda segundo a \acrshort{ONU} (\citeyear{ONUODS22}) é necessário implementar medidas para evitar a introdução de espécies exóticas e reduzir significativamente o impacto dessas espécies invasoras.

\subsection{Doenças Zoonóticas}

\indent Como citado por \citeonline{HumanAnimalInterface}, a própria interface homem-animal apresenta características versáteis e dinâmicas. Para esses autores, \textcolor{red}{The human-animal interface is characterized by a number of
attributes that have been acquired throughout the evolutionary history of the human species and the development of mankind. The main attributes of the human-animal interface include the evolutionary pathogen heritage of the human species as well as human demographics and behaviors associated with the human inventions of domestication, agriculture and food production, urbanization, worldwide migration, colonization and trade, and industrialization and globalization.}

\indent A \acrshort{OMS} (\citeyear{WHO2020Zoonoses}) considera zoonoses como doenças infecciosas trasmitidas, via ingestão hídrica ou sólida, pelo meio ambiente ou por contato direto, entre humanos e animais-não-humanos. Elas representam o principal problema em saúde coletiva no mundo. Seus agentes etiológicos podem ser de caráter bacteriano, viral, parasitário ou envolver agentes não convencionais.

%https://journals.asm.org/doi/epub/10.1128/microbiolspec.oh-0013-2012

\indent Pensando nos animais como potenciais reservatórios para a transmissão da dengue, \citeonline{Dengue_Animals_Gwee_2021} em uma revisão sistemática, constataram que aproximadamente 11\% dos animais testados (...) apresentaram positividade para DENV, tanto PCR quanto sorologia.


\subsection{Ecoepidemiologia}

\indent DESCREVER TRÍADE EPIDEMIOLÓGICA: Agente, Hospedeiro e Ambiente.

% https://www.who.int/news-room/fact-sheets/detail/dengue-and-severe-dengue

\indent A dengue é uma doença viral de grande importância na saúde coletiva e sua forma epidêmica tem se adaptado ao ambiente urbano durante centenas de anos. \cite{ArboviralTransmission}.

\indent Assim como citado por \citeonline{Fiocruz2010Atlas}, o vírus da dengue (\acrfull{DENV}) faz parte da família \latim{Flaviviridae} e possui estrutura genética formada por ácido ribonucleico de fita simples (\acrfull{ssRNA}). Essa mesma autora simplifica a estrutura do vírion em 3 partes: material genético, capsídeo viral e envelope.

\indent Aprofundando um pouco mais a análise da estrutura viral, percebe-se que o capsídeo tem composição proteíca (proteína-C) e que o envelope é formado por bicamada fosfolipídica com proteínas de membrana (proteína-M) e proteínas de envelope (proteína-E). Essas proteínas de membrana que conferem a infectividade. O tamanho do capsídeo é de 30nm, enquanto o vírus como um todo pode chegar a 65 nm \cite{Fiocruz2010Atlas}. 

\citeonline{ArboviralTransmission} também comentam que muitas pandemias tem sido atribuídas a capacidade de alguns vírus \acrshort{ssRNA} adaptarem e incluírem o ser humano como hospedeiros. Esses mesmo autores também citam que, para a dengue, essas mutações resultaram em adaptação ao mosquito urbano.

\indent A dengue é uma arbovirose, ou seja, é uma doença vetorizada por artrópodes, logo tem os seguintes mosquitos como transmissores no ciclo urbano peridomiciliar: \latim{Aedes aegypti} e \latim{Aedes albopictus} \cite{ArboviralTransmission}.

\indent Conforme adotado pela \citeonline{SBPGlossario} (\acrshort{SBP}), vetor é um artrópode, molusco ou veículo que transmite um parasito entre dois hospedeiros, que albergam o parasito; sendo vetor biológico quando o agente etiológico apresenta parte do ciclo biológico, multiplicando-se ou se desenvolvendo, no próprio animal vetor.

\subsection{Emergências Ambientais}

\begin{center}
\textcolor{red}{CONSIDERAÇÕES AQUI!}\\ 
\textcolor{red}{JOÃO: Regulamento Sanitário Internacional (2005) e Emergência de Saúde Pública de Importância Nacional (Decreto 7.616/2011).}
\end{center}

\indent Para iniciar, ao se tratar de riscos relacionados a desastres, refere-se ao
potencial de ocorrer algo danoso para a sociedade. Para a \acrfull{Cobrade}, os desastres podem ser categorizados em naturais ou tecnológicos. Desses, os desastres naturais ainda são agrupados em cinco (5): geológico, hidrológico, meteorológico, climatológico e biológico. Esse último desastre é dividido em dois (2) tipos: epidemias e insfestações/pragas [sic]. As epidemias são divididas em quatro (4) subtipos,  quanto ao agente etiológico: viral, bacteriano, parasítico ou fúngico. Especificamente para as doenças virais, esses desastres são interpretados como "aumento brusco, significativo e transitório da ocorrência de doenças infecciosas geradas por vírus", tendo 1.5.1.1.0 como código \acrshort{Cobrade} \cite{GIRD}.

\indent \citeonline{Cubas2014Tratado} confirmam a importância para as sociedades contemporâneas sobre os impactos de mudança climática global, de emergências ambientais e de modificação antrópica dos ecossistemas naturais. Os efeitos desses impactos se estendem à integridade do meio ambiente e à saúde coletiva, além de afetar diretamente a economia. As avaliações sistemáticas de vulnerabilidade socioambiental e de saúde em relação às modificações de larga escala do meio ambiente são urgentes, em seu sentido biofísico. 

\begin{citacao}
"Alguns estudos demonstram como a dinâmica ecoepidemiológica destes agravos tem sido afetada por mudanças ambientais. [...] Inventários faunísticos e microbiológicos, cenários de clima e seus efeitos em ecossistemas, e a implantação de sistemas permanentes e eficazes de monitoramento bioclimático, são aspectos a serem considerados pelos gestores públicos e pelas comunidades científica e conservacionista" \cite[pg-2325]{Cubas2014Tratado}.
\end{citacao}

%\begin{citacao}
%“Infere-se que, em primeiro lugar, o poder de ação dos médicos-veterinários engloba a monitoração e análise dos indicadores epidemiológicos. Em um segundo momento, conclui-se que os médicos-veterinários, vinculados ou não ao Programa Nacional de Controle da Dengue (PNCD), têm uma importante responsabilidade na geração de propostas de prevenção da dengue, chikungunya e zika” \cite[pg-12]{Silva2016O}.
%\end{citacao}
\textcolor{red}{INCLUIR\\Regulamento Sanitário Internacional (2005) e Emergência de Saúde Pública de
Importância Nacional (Decreto 7.616/2011).}

\begin{center}
\textcolor{red}{CONSIDERAÇÕES AQUI!}\\ 
\end{center}

\indent Levando em consideração a saúde única, devemos adotar um caráter preventivo frente a catástrofes, não apenas  intervindo de forma mitigadora para interromper processos crônicos estabelecidos de degradação ambiental, recuperação de ecossistemas e manejo de populações comprometidas \cite{Cubas2014Tratado}.
%\indent Em se tratando de populações comprometidas, é importante salientar que pessoas impactadas em desastres optam por não sair de suas residências para irem aos abrigos sem seus animais de estimação. Por isso, a Defesa Civil do Estado do \acrlong{PR} conta com uma rede de atendimento a animais em situação de emergência, tendo participação do \acrlong{CRMV} do \acrlong{PR} (\acrshort{CRMV}/\acrshort{PR}), \acrfull{CEGRADE} alinhada com a Comissão Nacional de Desastres do \acrfull{CFMV}, Secretarias Estaduais e Municipais, além de voluntários \cite{CRMVPR22CEGRADE}.

\subsection{Situação Epidemiológica Atualizada da Dengue em Santa Catarina (SC)}

%https://www.who.int/news-room/fact-sheets/detail/dengue-and-severe-dengue

\begin{center}
\textcolor{red}{CONSIDERAÇÕES AQUI!}\\
\indent \textcolor{red}{JOÃO: Situação Epidemiológica atualizada da dengue (referenciar o ano de que está se falando)\\ATUALIZAR}\\
\end{center}

\indent As informações atualizadas e expostas estão em acordo com a \citeonline{Informe8DiveSE12/23}.

\indent Embora haja redução (10,7\%) na quantidade de focos identificados, se comparados ao mesmo período do ano passado (28.114 focos do mosquito Aedes aegypti em 2022 e 25.105 focos em 2023), há aumento (8,97\%) de municípios infestados, em relação ao mesmo período de 2022 (132 municípios infestados anteriormente e 145, em 2023). Por ora, há 4.769 casos confirmados, o que significa uma diminuição de 39,55\% em relação a 2022 (12.058 confirmações até a décima segunda semana epidemiológica daquele ano), porém continuam 11.124 casos suspeitos atualmente.

\indent Outro ponto importante são os casos autóctones (transmissão dentro de \acrlong{SC}), que teve registro de 4.078 casos, distribuídos em 59 municípios. Desses casos, 361 amostras foram processadas para pesquisa viral pelo \acrfull{LACEN} de \acrlong{SC}, sendo: 98,34\% delas (355/228) identificadas o sorotipo DENV1 e 1,66\%, DENV2.

\indent Desses autóctones, o município de Palhoça concentra 1.593 casos e detém a maior taxa de incidência de dengue atualmente (673,18 casos/100 mil habitantes), sendo o primeiro município catarinense a atingir o nível de epidemia em 2023, na décima semana epidemiológica \cite{Informe6DiveSE10/23}. Nessa décima segunda semana epidemiológica, dois municípios atingiram o nível de epidemia: Palhoça e Saudades.

\indent Cabe ressaltar que o município de Joinville, o mais populoso do Estado de \acrlong{SC}, obteve alta taxa de incidência de casos autóctones em 2022, com 3.628,15 casos/100 mil habitantes (totalizando 21.423 casos autóctones) \cite{Informe31DiveSE52/22}.

\section{Descrição das Bases de Dados}

\subsection{Departamento de Informática do Sistema Único de Saúde (DataSUS)}

O \acrshort{DataSUS} está inserido atualmente na pasta da Secretaria Executiva do Ministério da Saúde e a \acrfull{Dive} é vinculada à Superintendência de Vigilância em Saúde, da Secretaria de Estado da Saúde de Santa Catarina. O \acrshort{DataSUS} disponibiliza informações, via plataforma TABNET, que subsidiam  análises objetivas da situação sanitária para tomada de decisões baseadas em evidências e elaboração de programas de ações para promoção da saúde. Atualmente, dados como condições de vida, acesso a serviços, qualidade da atenção à saúde, morbidade, incapacidade, além de fatores ambientais, passaram a ser métricas utilizadas na construção de Indicadores de Saúde, que se traduzem em informação relevante para a quantificação e a avaliação das informações em saúde \cite{TABNETMinisterio}.

\subsection{Sistema de Informação de Agravos de Notificação (Sinan)}

O \acrfull{Sinan} foi implantado em 1993, porém apenas em 1998 o \acrfull{Cenepi} coordenou e organizou as três esferas do governo por meio da Portaria \acrshort{Funasa}/Ministério da Saúde n.º 073 de 9/3/98, regulamentando  e tornando obrigatória a alimentação regular dessa base de dados nacional. A partir de 2003, com a criação do \acrfull{SVS}, essa secretaria passa a ser responsável pelo sistema. Apesar de o \acrshort{Sinan} ser alimentado por notificações  de casos de doenças e agravos que constam da lista nacional de doenças de notificação compulsória, é facultado aos estados e municípios incluir outros agravos de saúde importantes em sua região \cite{SINANWEB, SINAN07Ministerio}.

\indent O próprio \acrshort{Sinan} faz a publicização, em documento ofical ao final do ano, das semanas epidemiológicas do próximo ano. Essas, como mencionado pelo \citeonline{SemanaEpidemio}, foram convencionadas internacionalmente e são contadas a partir de domingo a sábado. "A primeira semana do ano é aquela que contém o maior número de dias de janeiro e a última a que contém o maior número de dias de dezembro."

\subsection{Vigilantos}

O Vigilantos é o sistema informatizado \ingles{on-line} desenvolvido em 2012 pelo Estado de \acrlong{SC}. Os municípios podem inserir dados relativos ao \latim{Aedes aegypti}, facilitando e agilizando o acesso de todos os níveis a essa informação. Além de permitir registro, é possível fazer análise das informações de vigilância e controle vetorial através de dois módulos: Módulo Focos e Módulo Programa de Controle da Dengue \cite{Vigilantos}.

\subsection{Instituto Brasileiro de Geografia e Estatístic (IBGE)}

O \acrshort{IBGE} tem como missão, resumidamente: identificar e analisar o território; realizar a contagem da população; mostrar como a economia evolui através do trabalho e da produção das pessoas, revelando ainda como elas vivem.  O Instituto é o principal provedor de dados e informações do Brasil, que atendem às necessidades da sociedade civil, assim como dos órgãos das esferas governamentais federal, estadual e municipal \cite{IBGE22}, \cite{IBGE23prev}.

% \subsection{Banco de Dados Meteorológicos do Instituto Nacional de Meteorologia (Bdmep/Inmet)}

% O \acrshort{Bdmep/Inmet} reune dados meteorológicos diários em formato digital, sendo atualizados a cada 90 dias e disponibilizados via \ingles{internet}. Essas séries históricas foram coletadas das várias estações meteorológicas convencionais da rede de estações do próprio \citeonline{INMET22}, que está vinculado diretamente ao \acrfull{MAPA}.

\subsection{\ingles{Global Forecast System (GFS)}}

\indent Segundo \citeonline{GFS}, o \ingles{\acrshort{GFS}} é a regressão da reanálise do modelo \ingles{\acrfull{MERRA2}}; e atualizado para a versão 16.0 em março de 2020. Os dados são disponibilizados quatro vezes ao dia, com até 16 dias de predição, tendo variações: dados horários para até as primeiras 120h (0,25º) e dados a cada três horas para até 16 dias (0,5º). \textcolor{red}{CITAR CFS? TROCAR GFS?!}

\subsection{\ingles{Merging Technique (MERGE)}}

\indent De acordo com \citeonline{Rozante2010MERGE} o produto \acrshort{MERGE} é a interpolação de dados do satélite \ingles{\acrfull{TRMM-TMPA}} e a precipitação observada por estações de superfície. Esse produto é de alta resolução (0.1º), com gradeamento de 10km² e disponibilização de dados diários a partir de junho de 2000, e horários desde 2010.

\indent Comentado por \citeonline{IMERG, TRMM-TMPA}, os dados do \ingles{\acrshort{TRMM-TMPA}} foram descontinuados e é fortemente sugerido utilizar dados do \ingles{\acrfull{GPM-IMERG}}. Esses dados são disponibilizados a cada 30 minutos e as estimativas são executadas em duas etapas: antecipada (\ingles{Early}) e tardia (\ingles{Late}). A estimativa \ingles{Early} tem atraso de apenas quatro horas e é compilada com dados do momento. Por outro lado, a estimativa \ingles{Late} tem atraso de 12 horas e é compilada com mais dados, o que a torna mais precisa.

\indent Atualmente, o produto \ingles{\acrshort{MERGE}} já faz a interpolação das observações das estações de superfície com dados do \ingles{\acrshort{GPM-IMERG}-Late} (figura \ref{fig:merge_obs24}). Essa atualização do \ingles{\acrshort{MERGE}} tem a inclusão de aproximadamente 2500 dados observados (figura \ref{fig:obs_prec24}) e a exlusão de Viés das estimativas de precipitação por modelos satelitais, em comparação à versão anterior \cite{MERGEatual}

\indent O produto é gerado e disponibilizado (figura \ref{fig:merge24}) operacionalmente pelo \acrshort{CPTEC}/\acrshort{INPE}, no formato \ingles{\acrfull{grib} (.grib2)}. Estão disponíveis dados horários, diários e climatológicos de precipitação, podendo ser acessados pelo seguinte endereço eletrônico: \url{http://ftp.cptec.inpe.br/modelos/tempo/MERGE/}.

\begin{figure}[htbp]
    \centering
    \caption{Produto \ingles{MERGE} e estações meteorológicas usadas na interpolação. Visualização dos dados durante o solstício de inverno de 2024.} %LEGENDA DA IMAGEM GLOBAL
    \label{fig:merge_obs24}
    \subfloat[Número estações meteorológicas disponíveis na América do Sul para interpolação do produto. \label{fig:obs_prec24}]{
    \includegraphics[width = 0.45 \textwidth]{PREC_OBS_20240620.png}
    }\hfill
    \subfloat[Produto \ingles{MERGE} no tamanho original, disponível para a América do Sul. \label{fig:merge24}]{
    \includegraphics[width = 0.45 \textwidth]{MERGE_DAILY_20240620.png}
    }\\
    \small{Fonte: \acrshort{CPTEC}/\acrshort{INPE}, \citeauthor{MERGEatual} (2024).}
\end{figure}

\subsection{\ingles{South American Mapping of Temperature (SAMeT)}}

\indent O produto \ingles{\acrshort{SAMeT}} também é de alta resolução (0,05º), com gradeamento de 5km² e sendo a interpolação de estimativas do modelo \ingles{\acrfull{ERA5}} [\ingles{\acrfull{ECMWF}}] e regressão linear simples com \ingles{\acrfull{GTOPO30}}, além de observações de estações (convencionais e automáticas) de temperatura a 2 metros (figura \ref{fig:samet_obs24}).
 
\indent Pelo fato de a reanálise ter atraso de cinco dias, o produto é inicialmente gerado com dados de observação e com modelos numéricos de previsão. Assim que a reanálise se torna disponível, o produto é gerado novamente com dados integrais da reanálise (modelo \ingles{\acrshort{ERA5}}) interpolados a dados de estações meteorológicas (figura \ref{fig:obs_temp24}). Essa combinação, entre reanálise e observação, traz uma acurácia maior ao produto, principalmente em locais de acentuada topografia \cite{Rozante2021SAMeT}.

\indent O produto é gerado e disponibilizado (figura \ref{fig:samet24}) operacionalmente pelo \acrshort{CPTEC}/\acrshort{INPE}, no formato \ingles{\acrfull{nc} (.nc)}. Estão disponíveis dados diários e climatológicos de temperaturas mínima, média e máxima, podendo ser acessados pelo seguinte endereço eletrônico: \url{http://ftp.cptec.inpe.br/modelos/tempo/SAMeT/}.

\begin{figure}[htbp]
    \centering
    \caption{Produto \ingles{SAMeT} e estações meteorológicas usadas na interpolação. Visualização dos dados durante o solstício de inverno de 2024.} %LEGENDA DA IMAGEM GLOBAL
    \label{fig:samet_obs24}
    \subfloat[Número estações meteorológicas disponíveis na América do Sul para interpolação do produto. \label{fig:obs_temp24}]{
    \includegraphics[width = 0.45 \textwidth]{TMIN_OBS_20240620.png}
    }\hfill
    \subfloat[Produto \ingles{SAMeT} no tamanho original, disponível para a América do Sul. \label{fig:samet24}]{
    \includegraphics[width = 0.45 \textwidth]{SAMeT_TMIN_20240620.png}
    }\\
    \small{Fonte: \acrshort{CPTEC}/\acrshort{INPE}, \citeauthor{Rozante2021SAMeT} (2024).}
\end{figure}

\subsection{\textcolor{red}{FALTA CITAR GFS/CFS!}}

\indent ...




%As citações diretas com menos de três linhas “devem estar entre aspas e devem mostrar entre parênteses o ano e a página da obra consultada.” (AUTOR, ano, página). Já as citações com mais de três linhas devem ser recuadas da margem esquerda em 4 cm, tamanho da fonte 10, espaçamento simples e texto sem aspas (ABNT, 2002, p. 2).


%\subsubsection{Subtítulo Quaternário}

%\section{Subtítulo Secundário 3a}
