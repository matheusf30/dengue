\chapter{Fundamentação Teórica}

Esta seção apresenta uma breve contextualização sobre questões sanitárias e climatológicas. Foi elaborada através de  fichamento do material científico acessado: livros, artigos, \ingles{sites}, boletins e informes oficiais, dentre outros. Essa revisão bibliográfica servirá como base para conciliar diferentes áreas do conhecimento, assim como facilitar o entendimento de terminologias abordadas no presente estudo.

\section{Caracterização do Clima e Ambiente}

\subsection{Climatologia e Sistemas Meteorológicos de Santa Catarina (SC)}

\indent A palavra 'clima' vem do radical grego 'klíma', semanticamente relacionado à 'inclinação'. Esse significado advém da inclinação dos raios solares em relação ao globo terrestre, variando conforme a posição geográfica. Logo, os gregos foram os primeiros a propor uma classificação climática, baseada apenas nas latitudes \cite{AtlasClimaticoSul}.

\indent Esse entendimento evolui conforme o avanço da ciência, sendo hoje compreendido o clima como "a sucessão habitual dos diversos tipos de tempo que compõem o cenário atmosférico de uma região ao longo de um período de pelo menos trinta anos" (\acrlong{SC}, \citeyear{AtlasSCnatureza}, pg-75 ).

\indent Segundo \citeonline{reboita2010}, \acrlong{SC} se encontra no setor R4 de precipitação, sendo similar ao restante do sul do Brasil, ao Paraguai e ao Uruguai. No entanto, utilizando a classificação de \citeonline{MERGEatual}, o Estado catarinense está compreendido no setor R1 de precipitação.

\indent Ambos os autores defendem um total pluviométrico elevado e bem distribuído ao longo do ano. Esse comportamento se deve a atuação de sistemas meteorológicos na região. A presença de vórtices ciclônicos e cavados em altos níveis sobre a costa oeste da América do Sul favorece o desenvolvimento de ciclones e \acrlong{FF} nesses setores (R1 \cite{MERGEatual} e R4 \cite{reboita2010}). Dessa forma, condições ciclogenéticas e frontogenéticas se desenvolvem sobre \acrlong{SC}.

\indent Os sistemas frontais, que se deslocam do sul em sentido Noroeste-Nordeste, passando pela Argentina e adentrando o Brasil, causam precipitação atuando diretamente na região ou fornecendo condições para o desenvolvimento de \acrfull{LI} pré-frontais \cite{reboita2010}.

\indent As condições favoráveis à ocorrência de tempestades aumentam quando o processo convectivo é acoplado a um ou mais sistemas de instabilidade, como vórtices ciclônicos, baixas térmicas, \acrfull{JBN} ou a passagem de \acrlong{FF} pelo litoral sul do Brasil (\acrlong{SC}, \citeyear{AtlasSCnatureza}).

DESCREVER \acrlong{CCM};

DESCREVER Sistema Clicônico em Médios Níveis (vírgula invertida);

DESCREVER \acrlong{ZCAS};

DESCREVER  sistemas de circulação locais (brisas).

\indent De acordo com o Atlas Geográfico de \acrlong{SC} (\citeyear{AtlasSCnatureza}), os maiores índices de precipitação anuais são observados na região Oeste, Grande Florianópolis e Litoral Norte. Entretanto, no Litoral Sul, entre Araranguá e Laguna, os totais anuais desses índices são os menores. O efeito sazonal da dinâmica dos sistemas atmosféricos, sobretudo pelas altas pressões, e a influência da orografia, principalmente por planícies e planaltos, resultam em variações na distribuição espacial de elementos climáticos, como temperatura e precipitação, que se observa menores temperaturas médias em regiões de maiores altitudes e tendência com maiores acumulados de precipitação próximos às serras. Em consonância com \citeonline{Oliveira2022Sistema}, a definição do sistema de anticiclone é tida como centros de alta pressão atmosférica.

\begin{citacao}
"...A maior influência ocorre pelo anticiclone Migratório Polar, centro de ação da Massa de Ar Polar, úmida quando a trajetória é marítima (mPa) e seca quando a trajetória é continental, sempre fria (mPc); pelo anticiclone do Atlântico, centro de ação da Massa Tropical Atlântica (mTa), quente e úmida e pela depressão do Chaco, que é o centro de ação da Massa Tropical Continental (mTc), quente e seca" (\acrlong{SC}, \citeyear{AtlasSCnatureza}, pg-77).
\end{citacao}

\indent "Os elementos que constituem as condições momentâneas de tempo passam a ser denominados elementos climáticos quando utilizados para fins de estudos relacionados ao clima" (\acrlong{SC}, \citeyear{AtlasSCnatureza}, pg-75).

\subsection{Aspectos Bióticos e Abióticos}

\indent Cabe destacar que, antes da promulgação da Constituição Federal de 1988, o Estado brasileiro estabelece a \acrfull{PNMA}, que tem a preservação, melhoria e recuperação da qualidade ambiental propícia à vida como objetivo principal, conforme a luz da lei. Também define recursos ambientais, englobando fauna, flora, elementos da biosfera, solo, subsolo, mar territorial, águas interiores, estuários, águas subterrâneas, águas superficiais e atmosfera  \cite{BRASIL1981LeiPNMA}.

\indent Essa mesma política ainda apresenta definição para meio ambiente, sendo "o conjunto de condições, leis, influências e interações de ordem física, química e biológica, que permite, abriga e rege a vida em todas as suas formas" \cite{BRASIL1981LeiPNMA}. Logo, meio ambiente é um conceito amplo que, dependendo da doutrina, vai além dos aspectos naturais.  

\indent O Estado catarinense está situado no sul do Brasil, tendo limites com o Estado do Paraná ao norte e, ao sul, com o Estado do Rio Grande do Sul. Esses três Estados  formam a região Sul do Brasil. \acrlong{SC} também tem limites a leste com o oceano Atlântico e a oeste com a República Argentina.  O Estado ocupa aproximadamente 1\% da território nacional e  cerca de 16\% da área total da Região Sul (\acrlong{SC}, \citeyear{AtlasSCterritorio}).

\indent É possível notar diferentes paisagens em \acrlong{SC}; essas, criadas por formas de relevos variadas. Sob esses aspectos, pode-se observar os principais compartimentos de relevo no Estado catarinense: Planície (Costeira), Serras (do Mar, do Tabuleiro-Itajaí e Geral), Planaltos (de São Bento do Sul, de Lages, dos Campos Gerais e Dissecado Rio Iguaçu-Rio Uruguai), Patamares (de Mafra, do Alto Rio Itajaí e da Serra Geral) e Depressão (da Zona Carbonífera) (\acrlong{SC}, \citeyear{AtlasSCnatureza}).

\indent O relevo define a grande divisão regional de \acrlong{SC}: região do
litoral e encostas, e região do planalto. Somado a esse relevo singular, o clima explica o processo colonial tardio do Estado catarinense (\acrlong{SC}, \citeyear{AtlasSCpopulacao}).

\indent Como verificado no Atlas Geográfico de \acrlong{SC} (\citeyear{AtlasSCnatureza}), hipsometria do Estado catarinense é muito diversa, variando de mais de 1.800 metros ao zero, no nível do mar. Esse comportamento ocorreu pelo soerguimento epirogenético durante o período Terciário, que afetou toda a porção leste do continente Sul Americano, também por conta de processos erosivos e a diferentes resistências de rochas.

\indent O  Atlas Geográfico de \acrlong{SC} (\citeyear{AtlasSCnatureza}) cita que a zona costeira de \acrlong{SC} varia de 0 a 200 metros, atingindo 400 metros em regiões próximas ao litoral, no baixo curso do rio Itajaí-Açú e no fundo do vale do rio Uruguai e seus afluentes. A maior parte da bacia hidrográfica do Itajaí se encontra entre 400 e 800 metros, assim como parte do vale do rio Uruguai e regiões do Meio Oeste catarinense.

\indent Altitudes entre 800 e 1.200 metros são mais comuns em \acrlong{SC}, sendo decorrentes de serras, escarpas, planaltos e patamares. Nessas altitudes se encontram os divisores de água do Estado catarinense. "Por causa destas altitudes, as temperaturas são mais amenas e o clima se torna mais propício para o cultivo de frutas de clima temperado" (\acrlong{SC}, \citeyear{AtlasSCnatureza}). "Altitudes maiores do que 1.200 m são pouco comuns em Santa Catarina, [...] as elevadas altitudes permitem, em condições meteorológicas favoráveis, a ocorrência de neve..." (\acrlong{SC}, \citeyear{AtlasSCnatureza}).

\indent  O Atlas Geográfico de \acrlong{SC} (\citeyear{AtlasSCnatureza}) também comenta que remanescentes de cobertura primária ainda são existentes, porém são pequenos grupamentos e estão em locais de difícil acesso. Parte dessa fragmentação, também em coberturas secundárias, ocorre devido ao avanço agropecuário em \acrlong{SC} e o  processo de urbanização, principalmente, na segunda metade do século XX.

\indent A cobertura vegetal catarinense é biodiversa, resultante de variedades de climas, relevo e solos. A vegetação litorânea é composta por restinga e mangue. As coberturas florestais ocorrem nas encostas de serras e vales de vertente atlântica (floresta ombrófila densa), nos planaltos (floresta ombrófila mista) e vales do rio Uruguai e afluentes (floresta estacional decidual subtropical) (\acrlong{SC}, \citeyear{AtlasSCnatureza}).

\indent Em elevadas altitudes, acima de 1.500 mestros, a cobertura vegetal é composta por floresta nebular e campos de altitude. Também ocorre vegetação campestre numa altitude aproximada de 800 metros, tendo precipitação bem distribuída ao longo do ano e médias de temperatura inferiores a 15 C. "Um tipo de floresta muito típico são os Faxinais, que aparecem em altitudes entre 700 e 1.200 metros [...] são associações mistas de espécies da mata Pluvial com elementos da floresta de Araucária" (\acrlong{SC}, \citeyear{AtlasSCnatureza}).

\subsection{População Catarinense e Contexto Histórico}

\indent Como \citeonline{Lages2004Territorios} argumentam, "território é o espaço apropriado por um ator, sendo definido e delimitado por e a partir de relações de poder, em suas múltiplas dimensões". Os próprios autores ainda definem a noção de espaço, que representa uma abstração maior, tendo o território como produto da intervenção e do trabalho dos atores sobre determinado espaço. "Tal como a cultura, o território não é rígido, mas produto da dinâmica histórica das sociedades" (\acrlong{SC}, \citeyear{AtlasSCpopulacao}).

\indent Como ressaltado no Atlas Geográfico de \acrlong{SC} (\citeyear{AtlasSCpopulacao}) o território era ocupado por povos originários há milhares de anos, muito antes da colonização.

\indent 

\indent São conhecidos três grupos indígenas, denominados: Guarani, Kaingang e Xokleng. "Os Xokleng no Alto Vale do Itajaí voltaram a se definir Laklãnõ recentemente". Com o processo histórico de colonização, houve redução  territorial desses povos, principalmente por uma política indigenista de aldeamento (\acrlong{SC}, \citeyear{AtlasSCpopulacao}).

\indent O grupo Guarani possuía uma agricultura mais sofisticada e ocupava terras baixas, principalmente do litoral sul brasileiro e várzeas da bacia do rio da Prata. "Há mais de um milênio foi iniciada a ocupação Guarani na costa Atlântica sul, inclusive de Santa Catarina, cujas datações mais antigas estão entre os anos 930 e 980 em Imbituba" (\acrlong{SC}, \citeyear{AtlasSCpopulacao}).

\indent Os Laklãnõ (Xokleng) se deslocavam no território entre mata Atlântica e florestas de araucárias do planalto. Os Kaingang tinham as terras altas como território, isso por associação à mitologia do grande dilúvio e por predomínio da mata de araucária, do qual proporcionava a base alimentar (\acrlong{SC}, \citeyear{AtlasSCpopulacao}).

\begin{citacao}
"Os indígenas resistiram a esse processo de esbulho do seu patrimônio e de violência
física. O conceito de resistência deve ser amplo, não no sentido de luta armada, mas
de estratégias cotidianas de enfrentamento aos ditames da política indigenista. [...] Atualmente, é importante salientar que os povos indígenas habitantes do Brasil, incluindo os povos que vivem em \acrlong{SC}a, estão em crescimento demográfico, sendo uma característica desse processo o reconhecimento da sua própria identidade" (\acrlong{SC}, \citeyear{AtlasSCpopulacao}, pg-47).
\end{citacao}

\indent Assim descrito por \acrlong{SC} (\citeyear{AtlasSCpopulacao}), os estados da região sul brasileira, assim como o território catarinense, foram conquistados e colonizados de forma tardia, a partir do século XVII. "Certamente dentre os elementos referentes ao quadro natural, o clima é o mais ressaltado como um dos determinantes na estruturação tardia da formação meridional brasileira" (\acrlong{SC}, \citeyear{AtlasSCpopulacao}). Seria muito difícil, para o universo colonial marcado pela tropicalidade, oferecer aos interesses comerciais da Metrópole Portuguesa uma produção que atendesse, abaixo do Trópico de Capricórnio, ao predomínio do clima subtropical.

\indent Nesse mesmo século, XVII, fazia-se necessário um sistema defensivo litorâneo da ilha de \acrlong{SC}. Porém, a partir do século XVIII, houve a incorporação do litoral catarinense ao sistema econômico colonial português, o que levou a migração de açorianos e madeirenses para o território. Também iniciando nesse período, surgiram novos povoados na estrada das tropas para pouso e descanso, sendo anexados ao território catarinense no século XIX (\acrlong{SC}, \citeyear{AtlasSCterritorio}). "... A exploração das terras do planalto começa com os paulistas no século XVIII, instalando a pecuária extensiva nas manchas de campos naturais de Lages, Curitibanos e Campos Novos" (\acrlong{SC}, \citeyear{AtlasSCpopulacao}).

\begin{citacao}
"Santa Catarina vivencia seu processo de conquista e colonização, tardio em cerca
de cem anos frente ao Nordeste e Sudeste brasileiros, com povos fruto do crescimento demográfico da Colônia, caracterizando um processo migratório interno, no sentido norte-sul. Isto tanto no século XVII (Litoral), quanto ainda no XVIII (Planalto), mesmo século em que também se inicia a migração externa com populações vindas de além-mar, no caso da 2ª ocupação do litoral catarinense. A partir do século XIX tal processo imigratório se intensifica com a ocupação dos vales atlânticos e encostas pelos imigrantes europeus, excedente populacional  relativo a uma Europa em transformação frente ao avanço e desenvolvimento do modo de produção capitalista" (\acrlong{SC}, \citeyear{AtlasSCpopulacao}, pg-22). 
\end{citacao}

\indent Por meados do século XIX, algumas regiões européias, já em fase industrial, atravessam um período de crise econômica e turbulência política. Esse cenário de crise foi provocado pela expansão do capitalismo, o que levou ao aprofundamento e ampliação do processo de expropriação, responsável pelo próprio crescimento demográfico europeu. É nesse contexto que as transformações em áreas de capitalismo
tardio levam milhões de emigrantes a cruzar o oceano em busca de novas oportunidades. "...Foram criados, principalmente por alemães e italianos, vários núcleos rurais e urbanos em áreas correspondentes aos vales atlânticos e encostas dos rios Itajaí, Cachoeira, Cubatão, Tijucas, Tubarão, Urussanga e Araranguá" (\acrlong{SC}, \citeyear{AtlasSCpopulacao}).

\indent Também pela metade do século XIX, o Governo Imperial criou mais uma colônia militar, a do Chapecó. "Implantada em 1882 em área correspondente hoje ao município de Xanxerê". Além de fazer a guarda da fronteira, também protegia a porção mais a oeste do caminho das tropas, como as primeiras colônias nas encostas do planalto a leste. No início do século XX, instalaram-se pequenas propriedades rurais policultoras (Joaçaba, Chapecó e Concórdia), por conta da expansão demográfica do Rio Grande do Sul, principalmente com a chegada da ferrovia São Paulo-Rio Grande do Sul (1908-1910), no vale do rio do Peixe (\acrlong{SC}, \citeyear{AtlasSCpopulacao}).

\indent Também devemos tomar nota da população de origem africana que, assim como os povos indígenas, foi escravizada durante os períodos colonial e imperial. "Das origens, na África, houve predomínio da região Centro-Ocidental (Congo, Angola, Benguela), com alguma presença de africanos orientais (Moçambique) e ocidentais (Mina, Jejes)". Há registros que apontam para o crescimento gradual de africanos escravizados e indígenas administrados, desde o século XVII, com a fundação de São Francisco do Sul, Desterro e Laguna. Ainda assim, a captura e a escravização de indígenas estava condenada por parte das autoridades judiciais da capitania de São Paulo na década de 1720. Fato esse que intensificou o tráfico transatlântico de escravos africanos para os territórios portugueses na América nos séculos XVIII e XIX, marcando o início de uma virada demográfica. Parte desses afrodescendentes foi utilizada como mão-de-obra para contrução de fortificações no litoral, estações baleeiras, trabalho em engenhos de farinha de mandioca, açúcar, aguardente e na lida com o gado no planalto (\acrlong{SC}, \citeyear{AtlasSCpopulacao}). 

\begin{citacao}
"Desde o século XIX, a autoimagem de Santa Catarina é a de uma província ou um estado europeu, uma singularidade que nos distanciaria do restante do Brasil, marcado pela mestiçagem. A associação de “branco” com “europeu”, e deste com “civilização” e “progresso”, é um traço persistente do senso comum. Embora seja
inegável tanto a presença quanto a participação de descendentes de europeus na
constituição demográfica e econômica de Santa Catarina, o efeito colateral é a invisibilização da população de origem africana" (\acrlong{SC}, \citeyear{AtlasSCpopulacao}, pg-79).
\end{citacao}

\indent Não coincidentemente, o Brasil tem registro de dengue desde o século XIX, relatado em literatura por epidemias no Rio de Janeiro (1846) e em São Paulo (1852) \cite{Valle2015Dengue}.

\section{Contextualização sobre Saúde}

\subsection{Saúde para Ciências Exatas e da Terra}

\begin{center}
\textcolor{red}{CONSIDERAÇÕES AQUI!}\\
\textcolor{red}{JOÃO: Geomedicina (Geografia médica, Geografia da saúde, medicina social, espaço).}\\
\end{center}

\indent Sendo o \acrfull{PCAM} vinculado a Ciências Exatas e da Terra, particularmente dentro das GeoCiências, de acordo com sistematização das áreas de conhecimento da \acrfull{CAPES} (\citeyear{CAPES_Tabela_Conhecimento}) e publicizado pelo \acrfull{CNPq} \citeyear{CNPq_Tabela_Conhecimento}), deve-se atentar para nomenclaturas e abordagens sobre saúde.

\indent Segundo \citeonline{Geomedicine2012Davenhall}, a Geomedicina é definida como uma nova área de inteligência médica que utiliza dados e infraestrutura espaciais para benefício da prória saúde humana; para \citeonline{Geomedicine1990JulLag}, é uma área da ciência que observa influências e relações do meio ambiente com a distribuição espacial de agravos em saúde de homens e animais.

\indent DESCREVER Geografia médica.

\indent DESCREVER Geografia da saúde.

\indent Medicina social/espaço. RENE AREIA, Associação de Medicina Social da Bélgica "enfatizou a importância do social, político e cultural na origem e persistência de doenças epidêmicas".

\begin{center}
\textcolor{red}{CONSIDERAÇÕES AQUI!}\\
\indent \textcolor{red}{JOÃO: Tópico para Saúde Coletiva/Pública e Saúde Única\\(SEPARAR EM TÓPICOS DISTINTOS)}\\
\end{center}

%\subsection{Saúde: Coletiva e Única}

\subsection{Saúde Coletiva}

\indent Durante a antiguidade grega, precisamente no século V a.C., Hipócrates defendia a idéia de que um ambiente saudável estava ligado com a própria saúde do população. Ele é considerado o pai da Medicina e adotava a visão de Saúde Pública integrada ao ambiente \cite{CFMVSaude}.

\indent De acordo com \citeonline{Foucault1990Microfisica}, a passagem de uma medicina coletiva para uma medicina privada não se deu com o capitalismo, muito pelo contrário, primeiro socializou o corpo enquanto força de trabalho. O controle social sobre os indivíduos começa no corpo, com o corpo, pois este é uma realidade biopolítica, sendo a medicina parte de sua estratégia. Porém, a socialização do corpo só ocorreu em última etapa do surgimento da Medicina Social, que teve sua formação a partir de três (3) etapas: Medicina de Estado, Medicina Urbana e Medicina da Força de Trabalho.

\indent A primeira etapa ocorreu durante o século XVIII na Alemanha, momento em que a burguesia, em queda, oferecia seus serviços à organização Estado. Nesse período, a partir da medicina, o Estado era responsável por fazer observações e registros, não apenas de nascimento e mortalidade, como também de morbidade. Ainda no final do século XVIII, na França, com a urbanização da cidade, era necessário ordenamento territorial por conta da densidade demográfica e da heterogeneidade política e sanitária. Assim, a medicina fazia ordenação do espaço urbano, realizando análise de perigo em áreas de aglomero, principalmente cemitérios, assim como controlando a circulação de água e ar, tanto de afluentes como de efluentes. A última etapa se deu na Inglaterra ao final do século XIX, com o desenvolvimento industrial, e por consequência, do proletariado. "A partir do momento em que o pobre se beneficia dos sistema de assistência, deve, por isso mesmo, se submeter a vários controles médicos" \cite{Foucault1990Microfisica}.

\indent Nesse momento, a ciência reformulava as relações entre o homem e suas condições de vida, incitando a medicina como assunto público, devendo intervir na vida política e social.
Essas idéias são voltadas para as reformas da saúde e alguns princípios básicos se
tornariam parte integrante do discurso sanitarista, sendo: que a saúde da população é de interesse societário e cabe à sociedade proteger e assegurar a saúde de seus membros; que as condições sociais e econômicas têm um impacto crucial sobre o processo saúde-doença, devendo ser estudadas cientificamente; que as medidas a serem tomadas para a proteção da saúde são tanto sociais como médicas \cite{TratadoSaudeColetiva}.

\indent Ainda, sob as palavras de \citeonline{Foucault1990Microfisica}, saúde e salubridade não deveriam ser vistos como iguais. "Salubridade é a base material e social capaz de assegurar a melhor saúde possível dos indivíduos. [...] Salubridade e insalubridade são o estado das coisas e do meio enquanto afetam a saúde"  \cite{Foucault1990Microfisica}.

\indent Adotando o conceito de saúde da Carta Magna para Saúde Mundial, desde planejamento de criação até a execução atual da \acrfull{OMS}, tem-se saúde como um estado de completo bem-estar físico, mental e social, e não apenas como a ausência de doença ou enfermidade \cite{OMS2024S1, ParranHEALTH}.

\indent  Na metade do século XX, concomitantemente ao final da Segunda Grande Guerra (1939-1945), quando se aumenta a utilização de antibióticos e técnicas cirúrgicas, reconsolida-se a atenção médica individualizada, repensando sobre conceitos de saúde. Nesse período, a \acrfull{OPAS} apoia a ampla discussão crítica sobre o modelo biomédico da medicina, momento em que há criação de departamentos de medicina preventiva e social nos projetos pedagógicos. Muitas análises sociais, demográficas e políticas percorreram a história da saúde pública, percebendo vínculo às políticas de saúde que se desenvolveram, na Europa e nas Américas, e trouxeram em seus conteúdos as especificidades de cada contexto histórico e suas circunstâncias \cite{TratadoSaudeColetiva}.

\indent Ao final dos anos 70, para \citeonline{TratadoSaudeColetiva}, dá-se inicio a Saúde Coletiva, com fundação de associação própria a nível nacional e assumindo posição técnico-política, que teria a concepção de saúde como direito do cidadão e dever do Estado. A Saúde Coletiva encontra ruptura com a concepção de Saúde Pública, não aceitando o próprio discurso biológico, sendo um mosaico entre as ciências sociais e humanas, a epidemiologia, a política e o planejamento. A presença das ciências sociais e humanas foi de grande importância e consolidou como fundamentais para a compreensão dos processos da vida, saúde, doença e morte.

\begin{citacao}
\indent "...Tornou-se difícil um consenso acerca de sua conceituação. Em realidade, a partir do momento em que se foram firmando as formas de tratar o coletivo, o social e o público, caminhou-se para entender a saúde coletiva como um campo estruturado e estruturante de práticas e conhecimentos, tanto teóricos como políticos" (\citeauthor{TratadoSaudeColetiva}, \citeyear{TratadoSaudeColetiva}, pg-27).
\end{citacao}


\indent Durante a década de 90, a Lei Orgânica da Saúde foi sancionada no Brasil. Essa lei dispõe sobre as condições  para a promoção, proteção  e  recuperação  da  saúde,  a  organização  e  o  funcionamento  dos  serviços correspondentes, além de também dispor sobre o Sistema Único de Saúde. Está expresso em seu segundo artigo que saúde é um direito fundamental e prover condições para seu pleno exercício é dever do Estado \cite{BRASIL1990LeiSUS}. Seu terceiro artigo tem a inclusão, por meio de lei, de fatores determinantes e condicionantes, ficando o texto assim:

\begin{citacao}
"Os níveis de saúde expressam a organização social e econômica do País, tendo a saúde como determinantes e condicionantes, entre outros, a alimentação, a moradia, o saneamento básico, o meio ambiente, o trabalho, a renda, a educação, a atividade física, o transporte, o lazer e o acesso aos bens e serviços essenciais "
(\citeauthor{BRASIL2013LeiDeterminantesCondicionantesSaude}, \citeyear{BRASIL2013LeiDeterminantesCondicionantesSaude}, pg-1).
\end{citacao}

\indent Nesse mesmo artigo da Lei Orgânica da Saúde, o parágrafo único continuou inalterado, mantendo o seguinte texto: "Dizem respeito também à saúde as ações que, por força do disposto no artigo anterior, se destinam a garantir às pessoas e à coletividade condições de bem-estar físico, mental e social" \cite{BRASIL1990LeiSUS}.


% \indent Abordagens articuladas à saúde: Foucault, Butler e Laqueur

% \indent Maria Cecília Minayo (2001), reforma sanitária.

% \indent Convergências epistemológicas entre a Bioética e saúde coletiva discutidas por Junges
% e Zoboli (2012)

% \indent Segundo Marsiglia (2013) o uso do termo “coletivo” no campo da saúde coletiva é de
% fundamental importância, haja vista que
% https://www.teses.usp.br/teses/disponiveis/5/5137/tde-09082017-100757/publico/MarceloJosedeSouzaeSilva.pdf


\subsection{Saúde Única}

\indent Ainda na segunda metade do século XX, exatamente em 1968, a \acrfull{OPAS} convoca a \acrfull{RIMSA}, realizando proteção e promoção da saúde humana e animal por meio da cooperação técnica em saúde pública veterinária, além de manter enfoque multissetorial \cite{S1_OPAS_OMS}. O termo Medicina Única foi concebido pelo médico-veterinário Calvin W. Schwabe (1927-2006) em sua obra "\ingles{Veterinary Medicine and Human Health}” em 1984, reforçando a importância da junção entre saúde humana, animal e ambiental \cite{CFMVSaude}. Aos poucos, o mundo foi despertando para a importância da interação entre saúde e meio ambiente.

\indent Durante a Conferência das Nações Unidas sobre Meio Ambiente e Desenvolvimento no Rio de Janeiro em 1992, conhecida como EcoRio92, discutiram-se as bases para avançar esforços conjugados quanto à colaboração entre setores de saúde e meio ambiente. Logo, em 1995, foi adotada a Carta Pan-Americana sobre Saúde e Meio Ambiente no Desenvolvimento Humano Sustentável \cite{S1_OPAS_OMS}, porém a evolução do termo para Saúde Única ocorreu no século XXI \cite{CFMVSaude}. 

\indent No ano de 2004, em Nova Iorque, especialistas se reuniram para debater e propor ações em resposta à potencial dinâmica de doenças entre humanos, animais domésticos e fauna silvestre. Naquela época, principalmente, alertar sobre ebola, influenza aviária, doença crônica debilitante dos cervídeos, encefalopatia espongiforme bovina, dentre outras. Esses encontros resultaram na elaboração de prioridades e recomendações, chamadas de "Princípios de Manhattan" durante o simpósio "Um Mundo, Saúde Única", que apresentavam visão holística e caráter interdiscipinar sobre saúde, dando ênfase na prevenção de epidemias/epizootias e na conservação íntegra dos ecossistemas \cite{ManhattanPrinciples2004}.

\indent Parte dessas recomendações eram para reconhecer a ligação da saúde entre animais domésticos, silvestres e sociedade; alertar para o uso do solo e da água; reduzir o consumo de carne de caça e de alimentos com baixa inocuidade; aumentar investimentos em pesquisas sobre saúde de maneira transdisciplinar e multissetorial; investir em educação, não apenas acadêmica, mas da população. A saúde deveria ser entendida e planejada na interface homem-animal-ecossistema, atingindo, assim, a Saúde Única \cite{ManhattanPrinciples2004}.

\indent Inicialmente, em 2008, a abordagem sobre Saúde Única era feita de forma Tripartite, entre a \acrfull{OMS}, \acrfull{FAO} e \acrfull{OMSA}. Esta última foi fundada como Gabinete Internacional de Epizootias (\ingles{\acrfull{OIE}}). Após a junção dessas três organizações, em junho de 2021, foi proposta uma quarta parte para compor essa aliança, com a inclusão do \acrfull{PNUMA}. Com uma gestão Quadripartite, desenvolve-se o \acrfull{SGISU}, que tem por objetivo melhorar a inteligência em Saúde Única, antecipando avisos e auxiliando na gestão de risco de ameaças à saúde global \cite{S1Quadripartite}.

%"É importante notar que a definição do preâmbulo sugere uma motivação idealista em favor da igualdade universal, que era novo, especialmente em muitos governos europeus, mesmo após a Segunda Guerra, e, além disso, o preâmbulo ligava saúde com termos como paz. Inspirados pelo postulados da medicina social, saúde pública não devia ser um produto isolado do resto da vida social, mas um processo intrínseco de desenvolvimento social", afirmou Marcos Cueto. --- %https://www.coc.fiocruz.br/index.php/pt/todas-as-noticias/319-saude-internacional-e-as-origens-da-oms\\
%DECRETO Nº 26.042, DE 17 DE DEZEMBRO DE 1948 --- https://www2.camara.leg.br/legin/fed/decret/1940-1949/decreto-26042-17-dezembro-1948-455751-publicacaooriginal-1-pe.html\\
%https://edisciplinas.usp.br/pluginfile.php/5733496/mod_resource/content/0/Constitui%C3%A7%C3%A3o%20da%20Organiza%C3%A7%C3%A3o%20Mundial%20da%20Sa%C3%BAde%20%28WHO%29%20-%201946%20-%20OMS.pdf\\

\indent Ainda para a \acrshort{OMS} (\citeyear{OMS2024S1}), Saúde Única é uma abordagem na interface humano-animal-ambiental para se alcançar melhores resultados em saúde coletiva. Vários setores que se intercomunicam, especialmente para o controle de zoonoses, promoção de segurança alimentar e combate a resistência a antibióticos, tendo interdependência e ligação entre saúde humana e saúde animal por meio de saúde de ecossistemas; para a \citeonline{FAO2022} (\acrshort{FAO}), há sinergismo entre essas políticas e estratégias, enquanto é entendido pelo \citeonline{CFMVSaude} (\acrshort{CFMV}) como uma união indissociável.

\begin{citacao}
"Saúde Única é um enfoque colaborativo, multidisciplinar e multissetorial que pode abordar as ameaças à saúde na interface homem-animal-ambiente no âmbito subnacional, nacional e internacional, com o objetivo final de obter resultados de saúde ótimos reconhecendo as interconexões entre pessoas, animais, plantas e meio ambiente. Essa  interface, um aspecto definidor de Saúde Única, consiste num continuum de interações entre pessoas, animais e meio ambiente que permite a transmissão entre espécies de patógenos zoonóticos e emergentes" (\citeauthor{S1_OPAS_OMS}, \citeyear{S1_OPAS_OMS}, pg-2).
\end{citacao}


\indent No atual ano, 2024, passa a vigorar a lei que institui o Dia Nacional de Saúde Única, a ser celebrado no dia 3 de novembro, "com o objetivo de conscientizar a sociedade sobre a relação indissociável entre as saúdes animal, humana e ambiental" \cite{BRASIL2024LeiS1}. O próprio Ministério da Saúde reconhece o termo Saúde Única, e sugere a sinonímia "Uma Só Saúde", com a mesma abordagem integrada que reconhece a conexão entre a saúde humana, animal, vegetal e ambiental. Esse Ministério é tido como o órgão do Poder Executivo Federal responsável pela organização e elaboração de planos e políticas públicas voltados para a promoção, a prevenção e a assistência à saúde dos brasileiros \cite{MinisterioSaudeS1}.

\indent Sobre a relação entre saúde humana, animal, vegetal e ambiental, o \citeonline{MinisterioSaudeS1} elenca quatro (4) tópicos, sendo: Doenças zoonóticas e novas epidemias/pandemias; Resistência aos antimicrobianos; Segurança alimentar e segurança dos alimentos; e Biodiversidade, mudanças climáticas e saúde.

% https://www.gov.br/saude/pt-br/assuntos/saude-de-a-a-z/u/uma-so-saude

% https://www.gov.br/saude/pt-br/assuntos/saude-de-a-a-z/u/uma-so-saude/doencas-zoonoticas

% https://www.gov.br/saude/pt-br/assuntos/saude-de-a-a-z/u/uma-so-saude/biodiversidade

\indent Neste mesmo ano, 2024, o \acrshort{CFMV} emite a Portaria nº 115, que cria e nomeia a \acrfull{CNSU} do Conselho Federal. Essa Portaria considera  a necessidade de criar políticas que assegurem à população qualidade em saúde, sendo o médico-veterinário um  profissional importante na equipe multidisciplinar dessa área. Também é incluído, no conceito de Saúde Única, a conexão com saúde vegetal, além da abordagem integrada entre saúde humana, animal e ambiental. Ao \acrshort{CNSU} fica atribuído: analisar, sugerir, articular e propor políticas de Saúde aos órgãos competentes, além de demais competências e representações perante o \acrshort{CFMV} (\acrlong{CFMV}, \citeyear{CFMV2024PORTARIA}).

\indent Para a \citeonline{ONUODS22} (\acrshort{ONU}), como objetivo de desenvolvimento sustentável, deve ser assegurado uma condição de vida saudável e promoção de bem-estar para todas e todos, em todas as idades; além de combater doenças veiculadas pela água e outras doenças transmissíveis. Também é tido de suma importância a integração das medidas da mudança do clima nas políticas, estratégias e planejamentos nacionais. Ainda segundo a \acrshort{ONU} (\citeyear{ONUODS22}) é necessário implementar medidas para evitar a introdução de espécies exóticas e reduzir significativamente o impacto dessas espécies invasoras.

\subsection{Doenças Zoonóticas}

\indent No século XIX, o médico patologista Rudolf L K. Virchow (1821-1902) foi um dos primeiros a utilizar o termo Zoonose e afirmava que “entre a medicina animal e a medicina humana não havia divisórias; e que nem deveria haver” \cite{CFMVSaude}.

\indent Como citado por \citeonline{HumanAnimalInterface}, a própria interface homem-animal apresenta características versáteis e dinâmicas. Para esses autores, a ligação homem-animal é caracterizada por muitos atributos adquiridos durante o próprio processo de evolução humana como espécie e o desenvolvimento do homem como sociedade, porém se mantém em processo dinâmico contínuo.

\indent O principal atributo é a co-evolução patogênica herdada pelo homem como espécie animal, além de outros atributos importantes, tais como: comportamento humano durante o processo de domesticação de espécies; a perda do comportamento nômade-coletor para se tornar agricultor e produtor de alimentos; amplos processos migratórios, de colonização e comercialização;  e aos próprios processos de urbanização, industrialização e globalização \cite{HumanAnimalInterface}.

\indent A \acrshort{OMS} (\citeyear{WHO2020Zoonoses}) considera zoonoses as doenças infecciosas trasmitidas, via ingestão hídrica ou sólida, pelo meio ambiente ou por contato direto, entre humanos e animais-não-humanos. Elas representam o principal problema em saúde coletiva no mundo, além de corresponderem a 75\% das doenças infecciosas emergentes em humanos. Seus agentes etiológicos podem ser de caráter bacteriano, viral, parasitário ou envolver agentes não convencionais.

\indent Pensando nos animais como potenciais reservatórios para a transmissão da dengue, \citeonline{Dengue_Animals_Gwee_2021} em uma revisão sistemática, constataram que aproximadamente 10\% dos animais testados apresentaram positividade para \acrshort{DENV}, tanto PCR quanto sorologia. Analisando as sorologias, tem-se positividade para: primatas não-humanos (26,94\%), morcegos (7,04\%), marsupiais (3,41\%) e roedores (0,67\%). Ao se analisar os RT-PCR, tem-se positividade para:  marsupiais (9,54\%), morcegos (3,04\%), roedores (2,78\%) e primatas não-humanos (0,32\%).

\indent Em outra revisão sistemática e meta-analítica, \citeonline{Aldana2024DengueAnimals} encontraram positividade sorológica \acrshort{DENV} em: suínos (49\%), primatas não-humanos (29\%), morcegos (10\%), equinos (11\%), aves (8\%), bubalinos (7\%), roedores (2\%) e ovinos (1\%). Também analisaram resultados positivos por RT-PCR em morcegos, apresentando positividade em 6\% dos animais avaliados.

\indent Ambos os estudos reconhecem os primatas não-humanos como potenciais hospedeiros em ciclos enzoóticos para amplificação da transmissão, porém constataram que nem todos as espécies de primatas não-humanas são sucetíveis. Também entram em acordo ao encontrar resultados positivos por métodos moleculares em morcegos e dão ênfase ao dizer que o vírus da dengue pode infectar diversas espécies de animais, incentivando estudos relacionados \cite{Aldana2024DengueAnimals, Dengue_Animals_Gwee_2021}.

\indent Em estudo de \citeonline{DengueDog2017First}, uma pequena quantidade de cães domiciliados apresentaram positividade para dengue em amostras de sangue testadas por RT-PCR, isolamento viral e replicação em cultura celular. Desses cães, a maioria era de região urbana e parte deles era domiciliado internamente na residência, tendo contato com o ambiente externo apenas para passeio. Os autores não descartam a possibilidade de cães serem reservatórios para o vírus, mas alertam necessidade de mais estudos.

\subsection{Ecoepidemiologia}

\indent DESCREVER TRÍADE EPIDEMIOLÓGICA: Agente, Hospedeiro e Ambiente.

% https://www.who.int/news-room/fact-sheets/detail/dengue-and-severe-dengue

\indent A dengue é uma doença viral de grande importância na saúde coletiva e sua forma epidêmica tem se adaptado ao ambiente urbano durante centenas de anos. \cite{ArboviralTransmission}.

\indent Assim como citado por \citeonline{Fiocruz2010Atlas}, o vírus da dengue (\acrfull{DENV}) faz parte da família \latim{Flaviviridae} e possui estrutura genética formada por ácido ribonucleico de fita simples e sentido positivo (\acrfull{ssRNA+}). Essa mesma autora simplifica a estrutura do vírion em 3 partes: material genético, capsídeo viral e envelope.

\indent Aprofundando um pouco mais a análise da estrutura viral, percebe-se que o capsídeo tem composição proteíca (proteína-C) e que o envelope é formado por bicamada fosfolipídica com proteínas de membrana (proteína-M) e proteínas de envelope (proteína-E). Essas proteínas de membrana que conferem a infectividade. O tamanho do capsídeo é de 30nm, enquanto o vírus como um todo pode chegar a 65 nm \cite{Fiocruz2010Atlas}. 

\citeonline{ArboviralTransmission} também comentam que muitas pandemias tem sido atribuídas a capacidade de alguns vírus \acrshort{ssRNA+} adaptarem e incluírem o ser humano como hospedeiros. Esses mesmo autores também citam que, para a dengue, essas mutações resultaram em adaptação ao mosquito urbano.

\indent A dengue é uma arbovirose, ou seja, é uma doença vetorizada por artrópodes, logo tem os seguintes mosquitos como transmissores no ciclo urbano peridomiciliar: \latim{Aedes aegypti} e \latim{Aedes albopictus} \cite{ArboviralTransmission}.

\indent Conforme adotado pela \citeonline{SBPGlossario} (\acrshort{SBP}), vetor é um artrópode, molusco ou veículo que transmite um parasito entre dois hospedeiros, que albergam o parasito; sendo vetor biológico quando o agente etiológico apresenta parte do ciclo biológico, multiplicando-se ou se desenvolvendo, no próprio animal vetor.

\subsection{Emergências Ambientais}

\indent Para iniciar, ao se tratar de riscos relacionados a desastres, refere-se ao
potencial de ocorrer algo danoso para a sociedade. Para a \acrfull{Cobrade}, os desastres podem ser categorizados em naturais ou tecnológicos. Desses, os desastres naturais ainda são agrupados em cinco (5): geológico, hidrológico, meteorológico, climatológico e biológico. Esse último desastre é dividido em dois (2) tipos: epidemias e insfestações/pragas [sic]. As epidemias são divididas em quatro (4) subtipos,  quanto ao agente etiológico: viral, bacteriano, parasítico ou fúngico. Especificamente para as doenças virais, esses desastres são interpretados como "aumento brusco, significativo e transitório da ocorrência de doenças infecciosas geradas por vírus", tendo 1.5.1.1.0 como código \acrshort{Cobrade} \cite{GIRD}.

\indent Segundo o \citeonline{CFMVSaude}, 80\% dos patógenos com potencial para bioterrorismo são zoonóticos. Além de impactos na saúde humana, por serem altamente letais ou incapacitantes, as zoonoses apresentam grande impacto na economia. Dessa maneira, o controle sanitário precisa ser efetivado, principalmente sob a ótica da saúde pública, embora muitas vezes as doenças zoonóticas sejam negligenciadas.

\indent \citeonline{Cubas2014Tratado} confirmam a importância para as sociedades contemporâneas sobre os impactos de mudança climática global, de emergências ambientais e de modificação antrópica dos ecossistemas naturais. Os efeitos desses impactos se estendem à integridade do meio ambiente e à saúde coletiva, além de afetar diretamente a economia. As avaliações sistemáticas de vulnerabilidade socioambiental e de saúde em relação às modificações de larga escala do meio ambiente são urgentes, em seu sentido biofísico. 

\begin{citacao}
"Alguns estudos demonstram como a dinâmica ecoepidemiológica destes agravos tem sido afetada por mudanças ambientais. [...] Inventários faunísticos e microbiológicos, cenários de clima e seus efeitos em ecossistemas, e a implantação de sistemas permanentes e eficazes de monitoramento bioclimático, são aspectos a serem considerados pelos gestores públicos e pelas comunidades científica e conservacionista" \cite[pg-2325]{Cubas2014Tratado}.
\end{citacao}

%\begin{citacao}
%“Infere-se que, em primeiro lugar, o poder de ação dos médicos-veterinários engloba a monitoração e análise dos indicadores epidemiológicos. Em um segundo momento, conclui-se que os médicos-veterinários, vinculados ou não ao Programa Nacional de Controle da Dengue (PNCD), têm uma importante responsabilidade na geração de propostas de prevenção da dengue, chikungunya e zika” \cite[pg-12]{Silva2016O}.
%\end{citacao}
%\textcolor{red}{INCLUIR\\Regulamento Sanitário Internacional (2005) e Emergência de Saúde Pública de
%Importância Nacional (Decreto 7.616/2011).}
%\begin{center}
%\textcolor{red}{CONSIDERAÇÕES AQUI!}\\ 
%\end{center}

\indent O \acrfull{RSI} representou um marco para a Saúde Pública entre países, tendo como propósito a prevenção, proteção, controle e resposta à saúde pública frente a riscos internacionais na propagação de doenças. Esse regulamento tem princípios planejados para evitar interferências não necessárias em tráfego e comércio, pondo regramento no trânsito de cargas e pessoas \cite{Brasil2005RegulamentoSI}.

\indent O controle de pontos de entrada e saída, passagens de fronteiras terrestres, portos e aeroportos se dará respeitando a dignidade, os direitos humanos e as liberdades fundamentais das pessoas. Também poderão ser exigidos documentos de saúde, tais como: certificados de vacinação ou outras medidas profiláticas; declaração marítima de saúde; e certificados de controle sanitário da embarcação \cite{Brasil2005RegulamentoSI}.

\indent Alguns anos após a publicação do \acrshort{RSI}, o decreto sobre \acrfull{ESPIN} é sancionado no Brasil, sendo declarado emergência em situações que demandam urgentemente de medidas para prevenção, controle e contenção de riscos, danos e agravos à saúde pública no território brasileiro \cite{Brasil2011ESPIN}.

\indent Será declarada emergência a nível nacional em virtude de eventos epidemiológicos, desastres ou desassistência à população. Em relação a situações epidemiológicas, serão consideradas as que apresentem risco de disseminação nacional, representem a reintrodução de agentes infecciosos erradicados, sejam produzidos por agentes infecciosos inesperados, apresentem gravidade elevada ou extrapolem a capacidade de resposta da direção estadual do \acrfull{SUS} \cite{Brasil2011ESPIN}.

\indent DECRETO Nº 11.358, DE 1º DE JANEIRO DE 2023 \acrfull{SVS} \acrfull{SVSA}

\indent \acrfull{PNVS}\cite{Brasil2018PNVS} Vigilância em Saúde Ambiental \cite{GuiaVigSaúde22}.

\indent Levando em consideração a saúde única, devemos adotar um caráter preventivo frente a catástrofes, não apenas  intervindo de forma mitigadora para interromper processos crônicos estabelecidos de degradação ambiental, recuperação de ecossistemas e manejo de populações comprometidas \cite{Cubas2014Tratado}.
%\indent Em se tratando de populações comprometidas, é importante salientar que pessoas impactadas em desastres optam por não sair de suas residências para irem aos abrigos sem seus animais de estimação. Por isso, a Defesa Civil do Estado do \acrlong{PR} conta com uma rede de atendimento a animais em situação de emergência, tendo participação do \acrlong{CRMV} do \acrlong{PR} (\acrshort{CRMV}/\acrshort{PR}), \acrfull{CEGRADE} alinhada com a Comissão Nacional de Desastres do \acrfull{CFMV}, Secretarias Estaduais e Municipais, além de voluntários \cite{CRMVPR22CEGRADE}.

\subsection{Situação Epidemiológica Atualizada da Dengue em Santa Catarina (SC)}

%https://www.who.int/news-room/fact-sheets/detail/dengue-and-severe-dengue

\begin{center}
\textcolor{red}{CONSIDERAÇÕES AQUI!}\\
\indent \textcolor{red}{JOÃO: Situação Epidemiológica atualizada da dengue (referenciar o ano de que está se falando)\\ATUALIZAR}\\
\end{center}

\indent As informações atualizadas e expostas estão em acordo com a \citeonline{Informe8DiveSE12/23}.

\indent Embora haja redução (10,7\%) na quantidade de focos identificados, se comparados ao mesmo período do ano passado (28.114 focos do mosquito Aedes aegypti em 2022 e 25.105 focos em 2023), há aumento (8,97\%) de municípios infestados, em relação ao mesmo período de 2022 (132 municípios infestados anteriormente e 145, em 2023). Por ora, há 4.769 casos confirmados, o que significa uma diminuição de 39,55\% em relação a 2022 (12.058 confirmações até a décima segunda semana epidemiológica daquele ano), porém continuam 11.124 casos suspeitos atualmente.

\indent Outro ponto importante são os casos autóctones (transmissão dentro de \acrlong{SC}), que teve registro de 4.078 casos, distribuídos em 59 municípios. Desses casos, 361 amostras foram processadas para pesquisa viral pelo \acrfull{LACEN} de \acrlong{SC}, sendo: 98,34\% delas (355/228) identificadas o sorotipo DENV1 e 1,66\%, DENV2.

\indent Desses autóctones, o município de Palhoça concentra 1.593 casos e detém a maior taxa de incidência de dengue atualmente (673,18 casos/100 mil habitantes), sendo o primeiro município catarinense a atingir o nível de epidemia em 2023, na décima semana epidemiológica \cite{Informe6DiveSE10/23}. Nessa décima segunda semana epidemiológica, dois municípios atingiram o nível de epidemia: Palhoça e Saudades.

\indent Cabe ressaltar que o município de Joinville, o mais populoso do Estado de \acrlong{SC}, obteve alta taxa de incidência de casos autóctones em 2022, com 3.628,15 casos/100 mil habitantes (totalizando 21.423 casos autóctones) \cite{Informe31DiveSE52/22}.

\section{Descrição das Bases de Dados}

\subsection{Departamento de Informática do Sistema Único de Saúde (DataSUS)}

O \acrshort{DataSUS} está inserido atualmente na pasta da Secretaria Executiva do Ministério da Saúde e a \acrfull{Dive} é vinculada à Superintendência de Vigilância em Saúde, da Secretaria de Estado da Saúde de Santa Catarina. O \acrshort{DataSUS} disponibiliza informações, via plataforma TABNET, que subsidiam  análises objetivas da situação sanitária para tomada de decisões baseadas em evidências e elaboração de programas de ações para promoção da saúde. Atualmente, dados como condições de vida, acesso a serviços, qualidade da atenção à saúde, morbidade, incapacidade, além de fatores ambientais, passaram a ser métricas utilizadas na construção de Indicadores de Saúde, que se traduzem em informação relevante para a quantificação e a avaliação das informações em saúde \cite{TABNETMinisterio}.

\subsection{Sistema de Informação de Agravos de Notificação (Sinan)}

O \acrfull{Sinan} foi implantado em 1993, porém apenas em 1998 o \acrfull{Cenepi} coordenou e organizou as três esferas do governo por meio da Portaria \acrshort{Funasa}/Ministério da Saúde n.º 073 de 9/3/98, regulamentando  e tornando obrigatória a alimentação regular dessa base de dados nacional. A partir de 2003, com a criação do \acrfull{SVS}, essa secretaria passa a ser responsável pelo sistema. Apesar de o \acrshort{Sinan} ser alimentado por notificações  de casos de doenças e agravos que constam da lista nacional de doenças de notificação compulsória, é facultado aos estados e municípios incluir outros agravos de saúde importantes em sua região \cite{SINANWEB, SINAN07Ministerio}.

\indent O próprio \acrshort{Sinan} faz a publicização, em documento ofical ao final do ano, das semanas epidemiológicas do próximo ano. Essas, como mencionado pelo \citeonline{SemanaEpidemio}, foram convencionadas internacionalmente e são contadas a partir de domingo a sábado. "A primeira semana do ano é aquela que contém o maior número de dias de janeiro e a última a que contém o maior número de dias de dezembro."

\subsection{Vigilantos}

O Vigilantos é o sistema informatizado \ingles{on-line} desenvolvido em 2012 pelo Estado de \acrlong{SC}. Os municípios podem inserir dados relativos ao \latim{Aedes aegypti}, facilitando e agilizando o acesso de todos os níveis a essa informação. Além de permitir registro, é possível fazer análise das informações de vigilância e controle vetorial através de dois módulos: Módulo Focos e Módulo Programa de Controle da Dengue \cite{Vigilantos}.

\subsection{Instituto Brasileiro de Geografia e Estatístic (IBGE)}

O \acrshort{IBGE} tem como missão, resumidamente: identificar e analisar o território; realizar a contagem da população; mostrar como a economia evolui através do trabalho e da produção das pessoas, revelando ainda como elas vivem.  O Instituto é o principal provedor de dados e informações do Brasil, que atendem às necessidades da sociedade civil, assim como dos órgãos das esferas governamentais federal, estadual e municipal \cite{IBGE22}, \cite{IBGE23prev}.

% \subsection{Banco de Dados Meteorológicos do Instituto Nacional de Meteorologia (Bdmep/Inmet)}

% O \acrshort{Bdmep/Inmet} reune dados meteorológicos diários em formato digital, sendo atualizados a cada 90 dias e disponibilizados via \ingles{internet}. Essas séries históricas foram coletadas das várias estações meteorológicas convencionais da rede de estações do próprio \citeonline{INMET22}, que está vinculado diretamente ao \acrfull{MAPA}.

\subsection{\ingles{Merging Technique (MERGE)}}

\indent De acordo com \citeonline{Rozante2010MERGE} o produto \acrshort{MERGE} é a interpolação de dados do satélite \ingles{\acrfull{TRMM-TMPA}} e a precipitação observada por estações de superfície. Esse produto é de alta resolução (0.1º), com gradeamento de 10km² e disponibilização de dados diários a partir de junho de 2000, e horários desde 2010.

\indent Comentado por \citeonline{IMERG, TRMM-TMPA}, os dados do \ingles{\acrshort{TRMM-TMPA}} foram descontinuados e é fortemente sugerido utilizar dados do \ingles{\acrfull{GPM-IMERG}}. Esses dados são disponibilizados a cada 30 minutos e as estimativas são executadas em duas etapas: antecipada (\ingles{Early}) e tardia (\ingles{Late}). A estimativa \ingles{Early} tem atraso de apenas quatro horas e é compilada com dados do momento. Por outro lado, a estimativa \ingles{Late} tem atraso de 12 horas e é compilada com mais dados, o que a torna mais precisa.

\indent Atualmente, o produto \ingles{\acrshort{MERGE}} já faz a interpolação das observações das estações de superfície com dados do \ingles{\acrshort{GPM-IMERG}-Late} (figura \ref{fig:merge_obs24}). Essa atualização do \ingles{\acrshort{MERGE}} tem a inclusão de aproximadamente 2500 dados observados (figura \ref{fig:obs_prec24}) e a exlusão de Viés das estimativas de precipitação por modelos satelitais, em comparação à versão anterior \cite{MERGEatual}

\indent O produto é gerado e disponibilizado (figura \ref{fig:merge24}) operacionalmente pelo \acrshort{CPTEC}/\acrshort{INPE}, no formato \ingles{\acrfull{grib} (.grib2)}. Estão disponíveis dados horários, diários e climatológicos de precipitação, podendo ser acessados pelo seguinte endereço eletrônico: \url{http://ftp.cptec.inpe.br/modelos/tempo/MERGE/}.

\begin{figure}[htbp]
    \centering
    \caption{Produto \ingles{MERGE} e estações meteorológicas usadas na interpolação. Visualização dos dados durante o solstício de inverno de 2024.} %LEGENDA DA IMAGEM GLOBAL
    \label{fig:merge_obs24}
    \subfloat[Número estações meteorológicas disponíveis na América do Sul para interpolação do produto. \label{fig:obs_prec24}]{
    \includegraphics[width = 0.45 \textwidth]{PREC_OBS_20240620.png}
    }\hfill
    \subfloat[Produto \ingles{MERGE} no tamanho original, disponível para a América do Sul. \label{fig:merge24}]{
    \includegraphics[width = 0.45 \textwidth]{MERGE_DAILY_20240620.png}
    }\\
    \small{Fonte: \acrshort{CPTEC}/\acrshort{INPE}, \citeauthor{MERGEatual} (2024).}
\end{figure}

\subsection{\ingles{South American Mapping of Temperature (SAMeT)}}

\indent O produto \ingles{\acrshort{SAMeT}} também é de alta resolução (0,05º), com gradeamento de 5km² e sendo a interpolação de estimativas do modelo \ingles{\acrfull{ERA5}} [\ingles{\acrfull{ECMWF}}] e regressão linear simples com \ingles{\acrfull{GTOPO30}}, além de observações de estações (convencionais e automáticas) de temperatura a 2 metros (figura \ref{fig:samet_obs24}).
 
\indent Pelo fato de a reanálise ter atraso de cinco dias, o produto é inicialmente gerado com dados de observação e com modelos numéricos de previsão. Assim que a reanálise se torna disponível, o produto é gerado novamente com dados integrais da reanálise (modelo \ingles{\acrshort{ERA5}}) interpolados a dados de estações meteorológicas (figura \ref{fig:obs_temp24}). Essa combinação, entre reanálise e observação, traz uma acurácia maior ao produto, principalmente em locais de acentuada topografia \cite{Rozante2021SAMeT}.

\indent O produto é gerado e disponibilizado (figura \ref{fig:samet24}) operacionalmente pelo \acrshort{CPTEC}/\acrshort{INPE}, no formato \ingles{\acrfull{nc} (.nc)}. Estão disponíveis dados diários e climatológicos de temperaturas mínima, média e máxima, podendo ser acessados pelo seguinte endereço eletrônico: \url{http://ftp.cptec.inpe.br/modelos/tempo/SAMeT/}.

\begin{figure}[htbp]
    \centering
    \caption{Produto \ingles{SAMeT} e estações meteorológicas usadas na interpolação. Visualização dos dados durante o solstício de inverno de 2024.} %LEGENDA DA IMAGEM GLOBAL
    \label{fig:samet_obs24}
    \subfloat[Número estações meteorológicas disponíveis na América do Sul para interpolação do produto. \label{fig:obs_temp24}]{
    \includegraphics[width = 0.45 \textwidth]{TMIN_OBS_20240620.png}
    }\hfill
    \subfloat[Produto \ingles{SAMeT} no tamanho original, disponível para a América do Sul. \label{fig:samet24}]{
    \includegraphics[width = 0.45 \textwidth]{SAMeT_TMIN_20240620.png}
    }\\
    \small{Fonte: \acrshort{CPTEC}/\acrshort{INPE}, \citeauthor{Rozante2021SAMeT} (2024).}
\end{figure}

\subsection{\ingles{Global Forecast System (GFS)}}

\indent Segundo a Administração Oceânica e Atmosférica Nacional (estadunidense) \citeyear{GFS} (\ingles{\acrfull{NOAA}}), o modelo \ingles{\acrshort{GFS}} é a regressão da reanálise do modelo \ingles{\acrfull{MERRA2}}; e atualizado para a versão 16.0 em março de 2020. Os dados são disponibilizados quatro vezes ao dia, com até 16 dias de previsão, tendo variações: dados horários para até as primeiras 120h (0,25º) e dados a cada três horas para até 16 dias (0,5º).

\indent Esse modelo preditivo é sintetizado pelo Centro Nacional para Previsões Ambientais daquele mesmo país (\ingles{\acrfull{NCEP}}), que faz acoplamento de quatro outros modelos (atmosférico, oceanográfico, terrestre e de gelo marinho), retornando valores preditivos de temperaturas, precipitação, umidade relativa do ar e do solo, radiação, relativas ao vento, assim como concentração de ozônio \cite{GFS24NOAA}.


%As citações diretas com menos de três linhas “devem estar entre aspas e devem mostrar entre parênteses o ano e a página da obra consultada.” (AUTOR, ano, página). Já as citações com mais de três linhas devem ser recuadas da margem esquerda em 4 cm, tamanho da fonte 10, espaçamento simples e texto sem aspas (ABNT, 2002, p. 2).


%\subsubsection{Subtítulo Quaternário}

%\section{Subtítulo Secundário 3a}
