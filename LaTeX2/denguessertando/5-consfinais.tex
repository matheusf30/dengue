\chapter{Considerações Finais}

\indent Neste projeto foram analisados dados entomológicos desde 2012 e epidemiológicos desde 2014, ambos com registros em semanas epidemiológicas. Os dados climatológicos foram adquiridos, em frequência de registros diária, desde 2000; porém apenas se pode analisar relação entre todos os dados ao padronizar o tempo em semanas epidemiológicas.

\indent Logo, utilizar sazonalidade em semanas epidemiológicas para dados climatológicos é uma excelente forma prática para fazer relações com o setor da saúde, principalmente ao se pesquisar doenças de notificação e transmitidas por vetores. Assim, o estudo possibilita auxiliar os gestores públicos, principalmente da área da saúde, visando aprimorar diretrizes de combate, controle e prevenção da dengue, como do vetor biológico \latim{Aedes} sp. Também possibilita base e comparação para estudos futuros, padronizando dados ambientais em semanas epidemiológicas.

\indent Para análise de relação entre os comportamentos entomo-epidemiológicos e climatológicos, o uso de produtos observacionais e de reanálise, \ingles{\acrshort{SAMeT}} e \ingles{\acrshort{MERGE}}, mostraram-se eficientes e oportunos em pesquisas que envolvam análises climatológicas e epidemiológicas, principalmente por doenças vetorizadas por mosquitos do gênero \latim{Aedes}.

\indent Como a utilização destes produtos, é possível obter grande volume de dados com altas frequências (horário/diário), disponíveis para a malha territorial do Estado de \acrlong{SC}. Desse modo, estudos entomo-epidemiológicos envolvendo questões climáticas podem regionalizar comportamentos que não seriam observados em dados mensais ou estações pontuais de dados meteorológicos. 

\indent Os resultados mostram a regionalização na distribuição de focos de \label{Aedes} sp. e casos de dengue no Estado de \acrlong{SC}, seguindo o padrão de regionalização climatológica apresentado por \citeonline{Guerra2023Regionalizacao}. Região Oeste, Vale do Itajaí e a faixa litorânea apresentam condições climatológicas favoráveis para o desenvolvimento dos mosquitos do gênero \latim{Aedes}. Logo, são regiões com notório risco para epidemias, não apenas, mas principalmente por dengue. Deve-se atentar para o aumento nos casos de Zika e Chikungunya nos últimos anos, também vetorizados por mosquitos do gênero \latim{Aedes} \cite{Valle2015Dengue}.

\indent As regiões serranas e de planaltos servem como limitadores ambientais na distribuição dos vetores da dengue no Estado catarinense, uma vez que não apresentam condições ideais para o desenvolvimento do mosquitos, como citado por \citeonline{AedesTemp}. Por essa razão, essas regiões apresentaram baixa compilação de modelos, uma vez que há baixa ocorrência de vetor (focos de \latim{Aedes} sp.) ou vírus circulante (casos de dengue).

\indent As distribuições de sazonalidade com maior ocorrência de casos de dengue ocorrem entre os meses de março e abril, como evidenciado por \citeonline{Valle2015Dengue} e por \citeonline{Drumond2020Dinamica} para outras regiões brasileiras. Esse padrão caracteriza o comportamento biológico do vetor, aumentando desenvolvimento durante o período de verão austral, e o comportamento epidemiológico da doença, corroborando a nomenclatura de doença tropical \cite{Valle2015Dengue}.

\indent Há comportamento epidemiológico comum entre os municípios analisados (Florianópolis, Itajaí, Joinville e Chapecó), ao correlacionar casos de dengue e limiares de temperaturas, evidenciando correlações médias a altas negativas ao retroceder no tempo. Esse fato corrobora a distribuição sazonal nesses municípios em \acrlong{SC}, quando há aumento no número de casos de dengue ao diminuir as temperaturas no outono austral. 

\indent Deve-se tomar nota para o município de Chapecó, assim como a região oeste catarinense, onde foi evidenciado antecipação do pico de casos registrados, estando em acordo ao observado por \citeonline{Guerra2023Regionalizacao} com a antecipação das temperaturas elevadas durante a primavera. Esse comportamento antecipado em Chapecó também foi observado por \citeonline{Matiola2019ANALISE} em análise a nível municipal.

\indent Durante esse estudo, a cobertura vacinal para o vírus da dengue no Estado de \acrlong{SC}, assim como a soltura de mosquitos \latim{Aedes} sp. laboratorialmente infectados com a bactéria do gênero \latim{Wolbachia} no município de Joinville, houve pouca influência para o processo de aprendizado de máquina durante a modelagem computacional, uma vez que essas ações de controle do vírus da dengue são recentes no território catarinense. Espera-se acompanhamento dos estudos para monitorar a dinâmica entomo-epidemiológica em \acrlong{SC} ao longo dos anos.

\indent Ao passo que as temperaturas globais atingem e ultrapassam os recordes máximos registrados, também é esperado que o comportamento entomo-epidemiológico tenha adaptação frente a esse novo cenário de mudança climática global. Por esse ,otivo, deve-se analisar e monitorar tais alterações de comportamento, tanto climáticos quanto entomo-epidemiológicos, no Estado de \acrlong{SC} nos próximos anos.

\indent Por apresentar ascendência na série de anomalias estacionárias entomológicas e epidemiológicas, não ocorrendo o mesmo com as séries de anomalias estacionárias climatológicas, estudos aprofundados devem ser realizados para confirmação na mudança do comportamento entomo-epidemiológico. Especula-se influência do confinamento (\latim{lockdown}) durante a pandemia de coronavírus da síndrome respiratória aguda grave 2 (\acrfull{SARS-CoV-2}).

\indent Por haver comportamentos confluentes e regionais entre dados entomo-epidemiológicos  e climatológicos, é possível utilizar o tempo meteorológico para previsão de aumento de focos de \latim{Aedes} sp. e casos de dengue, como já ocorre com outras ferramentas preditivas e pesquisas relacionadas \cite{Infodengue_2018, ForecastingDengueBrazil2019, Relatorio_Infodengue_2023}.

\indent Nesse estudo, a utilização de aprendizado de máquina por modelos \ingles{Random Forest} apresentou melhor resposta de predição, quando comparado à modelagem por \acrfull{RNAM}. Possivelmente um maior volume de dados e uma melhor manipulação do processo trariam melhores resultados no aprendizado de máquina por \acrshort{RNAM}. Em ambas as modelagens, \ingles{Random Forest} para focos de \latim{Aedes} sp. e casos de dengue, as variáveis mais importantes foram as próprias variáveis dependentes retroagidas no tempo.

\indent Ao analisar os histograma dos erros, é possível perceber adensamento próximo ao zero (0), indicando melhores resultados de \acrfull{r2}. Ainda assim, municípios como Florianópolis e Joinville apresentaram médias desviadas à direita em relação à mediana, sugerindo modelos que subestimam os valores de previsão durante o processo de aprendizado de máquina.

\indent Finalmente, o \acrlong{PTT} demonstra capacidade preditiva de 14 dias (2 semanas epidemiológicas) para os casos de dengue no Estado de \acrlong{SC}, através de modelos \ingles{Random Forest} executados de forma automatizada em frequência semanal. Esses valores preditos podem ser visualizados através de cartografias temáticas, disponíveis em semanas epidemiológicas. A primeira (1ª) semana de previsão ocorre utilizando dados climatológicos da associação \ingles{\acrshort{SAMeT}-\acrshort{MERGE}}. Para previsão da segunda semana (2ª), utilizam-se dados de previsão do modelo \ingles{\acrshort{GFS}}.



%http://www.linse.ufsc.br/~fernando/dicas/figuras.html
