\chapter{Metodologia}

\section{Materiais}

Para organização, os dados serão agrupados em três divisões distintas: 

\begin{alineas}
    \item \acrfull{DEE}: Elementos sanitários relacionados tanto ao vetor (dados entomológicos) quanto ao hospedeiro (dados epidemiológicos) e provenientes de bancos de dados oficiais. Os valores são relativos à quantidade de focos de \latim{Aedes} sp., disponibilizados oficialmente pela \acrshort{Dive}. Em relação à quantidade de casos de dengue, esses dados foram obtidos de plataformas \ingles{on-line}: TABNET-\acrshort{Sinan}-\acrshort{DataSUS} (base nacional) e TABNET-\acrshort{Sinan}-\acrshort{Dive} (base estadual); 
    
    \item \acrfull{DEC}: Informações referentes a variáveis meteorológicas/climatológicas (temperatura - mínima, máxima e média - e precipitação) e provenientes de banco de dados oficiais (reanálise e produtos de reanálise):  \ingles{\acrshort{GFS}}, \ingles{\acrshort{MERGE}} e \ingles{\acrshort{SAMeT}};

    \item \acrfull{DGR}: Informações relacionadas aos aspectos geográficos atualizados para o momento atual, provenientes do \acrshort{IBGE}. 
\end{alineas}

\indent Os dados de precipitação acumulada a superfície (mm), são armazenados no produto \ingles{\acrshort{MERGE}},  sendo adquiridos dados diários a partir de junho de 2000.\\
\indent Os dados do \ingles{\acrshort{SAMeT}} são aferidos a dois metros (2m) da superfície em escala de temperaturas dadas em celsius (C), agrupados em médias diárias e com série histórica iniciando em janeiro de 2000.\\
\indent \textcolor{red}{FALTA CITAR GFS/CFS!}.\\
\indent ...\\
\indent Os dados referentes a focos de \latim{Aedes} sp. são contabilizados no dia do próprio registro, ocorrendo apenas de segunda a sexta, e têm início no ano de 2012.\\
\indent A série histórica dos casos de dengue começa em 2014, sendo previamente agrupados e disponilizados em semanas epidemiológicas.\\
\indent Os \acrshort{DEE} e \acrshort{DEC} foram estruturados a ponto de compartilhar a mesma escala espaço-temporal. Os dados diários foram agrupados em semanas epidemiológicas, assim como o recorte espacial englobou área de estudo (figura N), o próprio Estado catarinense.


\section{Métodos}

\indent Para isso, pretende-se dividir o estudo em etapas, relativas ao percurso de execução do próprio projeto, sendo: pré-processamento, análise estatística descritiva, modelagem preditiva, espacialização dos dados preditos e síntese do \acrfull{PTT}.\\


\subsection{Pré-processamento}

Em relação aos \acrshort{DEE}, devem ser orientados ao modo que possam ser concatenados, adicionados, uma planilha após a outra. Para tal, utilizou-se a linguagem \ingles{Python} (versão 3) \cite{python3_2009_van}  e a biblioteca \ingles{pandas} \cite{pandas_2010_scipy, pandas_2020_reback}, como pacote principal para estruturação dos dados . Outras também foram utilizadas, como: \ingles{numpy} \cite{numpy_2020_harrisarray}, para tratar dados faltantes e tipagem de variáveis; \ingles{datetime} \cite{python2_1995_van}, para padronização de todas as datas e variáveis desse tipo; e \ingles{geopandas} \cite{geopandas_2020_kelseyjordahl}, para manipular arquivos georreferenciados, extrair deles a nomenclatura padrão dos municípios do \acrshort{IBGE} e aplicar aos próprios dados.\\
\indent Os \acrshort{DEC} foram previamente tratados utilizando o \ingles{\acrfull{CDO}} \cite{CDO_2023_schulzweida}, assim, pôde-se unir dados diários para composição de meses e anos, por fim, sintetizando a série histórica. Com esse mesmo \ingles{software} se fez o primeiro recorte espacial, para o sul do Brasil, diminuindo o tamanho do arquivo principal. Para abertura e manipulação dos arquivos climáticos, na extensão \ingles{netCDF4}, utilizou-se a biblioteca \ingles{xarray} \cite{xarray_2016_v0_8_0, xarray_2017_hoyer}.\\
\indent Após isso, foram extraídos os valores dos elementos climáticos de cada município e armazenados em um novo formato de arquivo, \ingles{\acrfull{csv}} (valores separados por vírgulas), utilizando as bibliotecas \ingles{pandas}, \ingles{numpy}, \ingles{geopandas} e \ingles{shapely} \cite{shapely_2007_gillies}.\\
\indent Finalmente,  os principais conjuntos de dados (temperatura mínima, temperatura média, temperatura máxima, precipitação, focos de \latim{Aedes} sp. e casos de dengue) eram próprios arquivos estruturados em tabela dinâmica, onde as colunas eram cada município catarinense e as linhas, a serié histórica dada em semanas epidemiológicas no formato \ingles{datetime64[ns](YYYY-mm-dd)}. Logo, a equiparação entre essas variáveis era possível.


\subsection{Análise Estatística Descritiva}

\indent Enquanto a etapa de Regionalização está em andamento, optou-se pelo desenvolvimento do \acrfull{ICOb}, que é resultado do acoplamento de \acrshort{DEC} [temperatura e precipitação - MERRA2 (0,5º / 0,25º) e SAMeT(0,05º) / MERGE (0,1º)]  interagindo com o \acrfull{LACOb} por \citeonline{Matiola2020Dissertação}.\\
\indent O índice normalizado foi calculado à partir do limiar (\acrshort{LACOb}) encontrado por \citeonline{Matiola2020Dissertação}, estabelecido como 15C através da correlação de Spearman (0,74), considerando 30 dias retroativos. O método aplicado nesse trabalho subtrai o \acrshort{LACOb} da matriz de temperaturas máximas diárias (TMAX) obtidas em produtos distintos: \acrshort{GFS} e \acrshort{SAMeT}. Essa matriz resultante da subtração foi somada à própria matriz em seus valores numéricos absolutos, onde os valores abaixo do limiar foram zerados. Essa nova matriz com valores zerados foi dividida por dois (2) para definição do \acrfull{ICOb}. As equações (\ref{eq:resultante}) e (\ref{eq:icob}) ilustram o cálculo do índice.
\begin{equation}
\hspace{4,5cm} [Resultante] = TMAX - \acrshort{LACOb}
    \label{eq:resultante}
\end{equation}
\begin{equation}
\hspace{4.5cm} \acrshort{ICOb} = \frac{(abs[Resultante])+[Resultante]}{2}
    \label{eq:icob}
\end{equation}
%ESCREVER COMO SE CHEGOU NO 'idenguen' >> 'define idenguen=10*at1*idengue/'_nvmax\\

\subsection{Modelagem Preditiva}

\indent Na terceira etapa do estudo, serão definidos os  \acrfull{LACRe} entre os elementos climáticos e os resultados observados em relação aos focos de \latim{Aedes aegypti} distribuídos no Estado catarinense. Como resultado preliminar, foi utilizado o \acrshort{LACOb} desenvolvido por \citeonline{Matiola2020Dissertação}.


\subsection{Espacialização dos Dados Preditos}

\indent Para esse estudo, foi utilizado o recorte espacial durante a execução do prório \ingles{script}, sendo: longitude entre 54º5' e 57º5', ambas sul; e latitude entre 29º5' e 25º5', ambas oeste. Com esse recorte, pode-se evidenciar a totalidade do Estado de \acrlong{SC} e um pouco além de seus limites.\\
\indent Obteve-se os \ingles{shapefiles} do \acrshort{IBGE} (2022) para os limites territoriais (federal, estadual e municipais) do Estado de \acrlong{SC}.\\

\indent Durante a quarta etapa, retorna-se ao acoplamento e espacialização, que resultará na determinação dos \acrfull{ICRe}. Esses \acrshort{ICRe} serão corrigidos pelo acoplamento dos \acrshort{DEC} à interação dos \acrshort{LACRe} e ajustados na distribuição do Estado de \acrlong{SC}. Após isso, os novos limiares serão aplicados ao modelo preditivo proposto \acrshort{SAMeT}-\acrshort{MERGE}-\acrshort{GFS}.


\subsection{Refinamento com Dados Socioeconômicos e Epidemiológicos}

\indent Após a aplicação dos \acrshort{ICRe} ao modelo, pretende-se refinar o próprio modelo com a inclusão de dados Socioeconômicos do \acrshort{IBGE} e dados Epidemiológicos da \acrshort{Dive}. Desta maneira, a dinâmica epidemiológica poderá ser entendida à partir da ótica de diversas variáveis.


\subsection{Síntese do \acrfull{PTT}} 

\indent O \acrshort{PTT} resultante será: i) o sistema computacional e ii) a visualização cartográfica desses resultados, como preditivo da dinâmica regionalizada de focos de \latim{Aedes} sp. no Estado de \acrlong{SC}. O limiar temporal estipulado para previsão da incidência dos focos será de 60 dias. Os primeiros 30 dias serão determinados a partir do sistema \acrshort{SAMeT}-\acrshort{MERGE}. Os demais dias serão determinados através do modelo de previsão \acrshort{GFS}.


%\section{Conjutos de Dados Utilizados}
%sobre as arboviroses emergentes e re-mergentes (\textbf{Dengue}, Febre Amarela, Zika, Chikungunya)
%Tornar cada parte do método em objetivo específico
%ATUALIZAR, PENSANDO NO PROCESSO FUTURO\\


%A princípio, os dados de regionalização espacial encontram-se em processo de definição.




\subsection{Pré-processamento dos Dados}
\indent Para desenvolvimento do trabalho, os dados foram ajustados em mesma escala temporal, deixando-os em semanas epidemiológicas. Dessa forma, fez-se o somatório dos registros de focos de \latim{Aedes} sp. por semana epidemiológica. Mesmo tratamento foi realizado com a precipitação, retornando o acumulado de precipitação por semana epidemiológica. As temperaturas foram agrupadas de forma diferente, porém também ajustando a dados semanais. Ao final, teríamos a média das temperaturas (mínima, média e máxima) por semana epidemiológica.\\


