%%%%%%%%%%%%%%%%%%%%%%%%%%%%%%%%%%%%%%%%%%%%%%%%%%%
% conf-comandos.sty
%%%%%%%%%%%%%%%%%%%%%%%%%%%%%%%%%%%%%%%%%%%%%%%%%%%
	
\ProvidesPackage{conf-comandos}

% ----------------------------------------------------------
% Imprimir a ficha catalográfica
% ----------------------------------------------------------
\newcommand{\imprimirficha}[1]{
    \begin{fichacatalografica}
        \includepdf{#1}
    \end{fichacatalografica}
}

% ----------------------------------------------------------
% Imprimir a folha de aprovação
% ----------------------------------------------------------
 %\providecommand{\imprimirtextoaprovacao}{}
% \newcommand{\textoaprovacao}[1]{
%     \renewcommand{\imprimirtextoaprovacao}{
%         \begin{center}#1\end{center}
%     }
% } 
% \newcommand{\imprimiraprovacao}{
%     \begin{folhadeaprovacao}
%         \begin{center}
%             \ABNTEXchapterfont\SingleSpacing\bfseries\Large\MakeUppercase\imprimirtitulo
%             \vspace*{2.0cm}
            
%             \ABNTEXchapterfont\normalsize\bfseries\MakeUppercase\imprimirautor
%             \vspace*{1.0cm}
%         \end{center}
        
%         \noindent\OnehalfSpacing\imprimirtextoaprovacao
        
%         \vspace*{1.0cm}
%         \begin{center}    
%             \imprimirlocal, 19 de abril, 2021.
%         \end{center}
        
%         \noindent Banca Examinadora:
        
%         \assinatura{\imprimirorientador, Dr. Eng.} 
% %        \assinatura{\imprimircoorientador, Dr.}
%         \assinatura{Matheus Leitzke Pinto, Msc.}
%         \assinatura{Gustavo Martins de Araújo Silvano, Eng.
            
%         }
%     \end{folhadeaprovacao}
% }

% ----------------------------------------------------------
% Imprimir a folha de dedicatória
% ----------------------------------------------------------
\newcommand{\imprimirdedicatoria}[1]{
    \begin{dedicatoria}
        \vspace*{\fill}
        \begin{flushright}
        \noindent
        \textit{#1}
        \vspace*{2cm}
        \end{flushright}
    \end{dedicatoria}
}

% ----------------------------------------------------------
% Imprimir a lista de códigos
% ----------------------------------------------------------
\newcommand{\listoflistings}{
    \begin{KeepFromToc}
        \lstlistoflistings
    \end{KeepFromToc}
}

% ----------------------------------------------------------
% Imprimir a lista de abreviaturas
% ----------------------------------------------------------
\newcommand{\sortitem}[2]{%
    \DTLnewrow{sortlist}%
    \DTLnewdbentry{sortlist}{sig}{#1}% 
    \DTLnewdbentry{sortlist}{desc}{#2}% 
}

\newenvironment{sortsiglas}{%
    \DTLifdbexists{sortlist}{\DTLcleardb{sortlist}}{\DTLnewdb{sortlist}}%
}{%
    \DTLsort{sig}{sortlist}% Sort list
    \begin{siglas}%
        \DTLforeach*{sortlist}{\theDesc=desc, \theSig=sig}{\item[\theSig] \theDesc}% Print each item
    \end{siglas}%
}%

\makeatletter

% \abreviatura{abrev.}{definição}
% ret = 
\newcommand{\abreviatura}[2]{%
    \write\@auxout{\noexpand\@writefile{abrv}{\noexpand\sortitem{#1}{\xmakefirstuc{#2}}}}%
}%

% \abreviatura*{abrev.}{definição}
% ret = abrev. (definição)
\WithSuffix\newcommand\abreviatura*[2]{%
    #1 (#2)%
    \write\@auxout{\noexpand\@writefile{abrv}{\noexpand\sortitem{#1}{\xmakefirstuc{#2}}}}%
}%

% \abreviatura.{abrev.}{definição}
% ret = definição (abrev.)
\WithSuffix\newcommand\abreviatura.[2]{%
    #2 (#1)%
    \write\@auxout{\noexpand\@writefile{abrv}{\noexpand\sortitem{#1}{\xmakefirstuc{#2}}}}%
}%

% \abreviatura'{abrev.}{definição}{tradução}
% ret = 
\WithSuffix\newcommand\abreviatura'[3]{%
    \write\@auxout{\noexpand\@writefile{abrv}{\noexpand\sortitem{#1}{\textit{#2} - \xmakefirstuc{#3}}}}%
}%

% \abreviatura&{abrev.}{definição}{tradução}
% ret = tradução (definição - abrev.)
\WithSuffix\newcommand\abreviatura&[3]{%
    #3 (\textit{#2} - #1)%
    \write\@auxout{\noexpand\@writefile{abrv}{\noexpand\sortitem{#1}{\textit{#2} - \xmakefirstuc{#3}}}}%
}%

\newcommand{\imprimirlistadeabreviaturas}{%
    \begin{sortsiglas}%
        \@starttoc{abrv}%
    \end{sortsiglas}%
}%
\makeatother

% ----------------------------------------------------------
% Imprimir a lista de simbolos
% ----------------------------------------------------------
\newenvironment{sortsimbolos}{%
    \DTLifdbexists{sortlist}{\DTLcleardb{sortlist}}{\DTLnewdb{sortlist}}%
}{%
    \DTLsort{desc}{sortlist}% Sort list
    \begin{simbolos}%
        \DTLforeach*{sortlist}{\theDesc=desc, \theSig=sig}{\item[\theSig] \theDesc}% Print each item
    \end{simbolos}%
}%

\makeatletter

% \simbolo{simbolo}{definição}
% ret = 
\newcommand{\simbolo}[2]{%
    \write\@auxout{\noexpand\@writefile{asbl}{\noexpand\sortitem{#1}{\xmakefirstuc{#2}}}}%
}%

% \simbolo*{simbolo}{definição}
% ret = simbolo
\WithSuffix\newcommand\simbolo*[2]{%
    #1%
    \write\@auxout{\noexpand\@writefile{asbl}{\noexpand\sortitem{#1}{\xmakefirstuc{#2}}}}%
}%

\newcommand{\imprimirlistadesimbolos}{%
    \begin{sortsimbolos}%
        \@starttoc{asbl}%
    \end{sortsimbolos}%
}%
\makeatother

% ----------------------------------------------------------
% Imprimir a fonte da figura
% ----------------------------------------------------------
\newcommand{\indentedfont}[2][\textwidth]{%
    \captionsetup{singlelinecheck=off,width=#1}%
    \caption*{\raggedright\footnotesize\mdseries Fonte: #2.}%
}%

% ----------------------------------------------------------
% Incluindo códigos que estão em um arquivo externo
% ----------------------------------------------------------
\newcommand{\includecode}[4][c]{
    \lstinputlisting[captionpos=t,caption=#3, label=#2,escapechar={@*}, style=#1]{#4}
}
%-- \includecode[linguagem]{label}{Titulo}{arquivo}
%-- \includecode[shell]{l_olamundo}{Olá mundo em shell script}{codigos/ola.sh}




% ----------------------------------------------------------
% Comandos para organizar a escrita
% ----------------------------------------------------------
% alterar | alterado | nota | add
\newcommand{\marcador}[2]{%
%
    \def\param{#1}%
    \def\ifalterar{alterar}%
    \def\ifalterado{alterado}%
    \def\ifnota{nota}%
    \def\ifadic{add}%
%
	\ifx\param\ifalterar%
		\textcolor{red}{#2}%
	\else%
		\ifx\param\ifalterado%
	        \textcolor{blue}{#2}%
	    \else%
        	\ifx\param\ifnota%
		        \textcolor{OliveGreen}{#2}%
            \else%
                \ifx\param\ifadic%
    		        \textcolor{Fuchsia}{#2}%
                \else%
                    \textcolor{black}{#2}%
                \fi%
            \fi%
        \fi%
	\fi%
}%

\WithSuffix\newcommand\marcador*[3]{%
%
    \def\param{#1}%
    \def\ifalterar{alterar}%
    \def\ifalterado{alterado}%
    \def\ifnota{nota}%
    \def\ifadic{add}%
%
	\ifx\param\ifalterar%
		\textcolor{red}{\underline{#2}} \textcolor{Mahogany}{(#3)}%
	\else%
		\ifx\param\ifalterado%
		    \textcolor{blue}{\underline{#2}} \textcolor{NavyBlue}{(#3)}%
	    \else%
        	\ifx\param\ifnota%
        	    \textcolor{OliveGreen}{\underline{#2}} \textcolor{Green}{(#3)}%
            \else%
                \ifx\param\ifadic%
            	    \textcolor{Fuchsia}{#2} \textcolor{Mulberry}{(\underline{#3})}%
                \else%
                    \textcolor{black}{\underline{#2}} \textcolor{gray}{(#3)}%
                \fi%
            \fi%
        \fi%
	\fi%
}%

\begin{comment}
    % Abreviatura em pt
    \abreviatura{abrev.}{definição}  -> ret: 
    \abreviatura*{abrev.}{definição} -> ret: abrev. (definição)
    \abreviatura.{abrev.}{definição} -> ret: definição (abrev.)
    % Abreviatura em língua estrangeira
    \abreviatura'{abrev.}{definição}{tradução} -> ret:
    \abreviatura&{abrev.}{definição}{tradução} -> ret: tradução (abrev. - definição)
    
    \simbolo{simbolo}{definição}  -> ret: 
    \simbolo*{simbolo}{definição} -> ret: simbolo
    
    \marcador{tipo}{texto}             -> ret: texto
    \marcador*{tipo}{texto}{alteração} -> ret: texto (alteração)
        alterar     | alterado  | nota      | add
        Vermelho    | Azul      | Verde     | Roxo
    
    \indentedfont{tamanho}{texto} -> ret: Fonte: texto.
    
    \includecode[linguagem]{label}{Titulo}{arquivo}
\end{comment}


%%%%%%%%%%%%%%%%%%%%%%%%%%%%%%%%%%%%%%%%%%%%%%%%%%%
% conf-ifsc.sty
%%%%%%%%%%%%%%%%%%%%%%%%%%%%%%%%%%%%%%%%%%%%%%%%%%%

% !TEX root = monografia.tex

\ProvidesPackage{conf-ifsc}

\usepackage{conf-pacotes}
\usepackage{conf-comandos}
\usepackage{tocloft}

% ----------------------------------------------------------
% Macros
% ----------------------------------------------------------
\newcommand{\buck}{Buck \textit{interleaved}}

% ----------------------------------------------------------
% Configuração do pacote de SI
% ----------------------------------------------------------
\sisetup{detect-all} % Configura fonte do pacote siunitx
\sisetup{output-decimal-marker = {,}}
\sisetup{mode = text, detect-italic = true}

% ----------------------------------------------------------
% Configuração das figuras
% ----------------------------------------------------------
\newlength{\imagewidth}  % the name can be chosen, e.g. picwidth...
\captionsetup[figure]{font={footnotesize,bf}}
\captionsetup[table]{font={footnotesize,bf}}

% ----------------------------------------------------------
% Definicao da geometria da página - Conforme Norma para TCC IFSC
% ----------------------------------------------------------
\RequirePackage[
    inner=3.0cm,
    outer=2.0cm,
    top=3.0cm,
    bottom=2.0cm,
    head=0.7cm,
    foot=0.7cm
    ]{geometry}

% ----------------------------------------------------------
% ..........................................................
% CONFIGURAÇÕES DE PACOTES
% ..........................................................
% ----------------------------------------------------------

% ----------------------------------------------------------
% Fontes padroes de part, chapter, section, subsection e subsubsection
% ----------------------------------------------------------
\renewcommand{\familydefault}{\sfdefault}             % Fonte Arial

\renewcommand{\ABNTEXchapterfont}{\sffamily}
\renewcommand{\ABNTEXchapterfontsize}{\normalsize\scshape\bfseries}

\renewcommand{\ABNTEXpartfont}{\ABNTEXchapterfont}
\renewcommand{\ABNTEXpartfontsize}{\ABNTEXchapterfontsize}

\renewcommand{\ABNTEXsectionfont}{\ABNTEXchapterfont}
\renewcommand{\ABNTEXsectionfontsize}{\normalsize\bfseries}

\renewcommand{\ABNTEXsubsectionfont}{\ABNTEXsectionfont}
\renewcommand{\ABNTEXsubsectionfontsize}{\normalsize}

\renewcommand{\ABNTEXsubsubsectionfont}{\ABNTEXsubsectionfont}
\renewcommand{\ABNTEXsubsubsectionfontsize}{\normalsize\itshape}

%\renewcommand{\ABNTEXsubsubsubsectionfont}{\ABNTEXsubsectionfont}
%\renewcommand{\ABNTEXsubsubsubsectionfontsize}{\normalsize}



\renewcommand{\cftpartleader}{\cftdotfill{\cftdotsep}} 
\renewcommand{\tocpartapendices}{                
    \addtocontents{toc}{\vspace{-0.6cm}}         
    \addtocontents{toc}{\cftsetindents{part}{\cftlastnumwidth}{2em}}          
    \cftinserthook{toc}{A}     
}

\renewcommand{\tocpartanexos}{
    \addtocontents{toc}{\vspace{-0.6cm}}
    \addtocontents{toc}{\cftsetindents{part}{\cftlastnumwidth}{2em}}
    \cftinserthook{toc}{A}
}
 
 
 
 
% ----------------------------------------------------------
% Fontes das entradas do sumario
% ----------------------------------------------------------
\renewcommand{\cftpartfont}{\bfseries\normalsize}
\renewcommand{\cftpartpagefont}{\bfseries\normalsize}

\renewcommand{\cftchapterfont}{\bfseries}
\renewcommand{\cftchapterpagefont}{\normalsize\cftchapterfont}

\renewcommand{\cftsectionfont}{\bfseries}
\renewcommand{\cftsectionpagefont}{\cftsectionfont}

\renewcommand{\cftsubsectionfont}{\normalsize}
\renewcommand{\cftsubsectionpagefont}{\cftsubsectionfont}

\renewcommand{\cftsubsubsectionfont}{\textit}
\renewcommand{\cftsubsubsectionpagefont}{\cftsubsubsectionfont}

\renewcommand{\cftparagraphfont}{\footnotesize}
\renewcommand{\cftparagraphpagefont}{\cftparagraphfont}


% ----------------------------------------------------------
% Remover cabeçalho nas paginas pares
% ----------------------------------------------------------
\makepagestyle{parpage}
\makeevenhead{parpage}{\ABNTEXfontereduzida\thepage}{}{}
\makeoddhead{parpage}{\ABNTEXfontereduzida\rightmark}{\rule[-.3\baselineskip]{\linewidth}{.4pt}}{\ABNTEXfontereduzida\thepage}

% ----------------------------------------------------------
% Configuração de espaçamento 
% ----------------------------------------------------------
\setlength{\afterchapskip}{0.5cm}   % Espaço 1,5 após os capítulos
\setlength{\parindent}{2.0cm}       % Identação do paragrafo conforme norma para TCC IFSC
\setlength{\parskip}{0.1cm}         % Controle do espaçamento entre um parágrafo

% ----------------------------------------------------------
% Configurações do pacote backref
% ----------------------------------------------------------
%\renewcommand{\backrefpagesname}{Citado na(s) página(s):~}
%\renewcommand{\backref}{}       % Texto padrão antes do número das páginas
%\renewcommand*{\backrefalt}[4]{ % Define os textos da citação
%	\ifcase #1 
%		Nenhuma citação no texto.
%	\or
%		Citado na página #2.
%	\else
%		Citado #1 vezes nas páginas #2.
%	\fi
%}

\def\UrlLeft{}
\def\UrlRight{}

% --------------------------------------------------------------------------
% Personalização do modelo da abnTeX2 para se adequar com o modelo do IFSC
% 2017-08-23 - Segue o modelo do IFSC publicado em setembro de 2016
% --------------------------------------------------------------------------

% ----------------------------------------------------------
% Alteração da capa
% ----------------------------------------------------------
\renewcommand{\imprimircapa}{
    \begin{capa}
        \begin{SingleSpacing}
            \center
            \ABNTEXchapterfont\bfseries\ INSTITUTO FEDERAL DE EDUCAÇÃO, CIÊNCIA E TECNOLOGIA DE\\SANTA CATARINA - CÂMPUS FLORIANÓPOLIS\\DEPARTAMENTO ACADÊMICO DE SAÚDE E SERVIÇO\\CURSO DE MESTRADO PROFISSIONAL EM CLIMA E AMBIENTE
            
            \vspace*{3.0cm}
            
            \ABNTEXchapterfont\bfseries\imprimirautor
        
            \begin{vplace}[0.5]
                \begin{center}
                    \ABNTEXchapterfont\SingleSpacing\bfseries\imprimirtitulo
                \end{center}
            \end{vplace}
            \begin{figure}[h]
    	\centering
    	\includegraphics[scale=0.12]{figuras/ifsc.jpg}
    \end{figure}
    \\
    \vspace*{2.5cm}
            \begin{center}
                \ABNTEXchapterfont\bfseries\imprimirlocal, \ABNTEXchapterfont\bfseries\imprimirdata
            \end{center}
        \end{SingleSpacing}
    \end{capa}
}

% ----------------------------------------------------------
% folha de rosto 
% ----------------------------------------------------------
\makeatletter
\renewcommand{\folhaderostocontent}{
    \begin{SingleSpacing}
        \center
        \ABNTEXchapterfont\bfseries\ INSTITUTO FEDERAL DE EDUCAÇÃO, CIÊNCIA E TECNOLOGIA DE\\SANTA CATARINA - CÂMPUS FLORIANÓPOLIS\\DEPARTAMENTO ACADÊMICO DE SAÚDE E SERVIÇO\\CURSO DE MESTRADO PROFISSIONAL EM CLIMA E AMBIENTE
        
        \vspace*{3.0cm}     % três espaços simples - conforme norma para TCC do IFSC
        
        \ABNTEXchapterfont\bfseries\imprimirautor
        
        %\begin{vplace}[0.5]
        \vspace*{\fill} 
            \begin{center}
                \ABNTEXchapterfont\SingleSpacing\bfseries\imprimirtitulo
            \end{center}
        \vspace*{\fill} 
        %\end{vplace}
        
        \hspace{.45\textwidth}
        \begin{minipage}{.5\textwidth}
            \begin{SingleSpacing}
                \normalfont\imprimirpreambulo
                \vspace*{1.0cm}

                \imprimirorientadorRotulo~Prof. \imprimirorientador\par
                 \vspace*{1.0cm}
                \imprimircoorientadorRotulo~ \imprimircoorientador%
                
            \end{SingleSpacing}
        \end{minipage}%
        \vspace*{\fill}

         \begin{center}
            \ABNTEXchapterfont\bfseries\imprimirlocal, \ABNTEXchapterfont\bfseries\imprimirdata
        \end{center}
    \end{SingleSpacing}
}
\makeatother


% ----------------------------------------------------------
% Configurações de aparência do PDF final
% ----------------------------------------------------------
\definecolor{blue}{RGB}{41,5,195}   % alterando o aspecto da cor azul

\makeatletter
\hypersetup{                        % informações do PDF
     %	pagebackref=true,
		pdftitle={\@title}, 
		pdfauthor={\@author},
    	pdfsubject={\imprimirpreambulo},
		pdfkeywords={Palavra chave 1}{Palavra chave 2}{Palavra chave 3}, 
		colorlinks=true,       		% false: boxed links; true: colored links
    	linkcolor=black,          	% color of internal links
    	citecolor=black,        		% color of links to bibliography
    	filecolor=black,      		% color of file links
		urlcolor=black,
		bookmarksdepth=4
}
\makeatother

% ----------------------------------------------------------
% compila o indice
% ----------------------------------------------------------
\makeindex 

% --------------------------------------------------------------------------
% Configurações para inserir código fonte de programas - pacote listings
% --------------------------------------------------------------------------

\renewcommand{\lstlistingname}{Código} % Altera o nome padrão do rótulo usado no comando \autoref{}
\renewcommand{\lstlistlistingname}{Lista de códigos} % Altera o rótulo a ser usando no elemento pré-textual

% ----------------------------------------------------------
% Configura a "Lista de Códigos" conforme as regras da ABNT
% ----------------------------------------------------------
\begingroup\makeatletter
\let\newcounter\@gobble\let\setcounter\@gobbletwo
  \globaldefs\@ne \let\c@loldepth\@ne
  \newlistof{listings}{lol}{\lstlistlistingname}
  \newlistentry{lstlisting}{lol}{0}
\endgroup

\renewcommand{\cftlstlistingaftersnum}{\hfill--\hfill}

\let\oldlstlistoflistings\lstlistoflistings
\renewcommand{\lstlistoflistings}{
   \begingroup
   \let\oldnumberline\numberline
   \renewcommand{\numberline}{\lstlistingname\space\oldnumberline}
   \oldlstlistoflistings
   \endgroup
}

% ----------------------------------------------------------
% Definindo cores
% ----------------------------------------------------------
\definecolor{hellgelb}{rgb}{1,1,0.9}
\definecolor{colKeys}{rgb}{0,0,0}
\definecolor{colIdentifier}{rgb}{0,0,0.9}
\definecolor{colComments}{rgb}{.4,.4,.4}
\definecolor{colString}{rgb}{0,0,0.6}

\definecolor{colBack}{rgb}{1,1,.98}
\definecolor{colKeys}{rgb}{0,0,0}
\definecolor{colIdentifier}{rgb}{0,0,0.9}
\definecolor{colComments}{rgb}{.4,.4,.4}
\definecolor{colString}{rgb}{0,0,0.6}

\lstset{ %
    aboveskip=\bigskipamount,
    backgroundcolor=\color{colBack},   % choose the background color; you must add \usepackage{color} or 
    basicstyle=\ttfamily\footnotesize,       % the size of the fonts that are used for the code
    breakatwhitespace=false,         % sets if automatic breaks should only happen at whitespace
    breaklines=true,                 % sets automatic line breaking
    captionpos=n,                    % sets the caption-position to bottom
    columns=flexible,
    commentstyle=\color{colComments},    % comment style
    deletekeywords={...},            % if you want to delete keywords from the given language
    escapechar={@*},          % if you want to add LaTeX within your code
    extendedchars=true,              % lets you use non-ASCII characters; for 8-bits encodings only, does not work with UTF-8
    linewidth=0.98\linewidth,
    tab=$\to$,
    float=tbph,
    xleftmargin=10pt,
    frame=single,	                    % adds a frame around the code
    keepspaces=true,                    % keeps spaces in text, useful for keeping indentation of code (possibly needs columns=flexible)
    identifierstyle=\color{colIdentifier},
    keywordstyle=\color{colKeys},       % keyword style
    %  otherkeywords={*,...},           % if you want to add more keywords to the set
    firstnumber=last,
    numbers=left,                       % where to put the line-numbers; possible values are (none, left, right)
    numbersep=5pt,                      % how far the line-numbers are from the code
    numberstyle=\tiny,
    rulecolor=\color{black},            % if not set, the frame-color may be changed on line-breaks within not-black text (e.g. comments (green here))
    showspaces=false,                   %show spaces everywhere adding particular underscores; it overrides 'showstringspaces'
    showstringspaces=false,             % underline spaces within strings only
    showtabs=false,                     % show tabs within strings adding particular underscores
    %   stepnumber=2,                   % the step between two line-numbers. If it's 1, each line will be numbered
    stringstyle=\color{colString},      % string literal style
    tabsize=2,	                        % sets default tabsize to 2 spaces
    title=\lstname                      % show the filename of files included with \lstinputlisting; also try caption instead of title
}

% ----------------------------------------------------------
% Permitindo caracteres acentuados dentro do ambiente lstlisting
% ----------------------------------------------------------
\lstset{%
        inputencoding=utf8,
        extendedchars=true,
        literate=%
        {é}{{\'{e}}}1
        {è}{{\`{e}}}1
        {ê}{{\^{e}}}1
        {ë}{{\¨{e}}}1
        {É}{{\'{E}}}1
        {Ê}{{\^{E}}}1
        {û}{{\^{u}}}1
        {ù}{{\`{u}}}1
        {â}{{\^{a}}}1
        {à}{{\`{a}}}1
        {á}{{\'{a}}}1
        {ã}{{\~{a}}}1
        {Á}{{\'{A}}}1
        {Â}{{\^{A}}}1
        {Ã}{{\~{A}}}1
        {ç}{{\c{c}}}1
        {Ç}{{\c{C}}}1
        {õ}{{\~{o}}}1
        {ó}{{\'{o}}}1
        {ú}{{\'{u}}}1
        {Ú}{{\'{U}}}1
        {ô}{{\^{o}}}1
        {Õ}{{\~{O}}}1
        {Ó}{{\'{O}}}1
        {Ô}{{\^{O}}}1
        {î}{{\^{i}}}1
        {Î}{{\^{I}}}1
        {í}{{\'{i}}}1
        {Í}{{\~{Í}}}1
}

% ----------------------------------------------------------
% Criando comandos para algumas linguagens de programação
% ----------------------------------------------------------
\lstdefinestyle{shell}{language=csh,basicstyle=\ttfamily\footnotesize}
\lstdefinestyle{shellp}{language=csh,basicstyle=\ttfamily\scriptsize}
\lstdefinestyle{php}{language=php,basicstyle=\ttfamily\footnotesize}
\lstdefinestyle{phpp}{language=php,basicstyle=\ttfamily\scriptsize}
\lstdefinestyle{ansic}{language=c,basicstyle=\ttfamily\footnotesize}
\lstdefinestyle{ansicp}{language=c,basicstyle=\ttfamily\scriptsize}
\lstdefinestyle{java}{language=java,basicstyle=\ttfamily\footnotesize}
\lstdefinestyle{javap}{language=java,basicstyle=\ttfamily\scriptsize}
\lstdefinestyle{matlab}{language=matlab,basicstyle=\ttfamily\footnotesize}
\lstdefinestyle{matlabp}{language=matlab,basicstyle=\ttfamily\scriptsize}
\lstdefinestyle{python}{language=python,basicstyle=\ttfamily\footnotesize}
\lstdefinestyle{pythonp}{language=python,basicstyle=\ttfamily\scriptsize}
\lstdefinestyle{xml}{language=xml,basicstyle=\ttfamily\footnotesize}
\lstdefinestyle{xmlp}{language=xml,basicstyle=\ttfamily\scriptsize}
\lstdefinestyle{sql}{language=sql,basicstyle=\ttfamily\footnotesize}
\lstdefinestyle{sqlp}{language=sql,basicstyle=\ttfamily\scriptsize}

\newcommand{\ansic}{\lstset{style=ansic}}
\newcommand{\ansicp}{\lstset{style=ansicp}}
\newcommand{\java}{\lstset{style=java}}
\newcommand{\javap}{\lstset{style=javap}}
\newcommand{\sql}{\lstset{style=sql}}
\newcommand{\sqlp}{\lstset{style=sqlp}}
\newcommand{\xml}{\lstset{style=xml}}
\newcommand{\xmlp}{\lstset{style=xmlp}}
\newcommand{\python}{\lstset{style=python}}
\newcommand{\pythonp}{\lstset{style=pythonp}}
\newcommand{\csh}{\lstset{style=shell}}
\newcommand{\cshp}{\lstset{style=shellp}}
\newcommand{\shell}{\lstset{style=shell}}
\newcommand{\shellp}{\lstset{style=shellp}}


%%%%%%%%%%%%%%%%%%%%%%%%%%%%%%%%%%%%%%%%%%%%%%%%%%%
% conf-pacotes.sty
%%%%%%%%%%%%%%%%%%%%%%%%%%%%%%%%%%%%%%%%%%%%%%%%%%%

\ProvidesPackage{conf-pacotes}

% ----------------------------------------------------------
% Fontes para equações
% ----------------------------------------------------------
%\usepackage{sansmathfonts} 
%\usepackage[eulergreek]{sansmath} \sansmath
%\usepackage{arev}

% ----------------------------------------------------------
% ABNTEX
% ----------------------------------------------------------
\renewcommand{\familydefault}{\sfdefault}
\usepackage[scaled=1]{helvet}   % Fonte Arial 
\usepackage[helvet]{sfmath}
%\everymath={\sf}

%\usepackage{lmodern}			% Usa a fonte Latin Modern			
\usepackage[T1]{fontenc}		% Selecao de codigos de fonte.
\usepackage[utf8]{inputenc}		% Codificacao do documento (conversão automática dos acentos)
\usepackage{lastpage}			% Usado pela Ficha catalográfica
\usepackage{indentfirst}		% Indenta o primeiro parágrafo de cada seção.
\usepackage{color}				% Controle das cores
\usepackage{graphicx}			% Inclusão de gráficos
\usepackage{listings}           % Inclusão de códigos de softwares
\usepackage{microtype} 			% para melhorias de justificação
\usepackage{url,listings,color}
\usepackage[fleqn]{amsmath}
\usepackage{caption}
\usepackage{subcaption}
\usepackage[printonlyused,withpage]{acronym}
\usepackage[withpage]{acronym}
\usepackage{lipsum}				% para geração de dummy text
\usepackage{float}              % Inclusão de numeros float
\usepackage[final]{pdfpages}    % Inclusão de PDFs
\usepackage{lipsum}				% para geração de dummy text
\usepackage{natbib}

% ----------------------------------------------------------
% Usados no anexo do modelo de folha de identificação
% ----------------------------------------------------------
\usepackage{multicol}
\usepackage{multirow}

% ----------------------------------------------------------
% Pacotes de citações
% ----------------------------------------------------------
\usepackage[brazilian,hyperpageref]{backref}    % Paginas com as citações na bibl

%marieli
%\usepackage[alf, abnt-etal-text=emph]{abntex2cite}                  % Citações padrão ABNT

%\usepackage[alf]{abntex2cite}	                % Citações padrão ABNT

\setcitestyle{round,aysep={,},yysep={;}}

% ----------------------------------------------------------
% Outros
% ----------------------------------------------------------
\usepackage[brazil]{babel}	% coloca as coisas em portugues no sumário.
\usepackage[normalem]{ulem}	% provê sublinhados para textos (\ul)

%\usepackage{remreset}		% reinicia contadores
\usepackage{setspace}       % pacote para espaçamentos
\usepackage{textcomp}       % pacote para simbolos REGISTERED e ESPECIAL
%\usepackage{pgfplots}      % pacote para uso do pgfplots

% ----------------------------------------------------------
% Meus pacotes
% ----------------------------------------------------------
\usepackage[binary-units=true]{siunitx}        % http://tug.ctan.org/macros/latex/exptl/siunitx/siunitx.pdf
\sisetup{locale = DE}

\usepackage{mathrsfs}
%\usepackage[pdftex]{hyperref}
\usepackage[dvipsnames]{xcolor}
\setlength{\mathindent}{0pt}                % remover margem antes das equações

%% --- Remover conflito com ABNTex ---
\let\su@ExpandTwoArgs\relax
\let\IfSubStringInString\relax
\let\su@IfSubStringInString\relax
%% -----------------------------------

\usepackage{datatool}       % Ordenar a lista de abrev. e simbolos
\usepackage{mfirstuc}       % Usada para deixar a primeira letra maiúscula


%%%%%%%%%%%%%%%%%%%%%%%%%%%%%%%%%%%%%%%%%%%%%%%%%%%
% main.tex
%%%%%%%%%%%%%%%%%%%%%%%%%%%%%%%%%%%%%%%%%%%%%%%%%%%

\DeclareOption*{\PassOptionsToClass{\CurrentOption}{abntex2}}
\ProcessOptions

\documentclass[
	% -- opções da classe memoir --
	12pt,				% tamanho da fonte
	openright,			% capítulos começam em pág ímpar (insere página vazia caso preciso)
	oneside,			% para impressão em verso e anverso. Oposto a oneside
	a4paper,			% tamanho do papel. 
	% -- opções da classe abntex2 --
	%chapter=TITLE,		% títulos de capítulos convertidos em letras maiúsculas
	%section=TITLE,		% títulos de seções convertidos em letras maiúsculas
	%subsection=TITLE,	% títulos de subseções convertidos em letras maiúsculas
	%subsubsection=TITLE,% títulos de subsubseções convertidos em letras maiúsculas
	% -- opções do pacote babel --
	english,			% idioma adicional para hifenização
	french,				% idioma adicional para hifenização
	spanish,			% idioma adicional para hifenização
	brazil				% o último idioma é o principal do documento
	]{abntex2}

%sudo apt update
%sudo apt install texlive-latex-extra (instalar pacote abntex2)

\usepackage{abntex2}

\usepackage{setspace}
\usepackage{conf-ifsc}	
\usepackage{hyperref}
\usepackage{graphicx}
\usepackage{caption}
\usepackage{subcaption}
\usepackage[utf8]{inputenc}
\usepackage{listings}
\usepackage{xcolor}
\usepackage{trivfloat}
\trivfloat{quadro}
\usepackage{enumitem}
\usepackage{xcolor}
\usepackage{multicol}
\usepackage{multirow}
\usepackage{hyperref}
\usepackage{float}
\floatstyle{plaintop}
\restylefloat{quadro}
\usepackage{chngcntr}
\usepackage{lipsum}
\usepackage{tocloft}
\usepackage{tocvsec2}
\usepackage{boxhandler}
\usepackage[font={bf,small},labelfont=small]{caption}


\newcommand{\source}[1]{\caption*{\normalfont Fonte: {#1}}}

\counterwithout{equation}{chapter}
\counterwithout{figure}{chapter}
\counterwithout{table}{chapter}

\newcommand\ChangeRT[1]{\noalign{\hrule height #1}}

\newcommand{\ingles}[1]{{\textit{#1}}}
\newcommand{\latim}[1]{{\textit{#1}}}
% \renewcommand{\coorientadorb}{coorientador:}

\definecolor{codegreen}{rgb}{0,0.6,0}
\definecolor{codegray}{rgb}{0.5,0.5,0.5}
\definecolor{codepurple}{rgb}{0.58,0,0.82}
\definecolor{backcolour}{rgb}{0.95,0.95,0.92}

\lstdefinestyle{mystyle}{
    backgroundcolor=\color{backcolour},   
    commentstyle=\color{codegreen},
    keywordstyle=\color{magenta},
    numberstyle=\tiny\color{codegray},
    stringstyle=\color{codepurple},
    basicstyle=\ttfamily\footnotesize,
    breakatwhitespace=false,         
    breaklines=true,                 
    captionpos=b,                    
    keepspaces=true,                 
    numbers=left,                    
    numbersep=5pt,                  
    showspaces=false,                
    showstringspaces=false,
    showtabs=false,                  
    tabsize=2
}

\lstset{style=mystyle}

%--- PACOTE E LISTA DE ACRÔNIMOS--------------------------------------%
\usepackage{glossaries}
\makeglossaries
%\newacronym{}{}{} Anvisa/Dive/Gezoo/DATASUS, TABNET, VIGILANTOS 
\newacronym{GFS}{GFS}{Global Forecast System}
\newacronym{MERRA2}{MERRA2}{Modern-Era Retrospective analysis for Research and Applications, Version 2}
\newacronym{MERGE}{MERGE}{Merging Technique}
\newacronym{SAMeT}{SAMeT}{South American Mapping of Temperature}
\newacronym{IBGE}{IBGE}{Instituto Brasileiro de Geografia e Estatística}
\newacronym{TRMM}{TRMM}{Tropical Rainfall Measuring Mission }
\newacronym{ERA5}{ERA5}{ECMWF Reanalysis 5th Generation}
\newacronym{ECMWF}{ECMWF}{European Centre for Medium-Range Weather Forecasts}
\newacronym{GTOPO30}{GTOPO30}{Global Digital Elevation}
\newacronym{PCAM}{PCAM}{Programa de Pós-Graduação Mestrado Profissional em Clima e Ambiente}
\newacronym{PTT}{PTT}{Produto Técnico-Tecnológico}
\newacronym{FF}{FF}{Frentes Frias}
\newacronym{LI}{LI}{Linhas de Instabilidade}
\newacronym{CCM}{CCM}{Complexos Convectivos de Mesoescala}
\newacronym{ZCAS}{ZCAS}{Zona de Convergência do Atlântico Sul}
\newacronym{CDO}{CDO}{Climate Data Operators}
\newacronym{GrADS}{GrADS}{Grid Analysis and Display System}
\newacronym{GTS}{GTS}{Global Telecommunications System}
\newacronym{CPTEC}{CPTEC}{Centro de Previsão de Tempo e Estudos Climáticos}
\newacronym{INPE}{INPE}{Instituto Nacional de Pesquisas Espaciais}
\newacronym{MEC}{MEC}{Ministério da Educação}
\newacronym{CNPq}{CNPq}{Conselho Nacional de Desenvolvimento Científico e Tecnológico}
\newacronym{CAPES}{CAPES}{Coordenaçao de Aperfeiçoamento de Pessoal de Nível Superior}
\newacronym{ASOS}{ASOS}{Automatic Surface Observation Station}
\newacronym{OMS}{OMS}{Organização Mundial de Saúde}
\newacronym{ONU}{ONU}{Organização das Nações Unidas}
\newacronym{FAO}{FAO}{Organização das Nações Unidas para a Alimentação e a Agricultura}
\newacronym{CFMV}{CFMV}{Conselho Federal de Medicina Veterinária}
\newacronym{CRMV}{CRMV}{Conselho Regional de Medicina Veterinária}
\newacronym{SC}{SC}{Santa Catarina}
\newacronym{PR}{PR}{Paraná}
\newacronym{CEGRADE}{CEGRADE}{Comissão Estadual de Gestão de Riscos de Animais em Desastres}
\newacronym{DF}{DF}{Distrito Federal}
\newacronym{PNCD}{PNCD}{Programa Nacional de Controle da Dengue}
\newacronym{DENV}{DENV}{Dengue Virus}
\newacronym{Funasa}{Funasa}{Fundação Nacional de Saúde}
\newacronym{ESRI}{ESRI}{Environmental Systems Research Institute}
\newacronym{DataSUS}{DataSUS}{Departamento de Informática do Sistema Único de Saúde}
\newacronym{Dive}{Dive}{Diretoria de Vigilância Epidemiológica}
\newacronym{LACEN}{LACEN}{Laboratório Central de Saúde Pública}
%\newacronym{DENV}{DENV}{vírus da dengue}
\newacronym{ssRNA}{ssRNA}{single-stranded Ribonucleic Acid}
\newacronym{IFSC}{IFSC}{Instituto Federal de Ensino, Ciência e Tecnologia de Santa Catarina}
\newacronym{UFSM}{UFSM}{Universidade Federal de Santa Maria}
\newacronym{ESBMet}{ESBMet}{Encontro Sul Brasileiro de Meteorologia}
%\newacronym{PCAM}{PCAM}{Programa de Pós-graduação em Clima e Ambiente}
\newacronym{DEE}{DEE}{Dados Entomo-Epidemiológicos}
\newacronym{DEC}{DEC}{Dados de Elementos Climáticos}
\newacronym{DSG}{DSG}{Dados Socioeconômicos e Geográficos}
\newacronym{DGR}{DGR}{Dados Georreferenciados}
\newacronym{ICOb}{ICOb}{Índice Combinado por Observação}
\newacronym{ICRe}{ICRe}{Índices Corrigidos por Região}
\newacronym{LACOb}{LACOb}{Limiar de Alta Correlação Observado}
\newacronym{LACRe}{LACRe}{Limiares de Alta Correlação Regionalizados}
\newacronym{SBP}{SBP}{Sociedade Brasileira Técnico-Científica de Parasitologia}
\newacronym{JBN}{JBN}{Jatos de Baixos Níveis}
\newacronym{Bdmep/Inmet}{Bdmep/Inmet}{Banco de Dados Meteorológicos do Instituto Nacional de Meteorologia}
\newacronym{Inmet}{Inmet}{Instituto Nacional de Meteorologia}
\newacronym{MAPA}{MAPA}{Ministério da Agricultura, Pecuária e Abastecimento}
\newacronym{Sinan}{Sinan}{Sistema de Informação de Agravos de Notificação}
\newacronym{Cenepi}{Cenepi}{Centro Nacional de Epidemiologia}
\newacronym{SVS}{SVS}{Secretaria de Vigilância em Saúde}
\newacronym{csv}{csv}{comma separated values}
\newacronym{ReLU}{ReLU}{Rectified Linear Unit}

%----------------------------------------------------------------------%


%---------------------------------------------------------------------%
%---------------------------------------------------------------------%
% Informações de dados para CAPA e FOLHA DE ROSTO
%---------------------------------------------------------------------%
%---------------------------------------------------------------------%

% \titulo{VARIABILIDADE CLIMÁTICA E SAÚDE ÚNICA:\\
% Papel dos elementos meteorológicos na reprodução do \latim{Aedes aegypti}\\
% no Estado de Santa Catarina}
\titulo{Utilização de Reanálise e Produtos de Reanálise\\para previsão de focos de  \latim{Aedes}\  sp. e casos de dengue\\no Estado de Santa Catarina}

\autor{MATHEUS FERREIRA DE SOUZA}

\local{FLORIANÓPOLIS}

\data{2024.}

\orientador[Orientador:\\]{Mário Francisco Leal de Quadro, Doutor em Meteorologia.}
\coorientador[Coorientadores:\\]{Adriano Vitor, Doutor em Métodos Numéricos em Engenharia.\\João Augusto Brancher Fuck, Doutor em Planejamento Territorial e Desenvolvimento Socioambiental. }
% \renewcommand{\coorientador}{Coorientadores}
% \coorientadorb[Coorientador:\\]{João Augusto Brancher Fuck, Doutor em Planejamento Territorial e Desenvolvimento Socioambiental.}
% \coorientadores[Coorientadores:\\]{}{Adriano Vitor, Doutor em Matemática.}{João Augusto Brancher Fuck, Doutor em Planejamento Territorial e Desenvolvimento Socioambiental.}

\tipotrabalho{Dissertação (Mestrado)}

% O preambulo deve conter o tipo do trabalho, o objetivo, o nome da instituição e a área de concentração 
\preambulo{Dissertação submetida ao Instituto Federal de Educação, Ciência e Tecnologia de Santa Catarina como parte dos requisitos para obtenção do título de Mestre Profissional em Clima e Ambiente.}

% \textoaprovacao{Este Trabalho foi julgado adequado para obtenção do Título de Engenheira Eletrônica em abril de 2021 e aprovado na sua forma final pela banca examinadora do Curso de Engenharia Eletrônica do instituto Federal de Educação Ciência, e Tecnologia de Santa Catarina.}


%---------------------------------------------------------------------%
% Início do documento
%---------------------------------------------------------------------%

\begin{document}



\selectlanguage{brazil}
\frenchspacing 


% ----------------------------------------------------------
% ELEMENTOS PRÉ-TEXTUAIS
% ----------------------------------------------------------
% \pretextual

\imprimircapa

\imprimirfolhaderosto* %(o * indica que haverá a ficha bibliográfica)

%---------------------------------------------------------------------%
% ATENÇÃO - Pergunte para a Biblioteca do IFSC
% Inserir a ficha bibliografica - 
%
% Para gerar a ficha catalográfica acesse:
% http://ficha.florianopolis.ifsc.edu.br/
% Precisa ser feito pelo navegador Mozilla Firefox
%---------------------------------------------------------------------%

\imprimirficha{pdf/fichacatalografica.pdf}
%\cleardoublepage

%---------------------------------------------------------------------%
% Inserir folha de aprovação
%---------------------------------------------------------------------%


%\imprimiraprovacao
\includepdf[pages=-]{pdf/tcc-aprovado.pdf}

%\cleardoublepage

%---------------------------------------------------------------------%
% Dedicatória
%---------------------------------------------------------------------%
\begin{dedicatoria}
    \vspace*{\fill}
	\begin{flushright}

\fcolorbox{red}{red}{|||||||||||||||||||||||||||||||||}\\
\fcolorbox{red}{red}{|||||||||||||||||||||||||||||||||}\\
\fcolorbox{red}{red}{|||||||||||||||||||||||||||||||||}\\
\fcolorbox{red}{red}{|||||||||||||||||||||||||||||||||}\\
\fcolorbox{red}{red}{|||||||||| AO AVANÇO}\\
\fcolorbox{red}{red}{  E PERSISTÊNCIA}\\
\fcolorbox{red}{red}{||||||||| DA CIÊNCIA,}\\
\fcolorbox{red}{red}{PRINCIPALMENTE}\\
\fcolorbox{red}{red}{|||||||||||||||||||||||| NOS}\\
\fcolorbox{red}{red}{|||||||||||| PERÍODOS}\\
\fcolorbox{red}{red}{ CONTURBADOS. }\\
\fcolorbox{red}{red}{|||||||||||||||||||||||||||||||||}\\
\fcolorbox{red}{red}{|||||||||||||||||||||||||||||||||}\\
\fcolorbox{red}{red}{||||||||||||||| A TODOS}\\
\fcolorbox{red}{red}{|||||||||| QUE A VIDA}\\
\fcolorbox{red}{red}{||||||||||||| CESSOU...}\\
\vspace{1.15cm}

\fcolorbox{red}{red}{|||||||| ...POR FALSA}\\
\fcolorbox{red}{red}{|||||| INFORMAÇÃO,}\\
\fcolorbox{red}{red}{DESINFORMAÇÃO}\\
\fcolorbox{red}{red}{||||||| OU NEGAÇÃO}\\
\fcolorbox{red}{red}{||||||||||| DA MESMA.}
      
	\end{flushright}
\end{dedicatoria}




%---------------------------------------------------------------------%
% Agradecimentos
%---------------------------------------------------------------------%
\begin{agradecimentos}
    Elemento opcional que não pode ultrapassar o limite de uma página.
\end{agradecimentos}
% ---

%---------------------------------------------------------------------%
% Epígrafe
%---------------------------------------------------------------------%
\begin{epigrafe}
    \vspace*{\fill}
	\begin{flushright}
        		"If I have seen further, it is by standing on the shoulders of giants"\\(Sir Isaac Newton).
	\end{flushright}
\end{epigrafe}

%---------------------------------------------------------------------%
% RESUMOS
%---------------------------------------------------------------------%
% resumo em português
\setlength{\absparsep}{18pt} % ajusta o espaçamento dos parágrafos do resumo
\renewcommand{\baselinestretch}{1} 
\begin{resumo}
    % O resumo deve mostrar a natureza e o objetivo do trabalho, o método que foi empregado, os resultados e as conclusões. O resumo deve conter entre 150 e 500 palavras e constitui-se de um único parágrafo, sem recuo.
A dengue é uma doença viral infecciosa, transmitida pela picada da fêmea do mosquito \latim{Aedes aegypti} (Linnaeus, 1762). Atualmente é um dos principais problemas de saúde coletiva no mundo. Está presente nas regiões tropicais e subtropicais do planeta, sendo estimado que 390 milhões de pessoas sejam infectadas anualmente pelos quatro sorotipos. Há tendência de resposta à sazonalidade em relação ao número de focos de dengue, na qual os meses com temperaturas mais elevadas se observa crescimento, principalmente quando retroagidos por um período de 30 dias; e tem diminuição nos meses mais frios, logo, temperatura é um elemento crítico para o desenvolvimento do mosquito. A geomedicina porporciona base para o próprio estudo espacializado de doenças, sendo assim, a seguinte proposta tem o objetivo direto de observar a relação entre o histórico e a distribuição de focos do vetor biológico \latim{Aedes aegypti}, transmissor das principais arboviroses no Estado de Santa Catarina (dengue, febre amarela, zika e chikungunya), e o comportamento de elementos climáticos. Foram utilizados dados de temperatura do \ingles{\acrfull{GFS}}, após isso, foi sintetizado o \acrfull{ICOb}, utilizando-se o \acrfull{LACOb}. Dessa forma, pode-se modelar e visualizar cartograficamente esses resultados, como previsão de aumento regionalizado de focos de \latim{Aedes aegypti} no Estado de Santa Catarina. Ademais, pretende-se alcançar outros objetivos, tais como: mapeamento das ocorrências dentro do Estado, análise de dados entomológicos e epidemiológico; além de relacionar aos dados socioeconômicos da população catarinense. Investigar esses múltiplos aspectos e compreender essas informações auxiliam no processo de tomada de decisões, buscando, assim, uma melhor resposta frente a zoonoses emergentes e re-emergentes.
 
   \noindent 
    \textbf{Palavras-chave}: Variabilidade Climática. Geomedicina. Saúde Única. Zoonose. 
\end{resumo}

% resumo em inglês
\renewcommand{\baselinestretch}{1} 
\begin{resumo}[Abstract]
 \begin{otherlanguage*}{english}
   % The abstract should show the nature and scope of work, the method that was used, the results and conclusions. The abstract may contain between 150 and 500 words, and it must be only one paragraph. 
 Dengue is an viral infectious disease transmitted by the female mosquito \latim{Aedes aegypti} (Linnaeus, 1762). Nowadays, it is one of the main public health problems in the world. It is present in the tropical and subtropical regions of the planet, with an estimated 390 million people are infected annually with all four serotypes. There is a tendency to respond to seasonality in relation to the number of dengue outbreaks, in which the months with higher temperatures are observed growth, mainly when retroacted for a period of 30 days; and observe decrease in the colder months, therefore, temperature is a critical element for the growth of the mosquito. Geomedicine provides a basis for the spatialized study of diseases, thus, this project has the direct objective of observing the relationship between the history and the distribution of outbreaks of the biological vector \latim{Aedes aegypti}, transmitter of the main arboviruses in the State of Santa Catarina (dengue, yellow fever, zika and chikungunya), and the observation of climate elements. The temperature data were used from the \acrfull{GFS}, after which the Combined Index by Observation [\acrfull{ICOb}] was synthesized, using the Observed High Correlation Limit [\acrfull{LACOb}]. In this way, it is possible to model and visualize these results cartographically, as forecast of a regionalized increase in outbreaks of \latim{Aedes aegypti} in the State of Santa Catarina. Furthermore, it is intended to achieve other objectives, such as: mapping of occurrences within the State, analysis of entomological and epidemiological data; in addition to relating to socioeconomic data of the Santa Catarina population. Investigate these multiple aspects and understand this information assist in the decision-making process, thus seeking a better response to emerging and reemerging  zoonotic diseases. 

   %\vspace{-0.8cm}
 
   \noindent 
   \textbf{Keywords}: Climate Variability. Geomedicine. One Health. Zoonosis.  
 \end{otherlanguage*}
\end{resumo}


%---------------------------------------------------------------------%
% inserir lista de ilustrações
%---------------------------------------------------------------------%
\renewcommand{\listfigurename}{Lista de Figuras}
\pdfbookmark[0]{\listfigurename}{lof}
\listoffigures*
\cleardoublepage

%---------------------------------------------------------------------%
% inserir lista de quadros
%---------------------------------------------------------------------%

\renewcommand{\listquadroname}{Lista de quadros}
\newfloat{quadro}{\quadroname}{loq}[chapter]
\setfloatlocations{quadro}{hbtp}
\newlistof{listofquadros}{loq}{\listquadroname}
\newlistentry{quadro}{loq}{0}
\renewcommand{\cftquadroname}{\quadroname\space}
\renewcommand*{\cftquadroaftersnum}{\hfill\textendash\hfill}
\counterwithout{quadro}{chapter}
\listofquadros* 
\cleardoublepage



%---------------------------------------------------------------------%
% inserir lista de tabelas
%---------------------------------------------------------------------%
\pdfbookmark[0]{\listtablename}{lot}
\listoftables*
\cleardoublepage

%---------------------------------------------------------------------%
% inserir lista de listings
%---------------------------------------------------------------------%
%\pdfbookmark[0]{\lstlistlistingname}{lol}
%\listoflistings
%\cleardoublepage

%---------------------------------------------------------------------%
% inserir lista de abreviaturas e simbolos
%---------------------------------------------------------------------%
%\listofabrev{tex/00-Abreviaturas}
% \imprimirlistadeabreviaturas
%ESTOU USANDO \usepackage{glossaries}---------------------------------%
\printglossary[title = Lista de Abreviaturas e Siglas, type = \acronymtype]
%---------------------------------------------------------------------%

% \imprimirlistadesimbolos
\cleardoublepage

%---------------------------------------------------------------------%
% inserir o sumario

%---------------------------------------------------------------------%


\setlength{\cftbeforechapterskip}{4pt plus 0pt}
\pdfbookmark[0]{\contentsname}{toc}
\tableofcontents*
\cleardoublepage

% ----------------------------------------------------------
% ELEMENTOS TEXTUAIS
% ----------------------------------------------------------
\textual
\pagestyle{parpage}%
\aliaspagestyle{chapter}{parpage}
% ----------------------------------------------------------
% Inclusão dos capítulos que estão em outros arquivos .tex
% ----------------------------------------------------------

\chapter{Introdução}
%Máximo 2pg:
% Contextualizar o Clima de SC;
\indent O clima é a composição do tempo meterológico ao longo de um período, pelo menosn trinta anos, em um determinado território. (\acrlong{SC}, \citeyear{AtlasSCnatureza}, pg-75 ).\\
\indent O estado de \acrfull{SC} está localizado no sul do Brasil, região onde ocorrem totais pluviométricos elevados e precipitação bem distribuída ao longo do ano. Parte desse comportamento é devido a atuação sistemas meteorológicos como: \acrfull{FF}, Sistemas Frontais, Clicones Extratropicais, Vórtices Clicônicos em Altos Níveis, Bloqueios Atmosféricos e \acrfull{CCM}, além a influência indireta da \acrfull{ZCAS} \cite{reboita2010}.\\
\begin{center}
\textcolor{red}{CONSIDERAÇÕES AQUI!}\\ 
\end{center}
% CONTEXTUALIZAR TEMPERATURA/CLIMA EM SC
% INCLUIR OUTROS EC?;
% Contextualizar Dengue;\\
\indent Assim, sendo expresso no Guia de Vigilância em Saúde, o mosquito \latim{Aedes aegypti} pode comumente transmitir algumas arboviroses, como: Dengue, Zika e Chikungunya \cite{GuiaVigSaúde22}; o \latim{A. aegypti} também é responsável pela transmissão de Febre Amarela em ciclo urbano. Ainda informado pelo \citeonline{GuiaVigSaúde22}, a dengue possui como agente etiológico o \acrfull{DENV}, sendo agrupado em quatro sorotipos distintos (\acrshort{DENV}-1, \acrshort{DENV}-2, \acrshort{DENV}-3 e \acrshort{DENV}-4), e é uma das zoonoses vetorizadas por artrópodes mais relevantes nas Américas.\\
% Relacionar Clima e Dengue (Geral);\\
\indent Em estudo realizado por \citeonline{Drumond2020Dinamica}, entre 2007 e 2017 no \acrfull{DF}, o número de casos foi maior nos primeiros semestres, particularmente entre o final do verão e o início do outono. Este período  coincide com o final do período chuvoso. Os autores também observaram que a dengue se manteve hiperendêmica no \acrlong{DF}, tendo casos registrados durante todos os meses do ano e circulação dos quatro sorotipos do vírus.\\
\begin{center}
\textcolor{red}{CONSIDERAÇÕES AQUI!}\\ 
\end{center}
% INCLUIR OUTROS ESTUDOS
% Relacionar Clima e Dengue (SC);\\
{\color{red} \rule{\linewidth}{0.5mm}}\color{red} \\ INÍCIO DE CONSIDERAÇÃO\\
\indent O Estado catarinense apresentou em 2014 apenas três casos autóctones de dengue (infectados dentro do próprio estado), no município de Itajaí, e 66 casos importados. Naquele mesmo período, a região serrana e planalto não apresentavam focos \cite{Matiola2020Dissertação}. De acordo com a \citeonline{Informe5DiveSE9/23}, para este momento (final de março de 2023 - décima segunda semana epidemiológica), há focos de dengue distribuídos por todas as regiões catarinenses.\\
FINAL DE CONSIDERAÇÃO \\ {\color{red} \rule{\linewidth}{0.5mm}}\color{black}
\indent Segundo \citeonline{Matiola2020Dissertação}, em \acrlong{SC}, há tendência de resposta à sazonalidade nos meses com temperaturas mais elevadas, observando-se o crescimento do número de focos do \latim{Aedes} sp.,  principalmente quando retroagidos por um período de 30 dias, além da diminuição nos meses mais frios.\\ 
\indent Portanto, a temperatura é um fator crítico para o desenvolvimento do mosquito no estado catarinense. com relação a previsibilidade das ocorrências, \citeonline{Matiola2020Dissertação} também sugerem que a precipitação do mês que antecede o registro do foco influencia no desenvolvimento do mosquito, corroborando com correlações maiores quando retroativas.\\
% Contextualizar estudos com modelos Dengue/A.a
\begin{center}
\textcolor{red}{CONSIDERAÇÕES AQUI!}\\ 
\end{center}
% INCLUIR INFODENGUE COMO MODELO PREDITIVO
%%%%%%%%%%%%%%%%%%%%%%%%%%%%%%%%%%%%%%%%%%%%%%%%%%%%%%%%%%%%%%%%%%%%%%%%%%%%
\newpage
\section{Justificativa}
%1parágrafo
\indent Pelo exposto, a dengue é uma zoonose emergente no estado de \acrlong{SC} e se faz necessário relacionar a nova dinâmica climatológica, a ocorrência de \latim{Aedes aegypti} e os fatores sociais frente a esse atual comportamento.
\begin{center}
\textcolor{red}{CONSIDERAÇÕES AQUI!}\\ 
\end{center}
% necessidade de reconhecer relação das variáveis climáticas com a reprodução do mosquito, mudança no cenário da doença no estado associado a disseminação do mosquito...
 
%%%%%%%%%%%%%%%%%%%%%%%%%%%%%%%%%%%%%%%%%%%%%%%%%%%%%%%%%%%%%%%%%%%%%%%%%%%%%
 \section{Objetivo Geral}
 Avaliar a relação entre os focos de \latim{Aedes} sp. e os casos de dengue e a condição climatológica predominante sobre o Estado de \acrlong{SC}, visando elaborar um modelo preditivo de proliferação do vetor biológico e da doença.

\begin{center}
\textcolor{red}{CONSIDERAÇÕES AQUI!}\\ 
\end{center}

 % Elaborar um modelo preditivo para a proliferação do mosquito \latim{Aedes aegypti}, a partir de análise das condições meteorológicas do Estado de \acrlong{SC}.

 % Analisar a relação entre a reprodução do mosquito \latim{Aedes aegypti} e as condições meteorológicas no Estado de \acrlong{SC}.

 
\section{Objetivos Específicos}
As estratégias para atingir o objetivo geral são descritas como objetivos específicos abaixo:

\begin{alineas}
\item Definir índices e limiares das ocorrências de \latim{Aedes aegypti} por regiões do Estado de \acrlong{SC};
\item Analisar estatisticamente esses índices e limiares para validação da ferramenta;
\item Realizar um estudo de caso para melhor entender o fenômeno a partir de um ciclo anual;
\item Sintetizar o \acrfull{PTT}:
\subitem Sistema Computacional;
\subitem Visualização Cartográfica desses resultados.
    %Último deve relacionar o produto
\end{alineas}

 % Introdução + Justificativa + Objetivos
\chapter{Fundamentação Teórica}

Esta seção apresenta uma breve contextualização sobre questões sanitárias e climatológicas. Foi elaborada através de  fichamento do material científico acessado (livros, artigos, \ingles{sites}, boletins e informes oficiais, dentre outros) e servirá como base para conciliar diferentes áreas do conhecimento, assim como facilitar o entendimento de terminologias abordadas no presente estudo.

\section{Caracterização do Clima e Ambiente}

\subsection{Climatologia e Sistemas Meteorológicos de Santa Catarina (SC)}

\indent A palavra 'clima' vem do radical grego 'klíma', semanticamente relacionado à 'inclinação'. Esse significado advém da inclinação dos raios solares em relação ao globo terrestre, variando conforme a posição geográfica. Logo, os gregos foram os primeiros a propor uma classificação climática, baseada apenas nas latitudes \cite{AtlasClimaticoSul}.

\indent Esse entendimento evolui conforme o avanço da ciência, sendo hoje compreendido o clima como "a sucessão habitual dos diversos tipos de tempo que compõem o cenário atmosférico de uma região ao longo de um período de pelo menos trinta anos" (\acrlong{SC}, \citeyear{AtlasSCnatureza}, pg-75 ).

\indent Segundo \citeonline{reboita2010}, \acrlong{SC} se encontra no setor R4 de precipitação, sendo similar ao restante do sul do Brasil, ao Paraguai e ao Uruguai. No entanto, utilizando a classificação de \citeonline{MERGEatual}, o Estado catarinense está compreendido no setor R1 de precipitação.

\indent Ambos os autores defendem um total pluviométrico elevado e bem distribuído ao longo do ano. Esse comportamento se deve a atuação de sistemas meteorológicos na região. A presença de vórtices ciclônicos e cavados em altos níveis sobre a costa oeste da América do Sul favorece o desenvolvimento de ciclones e \acrlong{FF} nesses setores (R1 \cite{MERGEatual} e R4 \cite{reboita2010}). Dessa forma, condições ciclogenéticas e frontogenéticas se desenvolvem sobre \acrlong{SC}.

\indent Os sistemas frontais, que se deslocam do sul em sentido Sudoeste-Nordeste, passando pela Argentina e adentrando o Brasil, causam precipitação atuando diretamente na região ou fornecendo condições para o desenvolvimento de \acrfull{LI} pré-frontais \cite{reboita2010}.

\indent As condições favoráveis à ocorrência de tempestades aumentam quando o processo convectivo é acoplado a um ou mais sistemas de instabilidade, como vórtices ciclônicos, baixas térmicas, \acrfull{JBN} ou a passagem de \acrlong{FF} pelo litoral sul do Brasil (\acrlong{SC}, \citeyear{AtlasSCnatureza}).

\indent Como citado por \citeonline{TempoClima}, os \acrfull{CCM} são sistemas definidos por sua forma, tamanho e tempo de vida. Em relação ao formato, deve apresentar círculos concêntricos e essa excentricidade deve ser maior que 0,7 (menor/maior).

\indent Ainda tratando sobre \acrshort{CCM}, o tamanho da maior área deve ser de 100.000 km² e temperatura no topo das nunvens inferior a -32C; internamente deve apresentar uma área com metade do tamanho, 50.000 km², e temperaturas ainda menores, -52C \citeonline{TempoClima}.

\indent  Caso as duas características anteriores, formato e tamanho, estejam presentes, o período de ocorrência acima de seis horas (6h) determina a denominação do \acrshort{CCM}. Não preenchendo essas três caraterísticas, os eventos são apenas denominados genericamente de \acrfull{SCM} \citeonline{TempoClima}.

\indent A \acrfull{ZCAS} se identifica, por imagens de satélite, como uma alongada nebulosidade distribuída no sentido Noroeste-Sudeste, estendendo-se desde a região sul da Amazônia até a região central do Atlântico Sul \cite{ClimatologiaBasica}.

\indent Esses mesmos autores, \citeonline{ClimatologiaBasica}, também comentam que a \acrshort{ZCAS} é resultante do encontro de massas de ar igualmente quentes e úmidas, intensificando essa temperatura e umidade nessa região de distribuição.

\indent Para \citeonline{TempoClima}, a \acrshort{ZCAS} apresenta um ciclo anual regular, tendo o pico da estação chuvosa entre dezembro e fevereiro sobre as regiões Sudeste e Centro-Oeste. Essa característica climatológica é associada ao escoamento convergente de umidade na baixa troposfera.

\begin{citacao}
"As análises sugerem que um episódio de \acrshort{ZCAS} deva ocorrer associado aos seguintes padrões meteorológicos: convergência de umidade na baixa e média troposfera; faixa de movimento ascendente do ar com orientação NW/SE; um cavado semi-estacionário sobre a costa leste da América do Sul em 500 hPa; intenso gradiente de 0$_0$ na média troposfera e também uma faixa de vorticidade relativa anticiclônica em altos níveis (200 hPa)" (\citeauthor{ZCASquadro}, \citeyear{ZCASquadro}, pg-210).
\end{citacao}

\indent Como citado por \citeonline{ClimatologiaBasica}, as brisas são ventos locais que "decorrem de um gradiente de pressão local que se estabelece como resultado do aquecimento diferencial da superfície com a alternância do dia e da noite" . São conhecidas as brisas: marinha, oceânica, terrestre, continental, de vale e de montanha.

\indent De acordo com o Atlas Geográfico de \acrlong{SC} (\citeyear{AtlasSCnatureza}), os maiores índices de precipitação anuais são observados na região Oeste, Grande Florianópolis e Litoral Norte. Entretanto, no Litoral Sul, entre Araranguá e Laguna, os totais anuais desses índices são os menores.

\indent O efeito sazonal da dinâmica dos sistemas atmosféricos, sobretudo pelas altas pressões, e a influência da orografia, principalmente por planícies e planaltos, resultam em variações na distribuição espacial de elementos climáticos, como temperatura e precipitação, que se observa menores temperaturas médias em regiões de maiores altitudes e tendência com maiores acumulados de precipitação próximos às serras. Em consonância com \citeonline{Oliveira2022Sistema}, a definição do sistema de anticiclone é tida como centros de alta pressão atmosférica.

\begin{citacao}
"...A maior influência ocorre pelo anticiclone Migratório Polar, centro de ação da Massa de Ar Polar, úmida quando a trajetória é marítima (mPa) e seca quando a trajetória é continental, sempre fria (mPc); pelo anticiclone do Atlântico, centro de ação da Massa Tropical Atlântica (mTa), quente e úmida e pela depressão do Chaco, que é o centro de ação da Massa Tropical Continental (mTc), quente e seca" (\acrlong{SC}, \citeyear{AtlasSCnatureza}, pg-77).
\end{citacao}

\indent "Os elementos que constituem as condições momentâneas de tempo passam a ser denominados elementos climáticos quando utilizados para fins de estudos relacionados ao clima" (\acrlong{SC}, \citeyear{AtlasSCnatureza}, pg-75).

\subsection{Aspectos Bióticos e Abióticos}

\indent Cabe destacar que, antes da promulgação da Constituição Federal de 1988, o Estado brasileiro estabelece a \acrfull{PNMA}, que tem a preservação, melhoria e recuperação da qualidade ambiental propícia à vida como objetivo principal, conforme a luz da lei. Também define recursos ambientais, englobando fauna, flora, elementos da biosfera, solo, subsolo, mar territorial, águas interiores, estuários, águas subterrâneas, águas superficiais e atmosfera  \cite{BRASIL1981LeiPNMA}.

\indent Essa mesma política ainda apresenta definição para meio ambiente, sendo "o conjunto de condições, leis, influências e interações de ordem física, química e biológica, que permite, abriga e rege a vida em todas as suas formas" \cite{BRASIL1981LeiPNMA}. Logo, meio ambiente é um conceito amplo que, dependendo da doutrina, vai além dos aspectos naturais.  

\indent O Estado catarinense está situado no sul do Brasil, tendo limites com o Estado do Paraná ao norte e, ao sul, com o Estado do Rio Grande do Sul. Esses três Estados  formam a região Sul do Brasil. \acrlong{SC} também tem limites a leste com o oceano Atlântico e a oeste com a República Argentina.  O Estado ocupa aproximadamente 1\% da território nacional e  cerca de 16\% da área total da Região Sul (\acrlong{SC}, \citeyear{AtlasSCterritorio}).

\indent É possível notar diferentes paisagens em \acrlong{SC}; essas, criadas por formas de relevos variadas. Sob esses aspectos, pode-se observar os principais compartimentos de relevo no Estado catarinense: Planície (Costeira), Serras (do Mar, do Tabuleiro-Itajaí e Geral), Planaltos (de São Bento do Sul, de Lages, dos Campos Gerais e Dissecado Rio Iguaçu-Rio Uruguai), Patamares (de Mafra, do Alto Rio Itajaí e da Serra Geral) e Depressão (da Zona Carbonífera) (\acrlong{SC}, \citeyear{AtlasSCnatureza}).

\indent O relevo define a grande divisão regional de \acrlong{SC}: região do
litoral e encostas, e região do planalto. Somado a esse relevo singular, o clima explica o processo colonial tardio do Estado catarinense (\acrlong{SC}, \citeyear{AtlasSCpopulacao}).

\indent Como verificado no Atlas Geográfico de \acrlong{SC} (\citeyear{AtlasSCnatureza}), hipsometria do Estado catarinense é muito diversa, variando de mais de 1.800 metros ao zero, no nível do mar. Esse comportamento ocorreu pelo soerguimento epirogenético durante o período Terciário, que afetou toda a porção leste do continente Sul Americano, também por conta de processos erosivos e a diferentes resistências de rochas.

\indent O  Atlas Geográfico de \acrlong{SC} (\citeyear{AtlasSCnatureza}) cita que a zona costeira de \acrlong{SC} varia de 0 a 200 metros, atingindo 400 metros em regiões próximas ao litoral, no baixo curso do rio Itajaí-Açú e no fundo do vale do rio Uruguai e seus afluentes. A maior parte da bacia hidrográfica do Itajaí se encontra entre 400 e 800 metros, assim como parte do vale do rio Uruguai e regiões do Meio Oeste catarinense.

\indent Altitudes entre 800 e 1.200 metros são mais comuns em \acrlong{SC}, sendo decorrentes de serras, escarpas, planaltos e patamares. Nessas altitudes se encontram os divisores de água do Estado catarinense. "Por causa destas altitudes, as temperaturas são mais amenas e o clima se torna mais propício para o cultivo de frutas de clima temperado" (\acrlong{SC}, \citeyear{AtlasSCnatureza}). "Altitudes maiores do que 1.200 m são pouco comuns em Santa Catarina, [...] as elevadas altitudes permitem, em condições meteorológicas favoráveis, a ocorrência de neve..." (\acrlong{SC}, \citeyear{AtlasSCnatureza}).

\indent  O Atlas Geográfico de \acrlong{SC} (\citeyear{AtlasSCnatureza}) também comenta que remanescentes de cobertura primária ainda são existentes, porém são pequenos grupamentos e estão em locais de difícil acesso. Parte dessa fragmentação, também em coberturas secundárias, ocorre devido ao avanço agropecuário em \acrlong{SC} e o  processo de urbanização, principalmente, na segunda metade do século XX.

\indent A cobertura vegetal catarinense é biodiversa, resultante de variedades de climas, relevo e solos. A vegetação litorânea é composta por restinga e mangue. As coberturas florestais ocorrem nas encostas de serras e vales de vertente atlântica (floresta ombrófila densa), nos planaltos (floresta ombrófila mista) e vales do rio Uruguai e afluentes (floresta estacional decidual subtropical) (\acrlong{SC}, \citeyear{AtlasSCnatureza}).

\indent Em elevadas altitudes, acima de 1.500 mestros, a cobertura vegetal é composta por floresta nebular e campos de altitude. Também ocorre vegetação campestre numa altitude aproximada de 800 metros, tendo precipitação bem distribuída ao longo do ano e médias de temperatura inferiores a 15 C. "Um tipo de floresta muito típico são os Faxinais, que aparecem em altitudes entre 700 e 1.200 metros [...] são associações mistas de espécies da mata Pluvial com elementos da floresta de Araucária" (\acrlong{SC}, \citeyear{AtlasSCnatureza}).

\subsection{População Catarinense e Contexto Histórico}

\indent Como \citeonline{Lages2004Territorios} argumentam, "território é o espaço apropriado por um ator, sendo definido e delimitado por e a partir de relações de poder, em suas múltiplas dimensões". Os próprios autores ainda definem a noção de espaço, que representa uma abstração maior, tendo o território como produto da intervenção e do trabalho dos atores sobre determinado espaço. "Tal como a cultura, o território não é rígido, mas produto da dinâmica histórica das sociedades" (\acrlong{SC}, \citeyear{AtlasSCpopulacao}).

\indent Como ressaltado no Atlas Geográfico de \acrlong{SC} (\citeyear{AtlasSCpopulacao}) o território era ocupado por povos originários há milhares de anos, muito antes da colonização. Logo, para compreender a ocupação do território, deve-se analisar a distribuição florística, faunística e climática. "O desenvolvimento da vegetação na Terra, assim como as culturas agrícolas, forrageiras e florestais, ocorre conforme os tipos de clima" \cite{AtlasClimaticoSul}.

%A classificação climática é importante para se conhecer qual tipo de vegetação ou quais espécies da fauna se desenvolvem em determinada região" \cite{AtlasClimaticoSul}.

\indent Em \acrlong{SC}, são conhecidos três grupos indígenas, denominados: Guarani, Kaingang e Xokleng. "Os Xokleng no Alto Vale do Itajaí voltaram a se definir Laklãnõ recentemente". Com o processo histórico de colonização, houve redução  territorial desses povos, principalmente por uma política indigenista de aldeamento (\acrlong{SC}, \citeyear{AtlasSCpopulacao}).

\indent O grupo Guarani possuía uma agricultura mais sofisticada e ocupava terras baixas, principalmente do litoral sul brasileiro e várzeas da bacia do rio da Prata. "Há mais de um milênio foi iniciada a ocupação Guarani na costa Atlântica sul, inclusive de Santa Catarina, cujas datações mais antigas estão entre os anos 930 e 980 em Imbituba" (\acrlong{SC}, \citeyear{AtlasSCpopulacao}).

\indent Os Laklãnõ (Xokleng) se deslocavam no território entre mata Atlântica e florestas de araucárias do planalto. Os Kaingang tinham as terras altas como território, isso por associação à mitologia do grande dilúvio e por predomínio da mata de araucária, do qual proporcionava a base alimentar (\acrlong{SC}, \citeyear{AtlasSCpopulacao}).

\begin{citacao}
"Os indígenas resistiram a esse processo de esbulho do seu patrimônio e de violência física. O conceito de resistência deve ser amplo, não no sentido de luta armada, mas de estratégias cotidianas de enfrentamento aos ditames da política indigenista. [...] Atualmente, é importante salientar que os povos indígenas habitantes do Brasil, incluindo os povos que vivem em \acrlong{SC}a, estão em crescimento demográfico, sendo uma característica desse processo o reconhecimento da sua própria identidade" (\acrlong{SC}, \citeyear{AtlasSCpopulacao}, pg-47).
\end{citacao}

\indent Assim descrito por \acrlong{SC} (\citeyear{AtlasSCpopulacao}), os estados da região sul brasileira, assim como o território catarinense, foram conquistados e colonizados de forma tardia, a partir do século XVII. "Certamente dentre os elementos referentes ao quadro natural, o clima é o mais ressaltado como um dos determinantes na estruturação tardia da formação meridional brasileira" (\acrlong{SC}, \citeyear{AtlasSCpopulacao}). Seria muito difícil, para o universo colonial marcado pela tropicalidade, oferecer aos interesses comerciais da Metrópole Portuguesa uma produção que atendesse, abaixo do Trópico de Capricórnio, ao predomínio do clima subtropical.

\indent Nesse mesmo século, XVII, fazia-se necessário um sistema defensivo litorâneo da ilha de \acrlong{SC}. Porém, a partir do século XVIII, houve a incorporação do litoral catarinense ao sistema econômico colonial português, o que levou a migração de açorianos e madeirenses para o território. Também iniciando nesse período, surgiram novos povoados na estrada das tropas para pouso e descanso, sendo anexados ao território catarinense no século XIX (\acrlong{SC}, \citeyear{AtlasSCterritorio}). "... A exploração das terras do planalto começa com os paulistas no século XVIII, instalando a pecuária extensiva nas manchas de campos naturais de Lages, Curitibanos e Campos Novos" (\acrlong{SC}, \citeyear{AtlasSCpopulacao}).

\begin{citacao}
"Santa Catarina vivencia seu processo de conquista e colonização, tardio em cerca de cem anos frente ao Nordeste e Sudeste brasileiros, com povos fruto do crescimento demográfico da Colônia, caracterizando um processo migratório interno, no sentido norte-sul. Isto tanto no século XVII (Litoral), quanto ainda no XVIII (Planalto), mesmo século em que também se inicia a migração externa com populações vindas de além-mar, no caso da 2ª ocupação do litoral catarinense. A partir do século XIX tal processo imigratório se intensifica com a ocupação dos vales atlânticos e encostas pelos imigrantes europeus, excedente populacional  relativo a uma Europa em transformação frente ao avanço e desenvolvimento do modo de produção capitalista" (\acrlong{SC}, \citeyear{AtlasSCpopulacao}, pg-22). 
\end{citacao}

\indent Por meados do século XIX, algumas regiões europeias, já em fase industrial, atravessam um período de crise econômica e turbulência política. Esse cenário de crise foi provocado pela expansão do capitalismo, o que levou ao aprofundamento e ampliação do processo de expropriação, responsável pelo próprio crescimento demográfico europeu (\acrlong{SC}, \citeyear{AtlasSCpopulacao}).

\indent É nesse contexto que as transformações em áreas de capitalismo tardio levam milhões de emigrantes a cruzar o oceano em busca de novas oportunidades. "...Foram criados, principalmente por alemães e italianos, vários núcleos rurais e urbanos em áreas correspondentes aos vales atlânticos e encostas dos rios Itajaí, Cachoeira, Cubatão, Tijucas, Tubarão, Urussanga e Araranguá" (\acrlong{SC}, \citeyear{AtlasSCpopulacao}).

\indent Também pela metade do século XIX, o Governo Imperial criou mais uma colônia militar, a do Chapecó. "Implantada em 1882 em área correspondente hoje ao município de Xanxerê". Além de fazer a guarda da fronteira, também protegia a porção mais a oeste do caminho das tropas, como as primeiras colônias nas encostas do planalto a leste. No início do século XX, instalaram-se pequenas propriedades rurais policultoras (Joaçaba, Chapecó e Concórdia), por conta da expansão demográfica do Rio Grande do Sul, principalmente com a chegada da ferrovia São Paulo-Rio Grande do Sul (1908-1910), no vale do rio do Peixe (\acrlong{SC}, \citeyear{AtlasSCpopulacao}).

\indent Deve-se tomar nota da população de origem africana que, assim como os povos indígenas, foi escravizada durante os períodos colonial e imperial. "Das origens, na África, houve predomínio da região Centro-Ocidental (Congo, Angola, Benguela), com alguma presença de africanos orientais (Moçambique) e ocidentais (Mina, Jejes)". Há registros que apontam para o crescimento gradual de africanos escravizados e indígenas administrados, desde o século XVII, com a fundação de São Francisco do Sul, Desterro e Santo Antônio dos Anjos da Laguna (\acrlong{SC}, \citeyear{AtlasSCpopulacao}).

\indent Ainda assim, a captura e a escravização de indígenas estava condenada por parte das autoridades judiciais da capitania de São Paulo na década de 1720. Fato esse que intensificou o tráfico transatlântico de escravos africanos para os territórios portugueses na América nos séculos XVIII e XIX, marcando o início de uma virada demográfica. Parte desses afrodescendentes foi utilizada como mão-de-obra para contrução de fortificações no litoral, estações baleeiras, trabalho em engenhos de farinha de mandioca, açúcar, aguardente e na lida com o gado no planalto (\acrlong{SC}, \citeyear{AtlasSCpopulacao}). 

\begin{citacao}
"Desde o século XIX, a autoimagem de Santa Catarina é a de uma província ou um estado europeu, uma singularidade que nos distanciaria do restante do Brasil, marcado pela mestiçagem. A associação de “branco” com “europeu”, e deste com “civilização” e “progresso”, é um traço persistente do senso comum. Embora seja inegável tanto a presença quanto a participação de descendentes de europeus na constituição demográfica e econômica de Santa Catarina, o efeito colateral é a invisibilização da população de origem africana" (\acrlong{SC}, \citeyear{AtlasSCpopulacao}, pg-79).
\end{citacao}

\indent Não coincidentemente, o Brasil há registro de dengue desde o século XIX, relatado em literatura por epidemias no Rio de Janeiro (1846) e em São Paulo (1852) \cite{Valle2015Dengue}.

\indent Recentemente, em especial após a década de 1980, a taxa de fecundidade em \acrlong{SC} se apresenta inferior à nacional, porém a população catarinense cresce a taxas superiores às nacionais. Esse motivo pode ser explicado em razão da força do processo migratório no Estado, como polo de absorção regional, nacional e mesmo internacional de migrantes (\acrlong{SC}, \citeyear{AtlasSCpopulacao}).

\indent Nesse mesmo período, o grau de urbanização praticamente dobrou, chegando  a 83,7\% em 2015. Em relação ao crescimento populacional e urbano, essa maior elevação ocorreu,  principalmente, nas regiões Norte, Vale do Itajaí e Grande Florianópolis (\acrlong{SC}, \citeyear{AtlasSCpopulacao}). 

\section{Contextualização sobre Saúde}

\subsection{Saúde para Ciências Exatas e da Terra}
%\begin{center}
%\textcolor{red}{CONSIDERAÇÕES AQUI!}\\
%textcolor{red}{JOÃO: Geomedicina (Geografia médica, Geografia da saúde, medicina social, espaço).}\\
%\end{center}

\indent Sendo o \acrfull{PCAM} vinculado a Ciências Exatas e da Terra, particularmente dentro das GeoCiências, de acordo com sistematização das áreas de conhecimento da \acrfull{CAPES} (\citeyear{CAPES_Tabela_Conhecimento}) e publicizado pelo \acrfull{CNPq} \citeyear{CNPq_Tabela_Conhecimento}), deve-se atentar para nomenclaturas e abordagens sobre saúde.

\indent Segundo \citeonline{Geomedicine2012Davenhall}, a Geomedicina é definida como uma nova área de inteligência médica que utiliza dados e infraestrutura espaciais para benefício da própria saúde humana; para \citeonline{Geomedicine1990JulLag}, é uma área da ciência que observa influências e relações do meio ambiente com a distribuição espacial de agravos em saúde de homens e animais.

\indent A Geografia Médica, assim como a Geografia da Saúde, tem influência de aspectos sociais, econômicos, políticos, tecnológicos e ambientais. Dessa maneira, exerce importante relevância, visto que esses aspectos, na maioria das vezes, são os fatores determinantes no estudo da saúde-doença de uma população \cite{GeografiaMedica2020}.

\indent Houve pedido de alteração da denominação, de Geografia Médica para Geografia da Saúde, devido à ampliação dos temas, questões e abordagens que esta foi desenvolvendo ao logo do tempo \cite{GeografiaMedica2009}.

\indent Ainda descrito por \citeonline{GeografiaMedica2020}, há enfoque na Geografia da Saúde com a situação econômica, social e ambiental em que este indivíduo está sujeito no cotidiano, influenciando seu processo saúde-doença. Por outro lado, na Geografia Médica há análises e estudos epidemiológicos.

\indent \citeonline{GeografiaMedica2009} argumenta que, a partir da década de 1950 e com o interesse geopolítico pela interiorização e integração do interior do território brasileiro, a Geografia Médica no Brasil resultou em pesquisas sobre doenças ditas tropicais presentes nas áreas de interesse de ocupação, principalmente da Amazônia e do Centro-Oeste.

\begin{citacao}
"Esses estudos atendiam ao interesse do governo que implantava projetos de produção de energia, agropecuária e de mineração no interior do país, no entanto, não possuíam maior reflexão sobre os problemas relacionados à saúde. Pode-se dizer que esse período foi marcado por uma Geografia Médica atrelada aos interesses do governo e da classe dominante, não havendo uma análise mais crítica que relacionasse esses estudos aos fatores
socioeconômicos e culturais" (\citeauthor{GeografiaMedica2009}, \citeyear{GeografiaMedica2009}, pg-6).
\end{citacao}

%\indent Medicina social/espaço. RENE AREIA, Associação de Medicina Social da Bélgica "enfatizou a importância do social, político e cultural na origem e persistência de doenças epidêmicas".

%\subsection{Saúde: Coletiva e Única}

\subsection{Saúde Coletiva}

\indent Durante a antiguidade grega, precisamente no século V a.C., Hipócrates defendia a idéia de que um ambiente saudável estava ligado com a própria saúde da população. Ele é considerado o pai da Medicina e adotava a visão de Saúde Pública integrada ao ambiente \cite{CFMVSaude}.

\indent De acordo com \citeonline{Foucault1990Microfisica}, a passagem de uma medicina coletiva para uma medicina privada não se deu com o capitalismo, muito pelo contrário, primeiro socializou o corpo enquanto força de trabalho. O controle social sobre os indivíduos começa no corpo, com o corpo, pois este é uma realidade biopolítica, sendo a medicina parte de sua estratégia. Porém, a socialização do corpo só ocorreu em última etapa do surgimento da Medicina Social, que teve sua formação a partir de três (3) etapas: Medicina de Estado, Medicina Urbana e Medicina da Força de Trabalho.

\indent A primeira etapa ocorreu durante o século XVIII na Alemanha, momento em que a burguesia, em queda, oferecia seus serviços à organização Estado. Nesse período, a partir da medicina, o Estado era responsável por fazer observações e registros, não apenas de nascimento e mortalidade, como também de morbidade. Ainda no final do século XVIII, na França, com a urbanização da cidade, era necessário ordenamento territorial por conta da densidade demográfica e da heterogeneidade política e sanitária \cite{Foucault1990Microfisica}.

\indent Assim, a medicina fazia ordenação do espaço urbano, realizando análise de perigo em áreas de aglomero, principalmente cemitérios, assim como controlando a circulação de água e ar, tanto de afluentes como de efluentes. A última etapa se deu na Inglaterra ao final do século XIX, com o desenvolvimento industrial, e por consequência, do proletariado. "A partir do momento em que o pobre se beneficia dos sistema de assistência, deve, por isso mesmo, se submeter a vários controles médicos" \cite{Foucault1990Microfisica}.

\indent Nesse momento, a ciência reformulava as relações entre o homem e suas condições de vida, incitando a medicina como assunto público, devendo intervir na vida política e social. Essas ideias são voltadas para as reformas da saúde e alguns princípios básicos se tornariam parte integrante do discurso sanitarista, sendo: que a saúde da população é de interesse societário e cabe à sociedade proteger e assegurar a saúde de seus membros; que as condições sociais e econômicas têm um impacto crucial sobre o processo saúde-doença, devendo ser estudadas cientificamente; que as medidas a serem tomadas para a proteção da saúde são tanto sociais como médicas \cite{TratadoSaudeColetiva}.

\indent Ainda, sob as palavras de \citeonline{Foucault1990Microfisica}, saúde e salubridade não deveriam ser vistos como iguais. "Salubridade é a base material e social capaz de assegurar a melhor saúde possível dos indivíduos. [...] Salubridade e insalubridade são o estado das coisas e do meio enquanto afetam a saúde"  \cite{Foucault1990Microfisica}.

\indent Adotando o conceito de saúde da Carta Magna para Saúde Mundial, desde planejamento de criação até a execução atual da \acrfull{OMS}, tem-se saúde como um estado de completo bem-estar físico, mental e social, e não apenas como a ausência de doença ou enfermidade \cite{OMS2024S1, ParranHEALTH}.

\indent  Na metade do século XX, concomitantemente ao final da Segunda Grande Guerra (1939-1945), quando se aumenta a utilização de antibióticos e técnicas cirúrgicas, reconsolida-se a atenção médica individualizada, repensando sobre conceitos de saúde. Nesse período, a \acrfull{OPAS} apoia a ampla discussão crítica sobre o modelo biomédico da medicina, momento em que há criação de departamentos de medicina preventiva e social nos projetos pedagógicos \cite{TratadoSaudeColetiva}.

\indent Muitas análises sociais, demográficas e políticas percorreram a história da saúde pública, percebendo vínculo às políticas de saúde que se desenvolveram, na Europa e nas Américas, e trouxeram em seus conteúdos as especificidades de cada contexto histórico e suas circunstâncias \cite{TratadoSaudeColetiva}.

\indent Ao final dos anos 70, para \citeonline{TratadoSaudeColetiva}, dá-se início à Saúde Coletiva, com fundação de associação própria a nível nacional e assumindo posição técnico-política, que teria a concepção de saúde como direito do cidadão e dever do Estado. A Saúde Coletiva encontra ruptura com a concepção de Saúde Pública, não aceitando o próprio discurso biológico, sendo um mosaico entre as ciências sociais e humanas, a epidemiologia, a política e o planejamento. A presença das ciências sociais e humanas foi de grande importância e consolidou como fundamentais para a compreensão dos processos da vida, saúde, doença e morte.

\begin{citacao}
\indent "...Tornou-se difícil um consenso acerca de sua conceituação. Em realidade, a partir do momento em que se foram firmando as formas de tratar o coletivo, o social e o público, caminhou-se para entender a saúde coletiva como um campo estruturado e estruturante de práticas e conhecimentos, tanto teóricos como políticos" (\citeauthor{TratadoSaudeColetiva}, \citeyear{TratadoSaudeColetiva}, pg-27).
\end{citacao}


\indent Durante a década de 90, a Lei Orgânica da Saúde foi sancionada no Brasil. Essa lei dispõe sobre as condições  para a promoção, proteção  e  recuperação  da  saúde,  a  organização  e  o  funcionamento  dos  serviços correspondentes, além de também dispor sobre o Sistema Único de Saúde. Está expresso em seu segundo artigo que saúde é um direito fundamental e prover condições para seu pleno exercício é dever do Estado \cite{BRASIL1990LeiSUS}. Seu terceiro artigo tem a inclusão, por meio de lei, de fatores determinantes e condicionantes, ficando o texto assim:

\begin{citacao}
"Os níveis de saúde expressam a organização social e econômica do País, tendo a saúde como determinantes e condicionantes, entre outros, a alimentação, a moradia, o saneamento básico, o meio ambiente, o trabalho, a renda, a educação, a atividade física, o transporte, o lazer e o acesso aos bens e serviços essenciais "
(\citeauthor{BRASIL2013LeiDeterminantesCondicionantesSaude}, \citeyear{BRASIL2013LeiDeterminantesCondicionantesSaude}, pg-1).
\end{citacao}

\indent Nesse mesmo artigo da Lei Orgânica da Saúde, o parágrafo único continuou inalterado, mantendo o seguinte texto: "Dizem respeito também à saúde as ações que, por força do disposto no artigo anterior, se destinam a garantir às pessoas e à coletividade condições de bem-estar físico, mental e social" \cite{BRASIL1990LeiSUS}.


% \indent Abordagens articuladas à saúde: Foucault, Butler e Laqueur

% \indent Maria Cecília Minayo (2001), reforma sanitária.

% \indent Convergências epistemológicas entre a Bioética e saúde coletiva discutidas por Junges
% e Zoboli (2012)

% \indent Segundo Marsiglia (2013) o uso do termo “coletivo” no campo da saúde coletiva é de
% fundamental importância, haja vista que
% https://www.teses.usp.br/teses/disponiveis/5/5137/tde-09082017-100757/publico/MarceloJosedeSouzaeSilva.pdf


\subsection{Saúde Única}

\indent Ainda na segunda metade do século XX, exatamente em 1968, a \acrfull{OPAS} convoca a \acrfull{RIMSA}, realizando proteção e promoção da saúde humana e animal por meio da cooperação técnica em saúde pública veterinária, além de manter enfoque multissetorial \cite{S1_OPAS_OMS}.

\indent O termo Medicina Única foi concebido pelo médico-veterinário Calvin W. Schwabe (1927-2006) em sua obra "\ingles{Veterinary Medicine and Human Health}” em 1984, reforçando a importância da junção entre saúde humana, animal e ambiental \cite{CFMVSaude}. Aos poucos, o mundo foi despertando para a importância da interação entre saúde e meio ambiente.

\indent Durante a Conferência das Nações Unidas sobre Meio Ambiente e Desenvolvimento no Rio de Janeiro em 1992, conhecida como EcoRio92, discutiram-se as bases para avançar esforços conjugados quanto à colaboração entre setores de saúde e meio ambiente. Logo, em 1995, foi adotada a Carta Pan-Americana sobre Saúde e Meio Ambiente no Desenvolvimento Humano Sustentável \cite{S1_OPAS_OMS}, porém a evolução do termo para Saúde Única ocorreu no século XXI \cite{CFMVSaude}. 

\indent No ano de 2004, em Nova Iorque, especialistas se reuniram para debater e propor ações em resposta à potencial dinâmica de doenças entre humanos, animais domésticos e fauna silvestre. Naquela época, principalmente, alertar sobre ebola, influenza aviária, doença crônica debilitante dos cervídeos, encefalopatia espongiforme bovina, dentre outras. Esses encontros resultaram na elaboração de prioridades e recomendações, chamadas de "Princípios de Manhattan" durante o simpósio "Um Mundo, Saúde Única", que apresentavam visão holística e caráter interdiscipinar sobre saúde, dando ênfase na prevenção de epidemias/epizootias e na conservação íntegra dos ecossistemas \cite{ManhattanPrinciples2004}.

\indent Parte dessas recomendações eram para reconhecer a ligação da saúde entre animais domésticos, silvestres e sociedade; alertar para o uso do solo e da água; reduzir o consumo de carne de caça e de alimentos com baixa inocuidade; aumentar investimentos em pesquisas sobre saúde de maneira transdisciplinar e multissetorial; investir em educação, não apenas acadêmica, mas da população. A saúde deveria ser entendida e planejada na interface homem-animal-ecossistema, atingindo, assim, a Saúde Única \cite{ManhattanPrinciples2004}.

\indent Inicialmente, em 2008, a abordagem sobre Saúde Única era feita de forma Tripartite, entre a \acrfull{OMS}, \acrfull{FAO} e \acrfull{OMSA}. Esta última foi fundada como Gabinete Internacional de Epizootias (\ingles{\acrfull{OIE}}) %\cite{S1Quadripartite}.

\indent Após a junção dessas três organizações, em junho de 2021, foi proposta uma quarta parte para compor essa aliança, com a inclusão do \acrfull{PNUMA}. Com uma gestão Quadripartite, desenvolve-se o \acrfull{SGISU}, que tem por objetivo melhorar a inteligência em Saúde Única, antecipando avisos e auxiliando na gestão de risco de ameaças à saúde global %\cite{S1Quadripartite}.

%"É importante notar que a definição do preâmbulo sugere uma motivação idealista em favor da igualdade universal, que era novo, especialmente em muitos governos europeus, mesmo após a Segunda Guerra, e, além disso, o preâmbulo ligava saúde com termos como paz. Inspirados pelo postulados da medicina social, saúde pública não devia ser um produto isolado do resto da vida social, mas um processo intrínseco de desenvolvimento social", afirmou Marcos Cueto. --- %https://www.coc.fiocruz.br/index.php/pt/todas-as-noticias/319-saude-internacional-e-as-origens-da-oms\\
%DECRETO Nº 26.042, DE 17 DE DEZEMBRO DE 1948 --- https://www2.camara.leg.br/legin/fed/decret/1940-1949/decreto-26042-17-dezembro-1948-455751-publicacaooriginal-1-pe.html\\
%https://edisciplinas.usp.br/pluginfile.php/5733496/mod_resource/content/0/Constitui%C3%A7%C3%A3o%20da%20Organiza%C3%A7%C3%A3o%20Mundial%20da%20Sa%C3%BAde%20%28WHO%29%20-%201946%20-%20OMS.pdf\\

\indent Ainda para a \acrshort{OMS} (\citeyear{OMS2024S1}), Saúde Única é uma abordagem na interface humano-animal-ambiental para se alcançar melhores resultados em saúde coletiva. Vários setores que se intercomunicam, especialmente para o controle de zoonoses, promoção de segurança alimentar e combate a resistência a antibióticos, tendo interdependência e ligação entre saúde humana e saúde animal por meio de saúde de ecossistemas; para a \citeonline{FAO2022} (\acrshort{FAO}), há sinergismo entre essas políticas e estratégias, enquanto é entendido pelo \citeonline{CFMVSaude} (\acrshort{CFMV}) como uma união indissociável.

\begin{citacao}
"Saúde Única é um enfoque colaborativo, multidisciplinar e multissetorial que pode abordar as ameaças à saúde na interface homem-animal-ambiente no âmbito subnacional, nacional e internacional, com o objetivo final de obter resultados de saúde ótimos reconhecendo as interconexões entre pessoas, animais, plantas e meio ambiente. Essa  interface, um aspecto definidor de Saúde Única, consiste num continuum de interações entre pessoas, animais e meio ambiente que permite a transmissão entre espécies de patógenos zoonóticos e emergentes" (\citeauthor{S1_OPAS_OMS}, \citeyear{S1_OPAS_OMS}, pg-2).
\end{citacao}


\indent No atual ano, 2024, passa a vigorar a lei que institui o Dia Nacional de Saúde Única, a ser celebrado no dia 3 de novembro, "com o objetivo de conscientizar a sociedade sobre a relação indissociável entre as saúdes animal, humana e ambiental" \cite{BRASIL2024LeiS1}.

\indent O próprio Ministério da Saúde reconhece o termo Saúde Única, e sugere a sinonímia "Uma Só Saúde", com a mesma abordagem integrada que reconhece a conexão entre a saúde humana, animal, vegetal e ambiental. Esse Ministério é tido como o órgão do Poder Executivo Federal responsável pela organização e elaboração de planos e políticas públicas voltados para a promoção, a prevenção e a assistência à saúde dos brasileiros \cite{MinisterioSaudeS1}.

\indent Sobre a relação entre saúde humana, animal, vegetal e ambiental, o \citeonline{MinisterioSaudeS1} elenca quatro (4) tópicos, sendo: Doenças zoonóticas e novas epidemias/pandemias; Resistência aos antimicrobianos; Segurança alimentar e segurança dos alimentos; e Biodiversidade, mudanças climáticas e saúde.

% https://www.gov.br/saude/pt-br/assuntos/saude-de-a-a-z/u/uma-so-saude

% https://www.gov.br/saude/pt-br/assuntos/saude-de-a-a-z/u/uma-so-saude/doencas-zoonoticas

% https://www.gov.br/saude/pt-br/assuntos/saude-de-a-a-z/u/uma-so-saude/biodiversidade

\indent Neste mesmo ano, 2024, o \acrshort{CFMV} emite a Portaria nº 115, que cria e nomeia a \acrfull{CNSU} do Conselho Federal. Essa Portaria considera  a necessidade de criar políticas que assegurem à população qualidade em saúde, sendo o médico-veterinário um  profissional importante na equipe multidisciplinar dessa área (\acrlong{CFMV}, \citeyear{CFMV2024PORTARIA}).

\indent Também é incluído, no conceito de Saúde Única, a conexão com saúde vegetal, além da abordagem integrada entre saúde humana, animal e ambiental. Ao \acrshort{CNSU} fica atribuído: analisar, sugerir, articular e propor políticas de Saúde aos órgãos competentes, além de demais competências e representações perante o \acrshort{CFMV} (\acrlong{CFMV}, \citeyear{CFMV2024PORTARIA}).

\indent Para a \citeonline{ONUODS22} (\acrshort{ONU}), como objetivo de desenvolvimento sustentável, deve ser assegurado uma condição de vida saudável e promoção de bem-estar para todas e todos, em todas as idades; além de combater doenças veiculadas pela água e outras doenças transmissíveis. Também é tido de suma importância a integração das medidas da mudança do clima nas políticas, estratégias e planejamentos nacionais.

\indent Ainda segundo a \acrshort{ONU} (\citeyear{ONUODS22}) é necessário implementar medidas para evitar a introdução de espécies exóticas e reduzir significativamente o impacto dessas espécies invasoras.

\subsection{Doenças Zoonóticas}

\indent No século XIX, o médico patologista Rudolf L K. Virchow (1821-1902) foi um dos primeiros a utilizar o termo Zoonose e afirmava que “entre a medicina animal e a medicina humana não havia divisórias; e que nem deveria haver” \cite{CFMVSaude}.

\indent Como citado por \citeonline{HumanAnimalInterface}, a própria interface homem-animal apresenta características versáteis e dinâmicas. Para esses autores, a ligação homem-animal é caracterizada por muitos atributos adquiridos durante o próprio processo de evolução humana como espécie e o desenvolvimento do homem como sociedade, porém se mantém em processo dinâmico contínuo.

\indent O principal atributo é a co-evolução patogênica herdada pelo homem como espécie animal, além de outros atributos importantes, tais como: comportamento humano durante o processo de domesticação de espécies; a perda do comportamento nômade-coletor para se tornar agricultor e produtor de alimentos; amplos processos migratórios, de colonização e comercialização;  e aos próprios processos de urbanização, industrialização e globalização \cite{HumanAnimalInterface}.

\indent A \acrshort{OMS} (\citeyear{WHO2020Zoonoses}) considera zoonoses as doenças infecciosas trasmitidas, via ingestão hídrica ou sólida, pelo meio ambiente ou por contato direto, entre humanos e animais não-humanos. Elas representam o principal problema em saúde coletiva no mundo, além de corresponderem a 75\% das doenças infecciosas emergentes em humanos. Seus agentes etiológicos podem ser de caráter bacteriano, viral, parasitário ou envolver agentes não convencionais.

\indent Pensando nos animais como potenciais reservatórios para a transmissão da dengue, em uma revisão sistemática, constataram que aproximadamente 10\% dos animais testados apresentaram positividade para \acrshort{DENV}, tanto PCR quanto sorologia. Analisando as sorologias, tem-se positividade para: primatas não-humanos (26,94\%), morcegos (7,04\%), marsupiais (3,41\%) e roedores (0,67\%). Ao se analisar os RT-PCR, tem-se positividade para:  marsupiais (9,54\%), morcegos (3,04\%), roedores (2,78\%) e primatas não-humanos (0,32\%)  \cite{DengueAnimalsGwee2021}.

\indent Em outra revisão sistemática e meta-analítica, \citeonline{Aldana2024DengueAnimals} encontraram positividade sorológica \acrshort{DENV} em: suínos (49\%), primatas não-humanos (29\%), morcegos (10\%), equinos (11\%), aves (8\%), bubalinos (7\%), roedores (2\%) e ovinos (1\%). Também analisaram resultados positivos por RT-PCR em morcegos, apresentando positividade em 6\% dos animais avaliados.

\indent Ambos os estudos reconhecem os primatas não-humanos como potenciais hospedeiros em ciclos enzoóticos para amplificação da transmissão, porém constataram que nem todos as espécies de primatas não-humanas são suscetíveis. Também entram em acordo ao encontrar resultados positivos por métodos moleculares em morcegos e dão ênfase ao dizer que o vírus da dengue pode infectar diversas espécies de animais, incentivando estudos relacionados \cite{Aldana2024DengueAnimals, DengueAnimalsGwee2021}.

\indent Em estudo de \citeonline{DengueDog2017First}, uma pequena quantidade de cães domiciliados apresentaram positividade para dengue em amostras de sangue testadas por RT-PCR, isolamento viral e replicação em cultura celular. Desses cães, a maioria era de região urbana e parte deles era domiciliado internamente na residência, tendo contato com o ambiente externo apenas para passeio. Os autores não descartam a possibilidade de cães serem reservatórios para o vírus, mas alertam necessidade de mais estudos.

\subsection{Ecoepidemiologia}

% https://www.who.int/news-room/fact-sheets/detail/dengue-and-severe-dengue
% https://bvsms.saude.gov.br/bvs/publicacoes/modulo_principios_epidemiologia_2.pdf
\indent A epidemiologia é a ciência que estuda a frequência, a distribuição e os determinantes dos eventos de saúde nas populações. Ao se estudar a distribuição de doenças sobre populações, três variáveis são clássicas: tempo, lugar e pessoa \cite{MOPECE2010}.

\indent No enfoque epidemiológico, não apenas se pesquisa a distribuição dos acometimentos, mas se orienta a busca de porquês sobre a ocorrência de eventos. Esse processo utiliza modelos de causalidade, sendo os mais aceitos: Tríade Epidemiológica e Causas Componentes \cite{MOPECE2010}.

\indent A Tríade Epidemiológica, como modelo tradicional de causalidade de doenças transmissíveis, aborda a doença como resultado da interação entre o agente, o hospedeiro suscetível e o ambiente. Para doenças transmitidas por vetores, o próprio vetor é um elemento de interação entre agente, hospedeiro e ambiente \cite{MOPECE2010}. 

\indent A dengue é uma doença viral de grande importância na saúde coletiva e sua forma epidêmica tem se adaptado ao ambiente urbano durante centenas de anos. \cite{ArboviralTransmission}. Atualmente é um agravo reemergente e registrado nos cinco (5) continentes, sendo endêmico em mais de 100 países e considerado como doença tropical negligenciada \cite{Valle2015Dengue}.

\indent A dengue, como registro de nomenclatura médica, apenas foi oficializada em 1983, praticamente um século após sua descoberta. O vírus teria origem asiática, filogeneticamente, e teve êxodo florestal ao mesmo passo em que o homem desmatava e desenvolvia assentamentos \cite{Valle2015Dengue}.

\indent Relatos de uma doença clinicamente compatível remonta a tempos antigos, sendo registrados na China desde o século III, durante a dinastia Chin. Naquele período era denominado 'veneno d'água', fazendo alusão a insetos voadores e distribuição de água \cite{Valle2015Dengue}.

\indent Ainda comentado por \citeonline{Valle2015Dengue}, após um longo período sem registros históricos, uma afecção semelhante torna a ser relatada nas Ilhas Francesas, 1635, e no Panamá, 1699. Quase um século após, meados de 1700, tal doença passa a ter distribuição ampla e tropical. "Essa distribuição generalizada coincidiu com o aumento do comércio global, em especial com as grandes colonizações e o tráficos de escravos[sic]".

\indent Em se tratando do agente etiológico, e como citado por \citeonline{Fiocruz2010Atlas}, o vírus da dengue (\acrfull{DENV}) faz parte da família \latim{Flaviviridae} e possui estrutura genética formada por ácido ribonucleico de fita simples e sentido positivo (\acrfull{ssRNA+}). Essa mesma autora simplifica a estrutura do vírion em 3 partes: material genético, capsídeo viral e envelope.

\indent Aprofundando um pouco mais a análise da estrutura viral, percebe-se que o capsídeo tem composição proteíca (proteína-C) e que o envelope é formado por bicamada fosfolipídica com proteínas de membrana (proteína-M) e proteínas de envelope (proteína-E). Essas proteínas de membrana que conferem a infectividade. O tamanho do capsídeo é de 30nm, enquanto o vírus como um todo pode chegar a 65 nm \cite{Fiocruz2010Atlas}. 

\citeonline{ArboviralTransmission} também comentam que muitas pandemias tem sido atribuídas a capacidade de alguns vírus \acrshort{ssRNA+} adaptarem e incluírem o ser humano como hospedeiros. Esses mesmo autores também citam que, para a dengue, essas mutações resultaram em adaptação ao mosquito urbano.

\indent Os arbovírus são aqueles vírus que infectam, em primeiro momento, artrópodes vetores e são transmitidos por esses invertebrados a vertebrados, em segundo momento, por meio da picada. Especificamente com a dengue, os mosquitos vetores são do gênero \latim{Aedes}, que transmitem o vírus entre primatas, incluindo o homem \cite{Valle2015Dengue}.

\indent Conforme adotado pela \citeonline{SBPGlossario} (\acrshort{SBP}), vetor é um artrópode, molusco ou veículo que transmite um parasito entre dois hospedeiros, que albergam o parasito; sendo vetor biológico quando o agente etiológico apresenta parte do ciclo biológico, multiplicando-se ou se desenvolvendo, no próprio animal vetor.

\indent A competência vetorial se refere a capacidade de o vetor permitir entrada e replicação do patógeno no seu próprio organismo, tornando-se infectante. A capacidade vetorial, por sua vez, é referida como a capacidade de o vetor transmitir o agente etiológico. Ambas as expressões matemáticas são aferidas em experimentos laboratoriais  \cite{OTPCampo}.

\indent Logo, a dengue é uma arbovirose, ou seja, é uma doença vetorizada por artrópodes. Ainda, por ampla distribuição (cosmopolita) de algumas espécies do gênero \latim{Aedes}, essas são consideradas as principais transmissoras no ciclo urbano (sinantrópicas antropofílicas): \latim{Aedes aegypti} (endofílico) e \latim{Aedes albopictus}  (peridomiciliar exofílico)\cite{ArboviralTransmission}.

\indent Os mosquitos do gênero \latim{Aedes} (subgênero \latim{Stegomyia}) tem distribuição biogeográfica nas principais regiões: Afrotropical, Australasiana e Oriental. Somente os mosquitos \latim{Aedes aegypti} e \latim{Aedes albopictus} ampliaram sua distribuição, em concordância com a distribuição humana, para outras regiões, sendo elas: Neoártica, Neotropical e Paleoártica  \cite{Valle2015Dengue}.

\indent Os pernilongos \latim{Aedes aegypti} apresentam origem biogeográfica na região Afrotropical, enquanto o \latim{Aedes albopictus}, na região Oriental. Nessas mesmas regiões ocorre também o ciclo rural de transmissão do vírus, onde humanos são acometidos com maior frequência \cite{Valle2015Dengue}.

\indent Nessas regiões, Afrotropical e Oriental, ainda há manutenção do ciclo silvestre, por mosquitos silvestres do gênero \latim{Aedes} transmitindo entre primatas não-humanos, onde humanos são acometidos ocasionalmente \cite{Valle2015Dengue}.

\indent Situado por \citeonline{Valle2015Dengue}, essas características originárias podem explicar diferenças na biologia dos vetores. Enquanto o \latim{Aedes aegypti} apresenta ovos com maior resistência à baixa umidade e elevada temperatura,  os ovos de \latim{Aedes albopictus} podem entrar em diapausa e eclodir assincronamente, assim como os adultos toleram uma faixa maior de temperatura, sendo capazes de colonizar áreas de invernos mais rigorosos.

\indent Ambos os mosquitos do gênero \latim{Aedes} desenvolvem seus ciclos de vida passando pelas fases de: ovo, larva, pupa e adulto alado. As três (3) primeiras fases são dependentes de ambiente aquático e os ovos são resistentes a períodos de seca. Com o aumento da umidade, os ovos eclodem e as larvas passam a viver na coluna d'água, subindo à superfície para respirar \cite{OTPCampo}.

\indent Durante a fase de pupa, os indivíduos ficam localizados próximos à superfície, com os tubos respiratórios para fora da água. As últimas transformações acontecem em poucos dias, emergindo, assim, o indivíduo adulto na sua forma alada. Apenas as fêmeas realizam repasto sanguíneo, hematofagia, \cite{OTPCampo}.

\indent Como argumentado pelos autores \citeonline{Valle2015Dengue}, "acredita-se que o \latim{Aedes aegypti} tenha sido introduzido no Brasil em dois momentos. A primeira, durante o tráfico negreiro [...] no período colonial; e a segunda, no fim dos anos 1960, após ter sido considerado erradicado do país em 1955".

\indent Em relação aos hospedeiros, inicialmente era mantido o vírus entre primatas não-humanos através de mosquitos acrodendrofílicos no ciclo silvestre, acometendo humanos acidentalmente. Não há registro de ciclo silvestre nas Américas, onde as características são essencialmente urbana e suburbana \cite{Valle2015Dengue}.

\indent "Em algum momento no passado, humanos ou mesmo primatas não-humanos adquiriram a infecção por meio da exposição à picada de mosquitos silvestres infectados pelo vírus dengue ao frequentar o ambiente selvagem" \cite{Valle2015Dengue}.

\indent No ambiente modificado, a propagação do vírus se dá por \latim{Aedes aegypti} e \latim{Aedes albopictus}, tornando o ciclo totalmente independente de reservatórios silvestres \cite{Valle2015Dengue}.

\subsection{Emergências Ambientais}

\indent Para iniciar, ao se tratar de riscos relacionados a desastres, refere-se ao potencial de ocorrer algo danoso para a sociedade. Para a \acrfull{Cobrade}, os desastres podem ser categorizados em naturais ou tecnológicos. Desses, os desastres naturais ainda são agrupados em cinco (5): geológico, hidrológico, meteorológico, climatológico e biológico. Esse último desastre é dividido em dois (2) tipos: epidemias e infestações/pragas [sic]  \cite{GIRD}.

\indent As epidemias são divididas em quatro (4) subtipos,  quanto ao agente etiológico: viral, bacteriano, parasítico ou fúngico. Especificamente para as doenças virais, esses desastres são interpretados como "aumento brusco, significativo e transitório da ocorrência de doenças infecciosas geradas por vírus", tendo 1.5.1.1.0 como código \acrshort{Cobrade} \cite{GIRD}.

\indent Segundo o \citeonline{CFMVSaude}, 80\% dos patógenos com potencial para bioterrorismo são zoonóticos. Além de impactos na saúde humana, por serem altamente letais ou incapacitantes, as zoonoses apresentam grande impacto na economia. Dessa maneira, o controle sanitário precisa ser efetivado, principalmente sob a ótica da saúde pública, embora muitas vezes as doenças zoonóticas sejam negligenciadas.

\indent \citeonline{Cubas2014Tratado} confirmam a importância para as sociedades contemporâneas sobre os impactos de mudança climática global, de emergências ambientais e de modificação antrópica dos ecossistemas naturais. Os efeitos desses impactos se estendem à integridade do meio ambiente e à saúde coletiva, além de afetar diretamente a economia. As avaliações sistemáticas de vulnerabilidade socioambiental e de saúde em relação às modificações de larga escala do meio ambiente são urgentes, em seu sentido biofísico. 

\begin{citacao}
"Alguns estudos demonstram como a dinâmica ecoepidemiológica destes agravos tem sido afetada por mudanças ambientais. [...] Inventários faunísticos e microbiológicos, cenários de clima e seus efeitos em ecossistemas, e a implantação de sistemas permanentes e eficazes de monitoramento bioclimático, são aspectos a serem considerados pelos gestores públicos e pelas comunidades científica e conservacionista" \cite[pg-2325]{Cubas2014Tratado}.
\end{citacao}

%\begin{citacao}
%“Infere-se que, em primeiro lugar, o poder de ação dos médicos-veterinários engloba a monitoração e análise dos indicadores epidemiológicos. Em um segundo momento, conclui-se que os médicos-veterinários, vinculados ou não ao Programa Nacional de Controle da Dengue (PNCD), têm uma importante responsabilidade na geração de propostas de prevenção da dengue, chikungunya e zika” \cite[pg-12]{Silva2016O}.
%\end{citacao}
%\textcolor{red}{INCLUIR\\Regulamento Sanitário Internacional (2005) e Emergência de Saúde Pública de
%Importância Nacional (Decreto 7.616/2011).}
%\begin{center}
%\textcolor{red}{CONSIDERAÇÕES AQUI!}\\ 
%\end{center}

\indent O \acrfull{RSI} representou um marco para a Saúde Pública entre países, tendo como propósito a prevenção, proteção, controle e resposta à saúde pública frente a riscos internacionais na propagação de doenças. Esse regulamento tem princípios planejados para evitar interferências não necessárias em tráfego e comércio, pondo regramento no trânsito de cargas e pessoas \cite{Brasil2005RegulamentoSI}.

\indent O controle de pontos de entrada e saída, passagens de fronteiras terrestres, portos e aeroportos se dará respeitando a dignidade, os direitos humanos e as liberdades fundamentais das pessoas. Também poderão ser exigidos documentos de saúde, tais como: certificados de vacinação ou outras medidas profiláticas; declaração marítima de saúde; e certificados de controle sanitário da embarcação \cite{Brasil2005RegulamentoSI}.

\indent Alguns anos após a publicação do \acrshort{RSI}, o decreto sobre \acrfull{ESPIN} é sancionado no Brasil, sendo declarado emergência em situações que demandam urgentemente de medidas para prevenção, controle e contenção de riscos, danos e agravos à saúde pública no território brasileiro \cite{Brasil2011ESPIN}.

\indent Será declarada emergência a nível nacional em virtude de eventos epidemiológicos, desastres ou desassistência à população. Em relação a situações epidemiológicas, serão consideradas as que apresentem risco de disseminação nacional, representem a reintrodução de agentes infecciosos erradicados, sejam produzidos por agentes infecciosos inesperados, apresentem gravidade elevada ou extrapolem a capacidade de resposta da direção estadual do \acrfull{SUS} \cite{Brasil2011ESPIN}.

\indent Para os casos de Emergência em Saúde Pública, fica a cargo da \acrfull{SVSA}, pelo próprio Departamento de Emergência em Saúde Pública, a coordenar a preparação, a vigilância e a resposta a essas emergências \cite{SVSA2023_2}.

\indent O \acrfull{CNS} institui em 2018 a \acrfull{PNVS}, entendendo a própria Vigilância em Saúde como um processo sistemático e contínuo "de coleta, consolidação, análise de dados e disseminação de informações sobre eventos
relacionados à saúde, [...], para a proteção e promoção da saúde da população, prevenção e controle de riscos, agravos e doenças" \cite{Brasil2018PNVS}.

\indent Por meio da própria \acrshort{PNVS}, é citado sobre Vigilância em Saúde Ambiental, tendo como finalidade a recomendação e adoção de medidas de promoção à saúde, prevenção e monitoramento dos fatores de riscos relacionados às doenças ou agravos. Logo, define-se Vigilância em Saúde Ambiental, sendo: "o conjunto de ações e serviços que propiciam o conhecimento e a detecção de mudanças nos fatores determinantes e condicionantes do meio ambiente que interferem na saúde humana" \cite{Brasil2018PNVS}.

\indent A Vigilância em Saúde Ambiental deve ter como base a organização política, territorial, social e cultural, assim como os modelos locais de produção. Por isso, a inter-relação entre saúde humana e meio ambiente em um território deve ser identificada e compreendida, pois é tida como essencial para o planejamento e execução de ações. "A análise do território e dos fatores socioambientais que condicionam e determinam a saúde humana deve direcionar a elaboração de políticas públicas e ações estratégicas que fortaleçam a promoção, a prevenção e a assistência em saúde" \cite{GuiaVigSaúde22}.

\indent Recentemente, 2023, o governo de \acrlong{SC} propôs o Plano de Contingência para Enfrentamento de Arboviroses (Dengue, Zika e Chikungunya). Esse plano traz estratégias para organizar a resposta em situações de emergência, que variam conforme o nível de alerta acionado, incluindo os próprios indicadores de acionamento para o estadiamento dos níveis \cite{contingenciaSCdengue}.

\indent Dessa maneira, o Plano de Contingência no Estado de \citeonline{contingenciaSCdengue} serve como documento norteador na área da saúde, tanto para profissionais quanto para gestores. Nele também há a descrição do Comitê Intersetorial, que deve ser implementado em todos os municípios infestados por \latim{Aedes} sp., e a possibilidade de ativação do \acrfull{COE} para promover a resposta coordenada.

\indent Levando em consideração a saúde única, devemos adotar um caráter preventivo frente a catástrofes, não apenas  intervindo de forma mitigadora para interromper processos crônicos estabelecidos de degradação ambiental, recuperação de ecossistemas e manejo de populações comprometidas \cite{Cubas2014Tratado}.
%\indent Em se tratando de populações comprometidas, é importante salientar que pessoas impactadas em desastres optam por não sair de suas residências para irem aos abrigos sem seus animais de estimação. Por isso, a Defesa Civil do Estado do \acrlong{PR} conta com uma rede de atendimento a animais em situação de emergência, tendo participação do \acrlong{CRMV} do \acrlong{PR} (\acrshort{CRMV}/\acrshort{PR}), \acrfull{CEGRADE} alinhada com a Comissão Nacional de Desastres do \acrfull{CFMV}, Secretarias Estaduais e Municipais, além de voluntários \cite{CRMVPR22CEGRADE}.

\subsection{Situação Epidemiológica Atualizada da Dengue em SC/2024}

%https://www.who.int/news-room/fact-sheets/detail/dengue-and-severe-dengue

% \begin{center}
% \textcolor{red}{CONSIDERAÇÕES AQUI!}\\
% \indent \textcolor{red}{JOÃO: Situação Epidemiológica atualizada da dengue (referenciar o ano de que está se falando)\\ATUALIZAR\\}
% \end{center}

\indent As informações atualizadas e expostas estão em acordo com a \citeonline{Informe30DiveSE/24}.

\indent Embora haja flutuação na quantidade de focos identificados atualmente (57.103 focos em 2024), se comparados ao mesmo período do ano passado (61.589 focos em 2023) e ano retrasado (60.379 focos em 2022), há aumento (8,05\%) de municípios com registro (255 municípios em 2024), em relação ao mesmo período de 2023 (236 municípios) e 2022 (233 municípios; aumento de 9,44\%).

\indent Desse municípios que registram focos de \latim{Aedes} sp. em 2024, 175 foram considerados municípios infestados, representando aumento de 14,38\% em relação a 2023 (153 municípios infestados) e 30,6\% em relação a 2022 (134 municípios infestados).

\indent Além disso, há 558.313 notificações, o que significa um aumento considerável de 133,58\% ao comparar com 2023 (239.019) e 314,54\% ao comparar com 2022 (134.680).

\indent Assim, para obter melhor resposta frente a essa alteração de comportamento epidemiológico, houve alteração da metodologia em 2024, sendo considerados os casos prováveis dengue (incluindo inconclusivos e suspeitos aos confirmados), não apenas os casos confirmados.

\indent Ainda na metodologia anterior, o ano de 2023 registrou 117.677 casos confirmados de dengue até o início de novembro, também apresentando aumento (40,75\%) se comparado ao mesmo período de 2022 (83.606 casos confirmados).

\indent Como expresso no próprio informe epidemiológico (\citeyear{Informe30DiveSE/24}), houve registro de 352.374, o que representa aumento de 152,41\% em relação aos casos prováveis de 2023.

\indent O Estado de \acrlong{SC} identificou até o momento apenas a circulação de dois (2) sorotipos virais: \acrshort{DENV}1 e \acrshort{DENV}2. Desses, \acrshort{DENV}1 é o sorotipo predominante no território catarinense.

\indent Outro ponto importante são os óbitos que ocorreram no Estado. Houve aumento expressivo de 244,44\%, que confirmou 99 óbitos em 2023, tendo em 2024 a confirmação de 341 óbitos. Há ainda dois (2) óbitos sob investigação no atual momento.

\indent Também é importante perceber que houve adiantamento nos registros de óbitos confirmados. No ano de 2023 os primeiros óbitos ocorreram ao final de fevereiro e início de março, nona (9ª) semana epidemiológica, para ser preciso. Quando em 2024 os primeiros óbitos foram observados ainda em janeiro, na segunda (2ª) semana epidemiológica.

\indent Essa breve comparação entre anos posteriores pode salientar o quão dinâmico se apresenta o comportamento epidemiológico da dengue em \acrlong{SC}, tornando-se atual e imprescindível sua pesquisa e compreensão.



%As citações diretas com menos de três linhas “devem estar entre aspas e devem mostrar entre parênteses o ano e a página da obra consultada.” (AUTOR, ano, página). Já as citações com mais de três linhas devem ser recuadas da margem esquerda em 4 cm, tamanho da fonte 10, espaçamento simples e texto sem aspas (ABNT, 2002, p. 2).


%\subsubsection{Subtítulo Quaternário}

%\section{Subtítulo Secundário 3a}
 % Fundamentação Teórica
\chapter{Metodologia}

\section{Materiais}

Para organização, os dados serão agrupados em três divisões distintas: 

\begin{alineas}
    \item \acrfull{DEE}: Informações referentes a questões sanitárias (tanto ao vetor, quanto ao hospedeiro) e provenientes de banco de dados oficiais.  Os casos de focos de \latim{Aedes} sp. foram obtidos diretamente da \acrshort{Dive}. Em relação aos casos de dengue, foram obtidos de plataformas \ingles{on-line}: TabNet-\acrshort{Sinan}-\acrshort{DataSUS} e TabNet-\acrshort{Sinan}-\acrshort{Dive}; 
    
    \item \acrfull{DEC}: Informações referentes a variáveis meteorológicas/climatológicas (temperatura - mínima, máxima e média - e precipitação) e provenientes de banco de dados oficiais (reanálise e produtos de reanálise):  \ingles{\acrshort{GFS}}, \ingles{\acrshort{MERGE}} e \ingles{\acrshort{SAMeT}};

    \item \acrfull{DGR}: Informações referentes a aspectos geográficos atualizados para o momento atual, provenientes do \acrshort{IBGE}. 
\end{alineas}

\indent O \acrshort{MERGE} são dados de precipitação acumulada a superfície (mm), sendo dados diários e tendo início da série histórica em junho de 2000. Os dados do \acrshort{SAMeT} são coletados a dois metros (2m) da superfície e agrupados em médias diárias, sendo as temperaturas dadas em celsius (C), e a série histórica tem início em janeiro de 2000. \textcolor{red}{FALTA CITAR GFS/CFS!}.\\
\indent ...\\
\indent Os dados referentes a focos de \latim{Aedes} sp. são contabilizados no dia do próprio registro, ocorrendo apenas de segunda a sexta, e têm início no ano de 2012. A série histórica dos casos de dengue começa em 2014, sendo previamente agrupado e disponilizado em semanas epidemiológicas.\\


\section{Métodos}

\indent Para isso, pretende-se dividir o estudo em etapas, relativas ao percurso de execução do próprio projeto, sendo: pré-processamento, análise estatística descritiva, modelagem (treinamento e predição), espacialização dos dados preditos e Síntese do \acrfull{PTT}.\\


\subsection{Pré-processamento}

Em relação aos \acrshort{DEE}, as planilhas 

import pandas as pd
import numpy as np
from datetime import datetime, timedelta
import geopandas as gpd


\subsection{Análise Estatística Descritiva}

\indent Enquanto a etapa de Regionalização está em andamento, optou-se pelo desenvolvimento do \acrfull{ICOb}, que é resultado do acoplamento de \acrshort{DEC} [temperatura e precipitação - MERRA2 (0,5º / 0,25º) e SAMeT(0,05º) / MERGE (0,1º)]  interagindo com o \acrfull{LACOb} por \citeonline{Matiola2020Dissertação}.\\
\indent O índice normalizado foi calculado à partir do limiar (\acrshort{LACOb}) encontrado por \citeonline{Matiola2020Dissertação}, estabelecido como 15C através da correlação de Spearman (0,74), considerando 30 dias retroativos. O método aplicado nesse trabalho subtrai o \acrshort{LACOb} da matriz de temperaturas máximas diárias (TMAX) obtidas em produtos distintos: \acrshort{GFS} e \acrshort{SAMeT}. Essa matriz resultante da subtração foi somada à própria matriz em seus valores numéricos absolutos, onde os valores abaixo do limiar foram zerados. Essa nova matriz com valores zerados foi dividida por dois (2) para definição do \acrfull{ICOb}. As equações (\ref{eq:resultante}) e (\ref{eq:icob}) ilustram o cálculo do índice.
\begin{equation}
\hspace{4,5cm} [Resultante] = TMAX - \acrshort{LACOb}
    \label{eq:resultante}
\end{equation}
\begin{equation}
\hspace{4.5cm} \acrshort{ICOb} = \frac{(abs[Resultante])+[Resultante]}{2}
    \label{eq:icob}
\end{equation}
%ESCREVER COMO SE CHEGOU NO 'idenguen' >> 'define idenguen=10*at1*idengue/'_nvmax\\


\subsection{Modelagem}

\indent Na terceira etapa do estudo, serão definidos os  \acrfull{LACRe} entre os elementos climáticos e os resultados observados em relação aos focos de \latim{Aedes aegypti} distribuídos no Estado catarinense. Como resultado preliminar, foi utilizado o \acrshort{LACOb} desenvolvido por \citeonline{Matiola2020Dissertação}.


\subsection{Espacialização dos Dados Preditos}

\indent Para esse estudo, foi utilizado o recorte espacial durante a execução do prório \ingles{script}, sendo: longitude entre 54º5' e 57º5', ambas sul; e latitude entre 29º5' e 25º5', ambas oeste. Com esse recorte, pode-se evidenciar a totalidade do Estado de \acrlong{SC} e um pouco além de seus limites.\\
\indent Obteve-se os \ingles{shapefiles} do \acrshort{IBGE} (2022) para os limites territoriais (federal, estadual e municipais) do Estado de \acrlong{SC}.\\

\indent Durante a quarta etapa, retorna-se ao acoplamento e espacialização, que resultará na determinação dos \acrfull{ICRe}. Esses \acrshort{ICRe} serão corrigidos pelo acoplamento dos \acrshort{DEC} à interação dos \acrshort{LACRe} e ajustados na distribuição do Estado de \acrlong{SC}. Após isso, os novos limiares serão aplicados ao modelo preditivo proposto \acrshort{SAMeT}-\acrshort{MERGE}-\acrshort{GFS}.


\subsection{Refinamento com Dados Socioeconômicos e Epidemiológicos}

\indent Após a aplicação dos \acrshort{ICRe} ao modelo, pretende-se refinar o próprio modelo com a inclusão de dados Socioeconômicos do \acrshort{IBGE} e dados Epidemiológicos da \acrshort{Dive}. Desta maneira, a dinâmica epidemiológica poderá ser entendida à partir da ótica de diversas variáveis.


\subsection{Síntese do \acrfull{PTT}} 

\indent O \acrshort{PTT} resultante será: i) o sistema computacional e ii) a visualização cartográfica desses resultados, como preditivo da dinâmica regionalizada de focos de \latim{Aedes} sp. no Estado de \acrlong{SC}. O limiar temporal estipulado para previsão da incidência dos focos será de 60 dias. Os primeiros 30 dias serão determinados a partir do sistema \acrshort{SAMeT}-\acrshort{MERGE}. Os demais dias serão determinados através do modelo de previsão \acrshort{GFS}.


%\section{Conjutos de Dados Utilizados}
%sobre as arboviroses emergentes e re-mergentes (\textbf{Dengue}, Febre Amarela, Zika, Chikungunya)
%Tornar cada parte do método em objetivo específico
%ATUALIZAR, PENSANDO NO PROCESSO FUTURO\\


%A princípio, os dados de regionalização espacial encontram-se em processo de definição.




\subsection{Pré-processamento dos Dados}
\indent Para desenvolvimento do trabalho, os dados foram ajustados em mesma escala temporal, deixando-os em semanas epidemiológicas. Dessa forma, fez-se o somatório dos registros de focos de \latim{Aedes} sp. por semana epidemiológica. Mesmo tratamento foi realizado com a precipitação, retornando o acumulado de precipitação por semana epidemiológica. As temperaturas foram agrupadas de forma diferente, porém também ajustando a dados semanais. Ao final, teríamos a média das temperaturas (mínima, média e máxima) por semana epidemiológica.\\



 % Metodologia + Métodos Aplicados
\chapter{Apresentação dos Resultados}

TMIN ESTADUAL
Mínima: -7.98 C
Média: 14.75 C
Desvio Padrão Máximo: 4.88 C
Máxima: 30.26 C
TMED ESTADUAL
Mínima: -4.39 C
Média: 18.66 C
Desvio Padrão Máximo: 4.78 C
Máxima: 36.14 C
TMAX ESTADUAL
Mínima: -1.21 C
Média: 24.31 C
Desvio Padrão Máximo: 5.41 C
Máxima: 40.72 C
PREC ESTADUAL
Mínima: 0.0 mm
Média: 4.3 mm
Desvio Padrão Máximo: 12.68 mm
Máxima: 261.25 mm

TMIN ESTADUAL SEMANAL
Mínima: -2.77 C
Média: 14.75 C
Desvio Padrão Máximo: 4.11 C
Máxima: 26.15 C
TMED ESTADUAL SEMANAL
Mínima: 1.68 C
Média: 18.66 C
Desvio Padrão Máximo: 4.19 C
Máxima: 33.67 C
TMAX ESTADUAL SEMANAL
Mínima: 7.94 C
Média: 24.31 C
Desvio Padrão Máximo: 4.52 C
Máxima: 37.91 C
PREC ESTADUAL SEMANAL
Mínima: 0.0 mm
Média: 30.07 mm
Desvio Padrão Máximo: 43.05 mm
Máxima: 478.25 mm


\indent Os dados de casos de dengue, quando comparados entre as fontes (\acrshort{Dive} e \acrshort{DataSUS}), há diferença de atualização. Sendo os dados da própria \acrshort{Dive} mais atualizados, sendo eleitos para treinamento do modelo.

\indent A rede neural apresentou baixo ajuste frente ao comportamento dos fenômenos.

Texto texto texto texto texto texto texto texto texto texto texto texto texto texto texto texto texto texto texto texto texto texto texto texto texto texto texto texto texto texto texto texto texto texto texto texto texto texto texto.

\section{Análise e discussão dos resultados}

Texto texto texto texto texto texto texto texto texto texto texto texto texto texto texto texto texto texto texto texto texto texto texto texto texto texto texto texto texto texto texto texto texto texto texto texto texto texto texto.

\indent \textcolor{red}{explicar como evitar "double dipping".}


%https://cdn.embedly.com/widgets/media.html?src=https%3A%2F%2Fwww.youtube.com%2Fembed%2F3tiuRHuzST4&display_name=YouTube&url=https%3A%2F%2Fwww.youtube.com%2Fwatch%3Fv%3D3tiuRHuzST4&image=http%3A%2F%2Fi.ytimg.com%2Fvi%2F3tiuRHuzST4%2Fhqdefault.jpg&key=40cb30655a7f4a46adaaf18efb05db21&type=text%2Fhtml&schema=youtube
 % Resultados + Análises + Discussões
\chapter{Considerações Finais}

\indent Neste projeto foram analisados dados entomológicos desde 2012 e epidemiológicos desde 2014, ambos com registros em semanas epidemiológicas. Os dados climatológicos foram adquiridos, em frequência de registros diária, desde 2000; porém apenas se pode analisar relação entre todos os dados ao padronizar o tempo em semanas epidemiológicas.

\indent Logo, utilizar sazonalidade em semanas epidemiológicas para dados climatológicos é uma excelente forma prática para fazer relações com o setor da saúde, principalmente ao se pesquisar doenças de notificação e transmitidas por vetores. Assim, o estudo possibilita auxiliar os gestores públicos, principalmente da área da saúde, visando aprimorar diretrizes de combate, controle e prevenção da dengue, como do vetor biológico \latim{Aedes} sp. Também possibilita base e comparação para estudos futuros, padronizando dados ambientais em semanas epidemiológicas.

\indent Para análise de relação entre os comportamentos entomo-epidemiológicos e climatológicos, o uso de produtos observacionais e de reanálise, \ingles{\acrshort{SAMeT}} e \ingles{\acrshort{MERGE}}, mostraram-se eficientes e oportunos em pesquisas que envolvam análises climatológicas e epidemiológicas, principalmente por doenças vetorizadas por mosquitos do gênero \latim{Aedes}.

\indent Como a utilização destes produtos, é possível obter grande volume de dados com altas frequências (horário/diário), disponíveis para a malha territorial do Estado de \acrlong{SC}. Desse modo, estudos entomo-epidemiológicos envolvendo questões climáticas podem regionalizar comportamentos que não seriam observados em dados mensais ou estações pontuais de dados meteorológicos. 

\indent Os resultados mostram a regionalização na distribuição de focos de \label{Aedes} sp. e casos de dengue no Estado de \acrlong{SC}, seguindo o padrão de regionalização climatológica apresentado por \citeonline{Guerra2023Regionalizacao}. Região Oeste, Vale do Itajaí e a faixa litorânea apresentam condições climatológicas favoráveis para o desenvolvimento dos mosquitos do gênero \latim{Aedes}. Logo, são regiões com notório risco para epidemias, não apenas, mas principalmente por dengue. Deve-se atentar para o aumento nos casos de Zika e Chikungunya nos últimos anos, também vetorizados por mosquitos do gênero \latim{Aedes} \cite{Valle2015Dengue}.

\indent As regiões serranas e de planaltos servem como limitadores ambientais na distribuição dos vetores da dengue no Estado catarinense, uma vez que não apresentam condições ideais para o desenvolvimento do mosquitos, como citado por \citeonline{AedesTemp}. Por essa razão, essas regiões apresentaram baixa compilação de modelos, uma vez que há baixa ocorrência de vetor (focos de \latim{Aedes} sp.) ou vírus circulante (casos de dengue).

\indent As distribuições de sazonalidade com maior ocorrência de casos de dengue ocorrem entre os meses de março e abril, como evidenciado por \citeonline{Valle2015Dengue} e por \citeonline{Drumond2020Dinamica} para outras regiões brasileiras. Esse padrão caracteriza o comportamento biológico do vetor, aumentando desenvolvimento durante o período de verão austral, e o comportamento epidemiológico da doença, corroborando a nomenclatura de doença tropical \cite{Valle2015Dengue}.

\indent Há comportamento epidemiológico comum entre os municípios analisados (Florianópolis, Itajaí, Joinville e Chapecó), ao correlacionar casos de dengue e limiares de temperaturas, evidenciando correlações médias a altas negativas ao retroceder no tempo. Esse fato corrobora a distribuição sazonal nesses municípios em \acrlong{SC}, quando há aumento no número de casos de dengue ao diminuir as temperaturas no outono austral. 

\indent Deve-se tomar nota para o município de Chapecó, assim como a região oeste catarinense, onde foi evidenciado antecipação do pico de casos registrados, estando em acordo ao observado por \citeonline{Guerra2023Regionalizacao} com a antecipação das temperaturas elevadas durante a primavera. Esse comportamento antecipado em Chapecó também foi observado por \citeonline{Matiola2019ANALISE} em análise a nível municipal.

\indent Durante esse estudo, a cobertura vacinal para o vírus da dengue no Estado de \acrlong{SC}, assim como a soltura de mosquitos \latim{Aedes} sp. laboratorialmente infectados com a bactéria do gênero \latim{Wolbachia} no município de Joinville, houve pouca influência para o processo de aprendizado de máquina durante a modelagem computacional, uma vez que essas ações de controle do vírus da dengue são recentes no território catarinense. Espera-se acompanhamento dos estudos para monitorar a dinâmica entomo-epidemiológica em \acrlong{SC} ao longo dos anos.

\indent Ao passo que as temperaturas globais atingem e ultrapassam os recordes máximos registrados, também é esperado que o comportamento entomo-epidemiológico tenha adaptação frente a esse novo cenário de mudança climática global. Por esse ,otivo, deve-se analisar e monitorar tais alterações de comportamento, tanto climáticos quanto entomo-epidemiológicos, no Estado de \acrlong{SC} nos próximos anos.

\indent Por apresentar ascendência na série de anomalias estacionárias entomológicas e epidemiológicas, não ocorrendo o mesmo com as séries de anomalias estacionárias climatológicas, estudos aprofundados devem ser realizados para confirmação na mudança do comportamento entomo-epidemiológico. Especula-se influência do confinamento (\latim{lockdown}) durante a pandemia de coronavírus da síndrome respiratória aguda grave 2 (\acrfull{SARS-CoV-2}).

\indent Por haver comportamentos confluentes e regionais entre dados entomo-epidemiológicos  e climatológicos, é possível utilizar o tempo meteorológico para previsão de aumento de focos de \latim{Aedes} sp. e casos de dengue, como já ocorre com outras ferramentas preditivas e pesquisas relacionadas \cite{Infodengue_2018, ForecastingDengueBrazil2019, Relatorio_Infodengue_2023}.

\indent Nesse estudo, a utilização de aprendizado de máquina por modelos \ingles{Random Forest} apresentou melhor resposta de predição, quando comparado à modelagem por \acrfull{RNAM}. Possivelmente um maior volume de dados e uma melhor manipulação do processo trariam melhores resultados no aprendizado de máquina por \acrshort{RNAM}. Em ambas as modelagens, \ingles{Random Forest} para focos de \latim{Aedes} sp. e casos de dengue, as variáveis mais importantes foram as próprias variáveis dependentes retroagidas no tempo.

\indent Ao analisar os histograma dos erros, é possível perceber adensamento próximo ao zero (0), indicando melhores resultados de \acrfull{r2}. Ainda assim, municípios como Florianópolis e Joinville apresentaram médias desviadas à direita em relação à mediana, sugerindo modelos que subestimam os valores de previsão durante o processo de aprendizado de máquina.

\indent Finalmente, o \acrlong{PTT} demonstra capacidade preditiva de 14 dias (2 semanas epidemiológicas) para os casos de dengue no Estado de \acrlong{SC}, através de modelos \ingles{Random Forest} executados de forma automatizada em frequência semanal. Esses valores preditos podem ser visualizados através de cartografias temáticas, disponíveis em semanas epidemiológicas. A primeira (1ª) semana de previsão ocorre utilizando dados climatológicos da associação \ingles{\acrshort{SAMeT}-\acrshort{MERGE}}. Para previsão da segunda semana (2ª), utilizam-se dados de previsão do modelo \ingles{\acrshort{GFS}}.



%http://www.linse.ufsc.br/~fernando/dicas/figuras.html
 % Considerações Finais
\chapter{Futuros trabalhos}

\indent Assim como muitos projetos, o atual se apresenta como um recorte limitado, tendo potencial para expensão e refinamento. Espera-se que, frente a essa nova dinâmica epidemiológica da dengue no Estado de \acrlong{SC}, projetos continuem a linha de pesquisa, assim como esse também o foi.

\section{Sugestões para trabalhos futuros}

\indent Para aprimorar o trabalho atual, foram listadas sugestões para implementar em trabalhos futuros, segue abaixo:

\begin{itemize}
    \item Utilização de pontos urbanos centrais das cidades;
    \item Utilização de série histórica de temperaturas (mínima, média e máxima) do \acrfull{Bdmep/Inmet} para cobrir a máscara do \ingles{\acrshort{SAMeT}} nos municípios de Balneário Camboriú, Bombinhas e Porto Belo;
    \item Inclusão de dados sociais para refinamento do modelo;
    \item Inclusão de cálculos epidemiológicos a partir de dados sociais;
    \item Inclusão de casos suspeitos e inconclusivos de dengue para adequação com a nova metodologia utilizada pela \acrshort{Dive}/\acrshort{SC};
    \item Utilização de outras bases de dados secundários para pesquisa em saúde;
    \item Utilização de dados do \ingles{\acrfull{CFS}} do \ingles{\acrshort{NCEP}/\acrshort{NOAA}};
    \item Utilização dos primeiros registros entomo-epidemiológicos para avaliar a distribuição temporal no território catarinense;
    \item Derivação dos dados, tanto climatológicos quanto entomo-epidemiológicos, para síntese de índices e taxas;
    \item Adaptação da modelagem para modelo dinâmico a partir de cálculos epidemiológicos e história natural da doença;
    \item Adaptação dos dados conforme Níveis de Emergência do Plano de Contingência de Dengue em \citeonline{contingenciaSCdengue};
    \item Ajustes de hiperparâmetros durante a modelagem;
    \item Ajustes de modelagem a partir de outras metodologias;
    \item Ajustes de validação de modelagem a partir de outras métricas;
    \item Análise pormenorizada das séries de anomalias estacionárias entomo-epidemiológicas e climatológicas;
    \item Análise de modelagem a partir de novos cenários presentes (vacinação e presença de \latim{Aedes-Wolbachia}).
\end{itemize}
 % Futuros Trabalhos + Sugestões Futuras



% ----------------------------------------------------------
% ELEMENTOS PÓS-TEXTUAIS
% ----------------------------------------------------------
\postextual
% ----------------------------------------------------------

% ----------------------------------------------------------
% Referências bibliográficas
% ----------------------------------------------------------
\bibliography{bibliografia}

% ----------------------------------------------------------
% Apêndices
% ----------------------------------------------------------
 % \begin{apendicesenv}         
 %     \partapendices
 %     \chapter{Apêndices}

\chapter{Título}

\begin{longtable}[htbp]{llcrr}
\label{tab:primeiros_focos}

\caption{Primeiros registros de focos de \textit{Aedes} sp. em municípios catarinenses.} \\
\hline
\rowcolor{darkgray} \textcolor{white}{Semana} & \textcolor{white}{Município} & \textcolor{white}{Focos} & \textcolor{white}{Latitude} & \textcolor{white}{Longitude} \\
\hline
\endfirsthead

\caption{(Continuação) Primeiros registros de focos de \textit{Aedes} sp. em municípios catarinenses.} \\
\rowcolor{darkgray} \textcolor{white}{Semana} & \textcolor{white}{Município} & \textcolor{white}{Focos} & \textcolor{white}{Latitude} & \textcolor{white}{Longitude} \\
\hline
\endhead

\hline
\textit{Continua na próxima página}
\hline 
\endfoot

\hline
\textit{Finalização da tabela} \\
\hline
\endlastfoot

2012-01-01 & BALNEÁRIO CAMBORIÚ & 2 & -27.004737 & -48.621741 \\
2012-01-01 & BLUMENAU & 1 & -26.885767 & -49.097309 \\
2012-01-01 & BOMBINHAS & 2 & -27.173256 & -48.517487 \\
2012-01-01 & CHAPECÓ & 1 & -27.125144 & -52.650339 \\
2012-01-01 & FLORIANÓPOLIS & 1 & -27.577834 & -48.508198 \\
2012-01-01 & SÃO MIGUEL DO OESTE & 3 & -26.727237 & -53.512121 \\
2012-01-01 & TIJUCAS & 2 & -27.247563 & -48.700886 \\
2012-01-01 & XANXERÊ & 1 & -26.872279 & -52.409754 \\
2012-01-08 & JOINVILLE & 1 & -26.244282 & -48.951405 \\
2012-01-15 & ARAQUARI & 2 & -26.461708 & -48.757777 \\
2012-01-15 & BRUSQUE & 1 & -27.125000 & -48.909743 \\
2012-01-15 & GUARAMIRIM & 1 & -26.476551 & -48.944882 \\
2012-01-22 & DIONÍSIO CERQUEIRA & 1 & -26.329843 & -53.533105 \\
2012-01-29 & BIGUAÇU & 1 & -27.433530 & -48.693947 \\
2012-01-29 & JARAGUÁ DO SUL & 1 & -26.481908 & -49.159974 \\
2012-01-29 & SAUDADES & 1 & -26.897207 & -53.040014 \\
2012-01-29 & SÃO JOSÉ & 1 & -27.578471 & -48.656256 \\
2012-02-05 & SÃO JOÃO DO OESTE & 1 & -27.091607 & -53.591664 \\
2012-02-12 & JAGUARUNA & 2 & -28.657896 & -49.044748 \\
2012-02-12 & SÃO JOSÉ DO CEDRO & 3 & -26.480851 & -53.533736 \\
2012-02-19 & ITAPOÁ & 1 & -26.082477 & -48.652354 \\
2012-02-26 & BARRA VELHA & 1 & -26.662441 & -48.727386 \\
2012-02-26 & CRICIÚMA & 1 & -28.715695 & -49.379716 \\
2012-02-26 & MASSARANDUBA & 1 & -26.626911 & -48.988192 \\
2012-02-26 & PALMITOS & 1 & -27.092622 & -53.179829 \\
2012-03-04 & ARARANGUÁ & 1 & -28.942293 & -49.471174 \\
2012-03-04 & GAROPABA & 1 & -28.046734 & -48.658936 \\
2012-03-04 & PALHOÇA & 1 & -27.771821 & -48.661708 \\
2012-03-04 & PAULO LOPES & 1 & -27.964855 & -48.760163 \\
2012-03-11 & BENEDITO NOVO & 1 & -26.800940 & -49.435396 \\
2012-03-11 & CAPIVARI DE BAIXO & 2 & -28.455564 & -48.943385 \\
2012-03-11 & CONCÓRDIA & 5 & -27.239127 & -52.007382 \\
2012-03-11 & PORTO UNIÃO & 1 & -26.383628 & -51.007785 \\
2012-03-11 & RIO DO SUL & 1 & -27.196051 & -49.629953 \\
2012-03-11 & SIDERÓPOLIS & 1 & -28.583272 & -49.526696 \\
2012-03-11 & SOMBRIO & 1 & -29.070191 & -49.655927 \\
2012-03-18 & LAURENTINO & 1 & -27.207982 & -49.733614 \\
2012-03-25 & APIÚNA & 1 & -27.124502 & -49.364245 \\
2012-03-25 & CANOINHAS & 1 & -26.249950 & -50.533371 \\
2012-03-25 & SÃO FRANCISCO DO SUL & 2 & -26.261597 & -48.640319 \\
2012-03-25 & TUBARÃO & 1 & -28.481581 & -49.037249 \\
2012-04-01 & NAVEGANTES & 1 & -26.828743 & -48.727396 \\
2012-04-01 & PINHALZINHO & 14 & -26.829373 & -52.976632 \\
2012-04-15 & PIRATUBA & 1 & -27.461219 & -51.772820 \\
2012-04-22 & GARUVA & 1 & -26.057112 & -48.867328 \\
2012-04-22 & POMERODE & 1 & -26.728186 & -49.173304 \\
2012-05-06 & XAXIM & 1 & -26.970309 & -52.529105 \\
2012-05-20 & PORTO BELO & 1 & -27.174796 & -48.616435 \\
2012-05-20 & SÃO BENTO DO SUL & 1 & -26.294902 & -49.349982 \\
2012-05-27 & GASPAR & 1 & -26.931343 & -48.966033 \\
2012-06-10 & COCAL DO SUL & 1 & -28.598697 & -49.333930 \\
2012-06-10 & ITAPIRANGA & 1 & -27.112222 & -53.711674 \\
2012-07-15 & SÃO LOURENÇO DO OESTE & 1 & -26.465488 & -52.859773 \\
2012-08-12 & CAMPO ALEGRE & 1 & -26.120556 & -49.217539 \\
2012-09-02 & CAIBI & 1 & -27.025176 & -53.263058 \\
2012-09-02 & MARAVILHA & 1 & -26.759974 & -53.198113 \\
2012-09-16 & BRAÇO DO NORTE & 1 & -28.240945 & -49.142365 \\
2012-10-07 & INDAIAL & 1 & -26.994979 & -49.224178 \\
2012-10-07 & ITAPEMA & 1 & -27.108583 & -48.634427 \\
2012-10-28 & LAGES & 1 & -28.020467 & -50.340573 \\
2012-11-18 & IMBITUBA & 1 & -28.194274 & -48.702663 \\
2012-12-23 & SÃO DOMINGOS & 1 & -26.539387 & -52.554602 \\
2012-12-30 & GUATAMBÚ & 10 & -27.116474 & -52.783390 \\
2013-01-20 & BALNEÁRIO PIÇARRAS & 13 & -26.759492 & -48.733290 \\
2013-01-20 & CORONEL FREITAS & 3 & -26.883161 & -52.725052 \\
2013-01-20 & LAGUNA & 1 & -28.486421 & -48.825930 \\
2013-01-27 & CORDILHEIRA ALTA & 1 & -26.975707 & -52.641660 \\
2013-01-27 & ITAJAÍ & 1 & -26.969013 & -48.753417 \\
2013-02-10 & RIQUEZA & 1 & -26.978455 & -53.344352 \\
2013-02-24 & PRINCESA & 1 & -26.432945 & -53.611776 \\
2013-02-24 & SANTA ROSA DO SUL & 2 & -29.121483 & -49.743751 \\
2013-03-03 & BELMONTE & 1 & -26.857082 & -53.618060 \\
2013-03-03 & CAÇADOR & 1 & -26.763016 & -51.084665 \\
2013-03-17 & ITUPORANGA & 1 & -27.453450 & -49.538870 \\
2013-03-17 & SANTO AMARO DA IMPERATRIZ & 3 & -27.741766 & -48.796550 \\
2013-04-14 & IMARUÍ & 1 & -28.231479 & -48.834413 \\
2013-04-14 & MONDAÍ & 1 & -27.099293 & -53.446497 \\
2013-04-14 & NOVA ERECHIM & 1 & -26.908169 & -52.906693 \\
2013-05-05 & ÁGUAS DE CHAPECÓ & 1 & -27.054386 & -52.956735 \\
2013-05-19 & ANTÔNIO CARLOS & 1 & -27.497474 & -48.830982 \\
2013-05-26 & QUILOMBO & 1 & -26.729236 & -52.717982 \\
2013-06-02 & GUARACIABA & 1 & -26.580106 & -53.571076 \\
2013-06-02 & IRANI & 1 & -27.025235 & -51.917948 \\
2013-06-02 & VIDEIRA & 1 & -27.006590 & -51.126791 \\
2013-06-16 & SÃO JOÃO BATISTA & 1 & -27.328928 & -48.859016 \\
2013-06-30 & FORMOSA DO SUL & 1 & -26.634189 & -52.793789 \\
2013-06-30 & RIO NEGRINHO & 1 & -26.442714 & -49.596156 \\
2013-06-30 & SÃO LUDGERO & 1 & -28.353135 & -49.168162 \\
2013-07-28 & DESCANSO & 1 & -26.858414 & -53.479301 \\
2013-07-28 & JOAÇABA & 1 & -27.151796 & -51.592935 \\
2013-09-29 & CAMBORIÚ & 1 & -27.070939 & -48.708989 \\
2013-09-29 & CORUPÁ & 1 & -26.439934 & -49.326711 \\
2013-09-29 & MAFRA & 1 & -26.200812 & -49.889559 \\
2013-10-06 & IBIRAMA & 1 & -27.018150 & -49.528544 \\
2013-10-06 & ITÁ & 1 & -27.244015 & -52.332158 \\
2013-10-06 & PARAÍSO & 1 & -26.662800 & -53.678304 \\
2013-11-03 & CAPÃO ALTO & 1 & -28.062182 & -50.606912 \\
2013-11-24 & ORLEANS & 1 & -28.279938 & -49.371860 \\
2013-12-29 & CATANDUVAS & 3 & -27.044998 & -51.696411 \\
2013-12-29 & GUARUJÁ DO SUL & 1 & -26.398336 & -53.484021 \\
2013-12-29 & RIO DO OESTE & 1 & -27.158977 & -49.841673 \\
2014-01-05 & IPUAÇU & 1 & -26.678739 & -52.476014 \\
2014-01-12 & GRAVATAL & 1 & -28.354020 & -49.019524 \\
2014-01-19 & IÇARA & 5 & -28.751615 & -49.277346 \\
2014-02-09 & PENHA & 1 & -26.807760 & -48.650765 \\
2014-02-16 & GOVERNADOR CELSO RAMOS & 1 & -27.376617 & -48.577230 \\
2014-02-16 & PASSO DE TORRES & 1 & -29.267248 & -49.721932 \\
2014-02-16 & TIMBÓ & 1 & -26.808192 & -49.268848 \\
2014-03-02 & SALTINHO & 1 & -26.590522 & -53.022279 \\
2014-03-09 & CAMPO ERÊ & 1 & -26.450419 & -53.130290 \\
2014-03-09 & LUZERNA & 2 & -27.090413 & -51.505633 \\
2014-03-09 & SERRA ALTA & 1 & -26.692621 & -53.025238 \\
2014-03-23 & TREZE TÍLIAS & 1 & -26.964791 & -51.449354 \\
2014-04-06 & CUNHATAÍ & 2 & -26.975087 & -53.100870 \\
2014-04-06 & IRATI & 2 & -26.627592 & -52.890099 \\
2014-04-06 & SUL BRASIL & 2 & -26.695342 & -52.946800 \\
2014-04-13 & NOVA TRENTO & 1 & -27.313268 & -49.041367 \\
2014-04-13 & SÃO CARLOS & 1 & -27.030109 & -53.031412 \\
2014-04-20 & ENTRE RIOS & 1 & -26.739710 & -52.578368 \\
2014-04-20 & SCHROEDER & 1 & -26.367952 & -49.060456 \\
2014-04-27 & VARGEM BONITA & 1 & -26.944996 & -51.756631 \\
2014-05-11 & CUNHA PORÃ & 2 & -26.879228 & -53.190665 \\
2014-05-11 & TRÊS BARRAS & 1 & -26.169921 & -50.255230 \\
2014-05-18 & ÁGUAS MORNAS & 1 & -27.733481 & -48.936027 \\
2014-05-25 & UNIÃO DO OESTE & 3 & -26.786666 & -52.856010 \\
2014-08-10 & CAPINZAL & 2 & -27.411939 & -51.630815 \\
2014-08-10 & FORQUILHINHA & 3 & -28.781693 & -49.493283 \\
2014-08-10 & SÃO JOÃO DO ITAPERIÚ & 1 & -26.591052 & -48.797733 \\
2014-08-10 & TREVISO & 5 & -28.499162 & -49.489841 \\
2014-09-07 & ARVOREDO & 1 & -27.070854 & -52.441987 \\
2014-09-07 & CAXAMBU DO SUL & 1 & -27.146711 & -52.920211 \\
2014-09-07 & IBICARÉ & 1 & -27.092470 & -51.370040 \\
2014-09-07 & OURO & 5 & -27.283437 & -51.681886 \\
2014-10-05 & BOM JESUS & 1 & -26.739042 & -52.389991 \\
2014-10-12 & PLANALTO ALEGRE & 1 & -27.051964 & -52.868983 \\
2014-11-09 & GUABIRUBA & 4 & -27.103338 & -49.020750 \\
2014-12-14 & ARABUTÃ & 1 & -27.140868 & -52.181545 \\
2014-12-14 & BALNEÁRIO BARRA DO SUL & 8 & -26.446419 & -48.652569 \\
2014-12-14 & BALNEÁRIO RINCÃO & 3 & -28.826740 & -49.254237 \\
2014-12-14 & BOTUVERÁ & 3 & -27.218576 & -49.123526 \\
2014-12-14 & ILHOTA & 7 & -26.859041 & -48.857351 \\
2014-12-14 & JARDINÓPOLIS & 1 & -26.717195 & -52.871306 \\
2014-12-14 & LUIZ ALVES & 1 & -26.727352 & -48.891082 \\
2014-12-14 & PERITIBA & 4 & -27.351361 & -51.875059 \\
2014-12-14 & PESCARIA BRAVA & 2 & -28.401764 & -48.883642 \\
2014-12-14 & SEARA & 3 & -27.147833 & -52.340694 \\
2014-12-21 & HERVAL D'OESTE & 10 & -27.195376 & -51.410665 \\
2014-12-21 & IBIAM & 1 & -27.208374 & -51.215463 \\
2014-12-21 & TANGARÁ & 1 & -27.147028 & -51.147069 \\
2014-12-28 & SANTA TEREZINHA DO PROGRESSO & 2 & -26.592749 & -53.169925 \\
2015-01-04 & ANCHIETA & 1 & -26.532948 & -53.333109 \\
2015-01-04 & NOVA ITABERABA & 3 & -26.961142 & -52.833166 \\
2015-01-04 & SÃO BERNARDINO & 1 & -26.487695 & -52.987611 \\
2015-01-11 & ARMAZÉM & 1 & -28.237881 & -49.014678 \\
2015-01-18 & CAMPOS NOVOS & 1 & -27.413002 & -51.244612 \\
2015-01-25 & CORONEL MARTINS & 1 & -26.534646 & -52.671875 \\
2015-02-01 & ABELARDO LUZ & 1 & -26.562427 & -52.255719 \\
2015-02-08 & JUPIÁ & 1 & -26.394106 & -52.720920 \\
2015-02-15 & XAVANTINA & 1 & -27.021214 & -52.321890 \\
2015-03-01 & PALMA SOLA & 1 & -26.373544 & -53.314529 \\
2015-03-29 & LONTRAS & 1 & -27.189434 & -49.507309 \\
2015-04-12 & BOM JESUS DO OESTE & 2 & -26.684021 & -53.095288 \\
2015-04-12 & NOVO HORIZONTE & 2 & -26.494979 & -52.796361 \\
2015-04-19 & NOVA VENEZA & 1 & -28.684718 & -49.584912 \\
2015-04-26 & GALVÃO & 2 & -26.449409 & -52.659011 \\
2015-04-26 & SÃO MARTINHO & 1 & -28.124557 & -48.971904 \\
2015-05-03 & FLOR DO SERTÃO & 1 & -26.763781 & -53.337342 \\
2015-05-10 & SÃO PEDRO DE ALCÂNTARA & 1 & -27.591567 & -48.837131 \\
2015-05-31 & IRINEÓPOLIS & 1 & -26.357066 & -50.760691 \\
2015-12-13 & MODELO & 2 & -26.774777 & -53.051096 \\
2016-01-03 & IPUMIRIM & 1 & -27.040309 & -52.152891 \\
2016-01-03 & PASSOS MAIA & 1 & -26.706083 & -51.961063 \\
2016-01-17 & RIO DOS CEDROS & 1 & -26.617927 & -49.368385 \\
2016-01-24 & SANGÃO & 1 & -28.651315 & -49.125834 \\
2016-02-14 & IRACEMINHA & 1 & -26.845317 & -53.324456 \\
2016-02-14 & OURO VERDE & 1 & -26.711533 & -52.278758 \\
2016-02-21 & TUNÁPOLIS & 1 & -26.988117 & -53.649949 \\
2016-03-13 & BANDEIRANTE & 1 & -26.768596 & -53.648168 \\
2016-03-13 & CORREIA PINTO & 3 & -27.596830 & -50.378096 \\
2016-03-13 & SÃO MIGUEL DA BOA VISTA & 2 & -26.689643 & -53.239962 \\
2016-03-20 & ROMELÂNDIA & 1 & -26.648390 & -53.317124 \\
2016-03-20 & ÁGUA DOCE & 1 & -26.764469 & -51.612406 \\
2016-04-10 & CURITIBANOS & 1 & -27.292934 & -50.617548 \\
2016-05-08 & PRESIDENTE GETÚLIO & 1 & -27.062584 & -49.714430 \\
2016-06-05 & MORRO DA FUMAÇA & 1 & -28.637696 & -49.242862 \\
2016-10-02 & TURVO & 1 & -28.903082 & -49.703422 \\
2016-10-30 & ASCURRA & 1 & -26.973634 & -49.395966 \\
2016-11-06 & AGROLÂNDIA & 1 & -27.455116 & -49.824780 \\
2016-11-20 & MARACAJÁ & 1 & -28.863963 & -49.455547 \\
2016-11-27 & FRAIBURGO & 1 & -27.037815 & -50.872432 \\
2016-11-27 & SALETE & 1 & -26.969623 & -50.010685 \\
2016-12-18 & ÁGUAS FRIAS & 1 & -26.851639 & -52.852311 \\
2017-01-01 & MELEIRO & 1 & -28.841751 & -49.601794 \\
2017-03-05 & PAINEL & 1 & -27.963630 & -50.072135 \\
2017-03-12 & TIGRINHOS & 1 & -26.676067 & -53.156546 \\
2017-03-19 & IPORÃ DO OESTE & 5 & -27.000432 & -53.489414 \\
2017-04-16 & AGRONÔMICA & 1 & -27.327559 & -49.720747 \\
2017-04-30 & SANTIAGO DO SUL & 1 & -26.630874 & -52.697312 \\
2017-05-21 & LAJEADO GRANDE & 1 & -26.854941 & -52.552085 \\
2017-06-18 & SÃO JOÃO DO SUL & 1 & -29.205567 & -49.813468 \\
2017-07-09 & ALTO BELA VISTA & 1 & -27.421535 & -51.925948 \\
2017-07-30 & MONTE CASTELO & 1 & -26.660694 & -50.291540 \\
2017-07-30 & VARGEÃO & 1 & -26.778693 & -52.124176 \\
2017-12-31 & FAXINAL DOS GUEDES & 3 & -26.846669 & -52.244323 \\
2018-01-07 & BALNEÁRIO GAIVOTA & 1 & -29.153516 & -49.615368 \\
2018-01-21 & LINDÓIA DO SUL & 1 & -27.031507 & -52.052466 \\
2018-01-28 & BARRA BONITA & 1 & -26.668358 & -53.425055 \\
2018-03-25 & BALNEÁRIO ARROIO DO SILVA & 1 & -29.019465 & -49.475057 \\
2018-03-25 & GRÃO-PARÁ & 1 & -28.155035 & -49.314022 \\
2018-05-13 & PONTE SERRADA & 6 & -26.861769 & -51.928214 \\
2018-06-10 & JACINTO MACHADO & 1 & -28.994832 & -49.848148 \\
2018-07-01 & PAIAL & 1 & -27.213003 & -52.487869 \\
2018-11-11 & LACERDÓPOLIS & 1 & -27.250882 & -51.590311 \\
2019-01-06 & MAREMA & 3 & -26.812383 & -52.630070 \\
2019-01-13 & LAURO MÜLLER & 1 & -28.384585 & -49.451932 \\
2019-01-13 & PONTE ALTA DO NORTE & 2 & -27.180289 & -50.418567 \\
2019-01-13 & SANTA HELENA & 3 & -26.921956 & -53.619652 \\
2019-01-20 & SÃO CRISTÓVÃO DO SUL & 1 & -27.295206 & -50.349497 \\
2019-01-20 & WITMARSUM & 1 & -26.937345 & -49.841761 \\
2019-02-10 & POUSO REDONDO & 1 & -27.286166 & -49.977272 \\
2019-02-10 & SALTO VELOSO & 1 & -26.895935 & -51.428018 \\
2019-02-17 & IPIRA & 1 & -27.370819 & -51.798708 \\
2019-03-03 & TAIÓ & 1 & -27.078456 & -50.092435 \\
2019-03-24 & PAPANDUVA & 3 & -26.502424 & -50.172362 \\
2019-06-30 & BRAÇO DO TROMBUDO & 1 & -27.368599 & -49.903617 \\
2019-10-20 & PRAIA GRANDE & 1 & -29.181868 & -49.989502 \\
2019-11-24 & URUSSANGA & 1 & -28.492699 & -49.328277 \\
2020-01-26 & PETROLÂNDIA & 1 & -27.545741 & -49.682724 \\
2020-02-09 & ZORTÉA & 1 & -27.476988 & -51.535615 \\
2020-02-16 & RODEIO & 1 & -26.895097 & -49.351036 \\
2020-03-15 & AURORA & 1 & -27.329632 & -49.590347 \\
2020-04-12 & JOSÉ BOITEUX & 1 & -26.859487 & -49.646306 \\
2020-04-19 & CANELINHA & 1 & -27.234604 & -48.802310 \\
2020-04-26 & TROMBUDO CENTRAL & 1 & -27.312097 & -49.811416 \\
2020-05-03 & ANGELINA & 1 & -27.543006 & -49.067224 \\
2020-05-03 & MACIEIRA & 1 & -26.803514 & -51.353132 \\
2020-05-03 & SÃO BONIFÁCIO & 1 & -27.956554 & -48.939098 \\
2020-06-14 & MAJOR GERCINO & 1 & -27.427926 & -49.052632 \\
2020-08-16 & TREZE DE MAIO & 1 & -28.566224 & -49.154191 \\
2021-01-03 & JABORÁ & 1 & -27.141510 & -51.772667 \\
2021-01-10 & ATALANTA & 1 & -27.437304 & -49.742550 \\
2021-01-10 & ERVAL VELHO & 1 & -27.284351 & -51.421890 \\
2021-01-10 & PRESIDENTE NEREU & 1 & -27.268253 & -49.354952 \\
2021-01-17 & ERMO & 1 & -28.990337 & -49.652921 \\
2021-01-24 & ITAIÓPOLIS & 1 & -26.472148 & -49.891245 \\
2021-02-21 & RIO DAS ANTAS & 1 & -26.909076 & -51.063409 \\
2021-02-28 & RIO DO CAMPO & 1 & -26.892307 & -50.131394 \\
2021-03-14 & LEBON RÉGIS & 1 & -26.858429 & -50.702569 \\
2021-03-14 & MAJOR VIEIRA & 2 & -26.481213 & -50.361577 \\
2021-05-09 & CELSO RAMOS & 1 & -27.651969 & -51.308241 \\
2021-08-08 & ARROIO TRINTA & 1 & -26.921499 & -51.336678 \\
2021-09-12 & VIDAL RAMOS & 1 & -27.387882 & -49.347753 \\
2021-10-31 & MORRO GRANDE & 1 & -28.722065 & -49.740484 \\
2021-11-28 & MIRIM DOCE & 1 & -27.185593 & -50.175136 \\
2022-01-09 & IOMERÊ & 5 & -26.987730 & -51.275507 \\
2022-01-16 & DONA EMMA & 1 & -26.988701 & -49.778082 \\
2022-01-23 & PINHEIRO PRETO & 1 & -27.049431 & -51.236054 \\
2022-01-30 & DOUTOR PEDRINHO & 1 & -26.705567 & -49.553839 \\
2022-02-27 & SÃO JOSÉ DO CERRITO & 1 & -27.598272 & -50.650263 \\
2022-03-13 & BOCAINA DO SUL & 1 & -27.763118 & -49.909163 \\
2022-03-13 & IMBUIA & 2 & -27.502913 & -49.406526 \\
2022-03-13 & PRESIDENTE CASTELLO BRANCO & 2 & -27.233335 & -51.780936 \\
2022-03-20 & BOM RETIRO & 1 & -27.785360 & -49.563009 \\
2022-07-31 & MATOS COSTA & 1 & -26.481288 & -51.141830 \\
2023-02-26 & SANTA TEREZINHA & 1 & -26.667603 & -50.005296 \\
2023-03-19 & ANITÁPOLIS & 1 & -27.879776 & -49.138750 \\
2023-03-26 & SANTA CECÍLIA & 1 & -26.926038 & -50.446240 \\
2023-05-14 & ABDON BATISTA & 5 & -27.583825 & -51.056263 \\
2023-11-12 & PEDRAS GRANDES & 1 & -28.479421 & -49.204475 \\
\end{longtable}


\begin{longtable}[htbp]{llcrr}
\label{tab:primeiros_casos}

\caption{Primeiros registros de casos de dengue em municípios catarinenses.} \\
\hline
\rowcolor{darkgray} \textcolor{white}{Semana} & \textcolor{white}{Município} & \textcolor{white}{Casos} & \textcolor{white}{latitude} & \textcolor{white}{longitude} \\
\hline
\endfirsthead

\caption{(Continuação) Primeiros registros de casos de dengue em municípios catarinenses.} \\
\rowcolor{darkgray} \textcolor{white}{Semana} & \textcolor{white}{Município} & \textcolor{white}{Casos} & \textcolor{white}{latitude} & \textcolor{white}{longitude} \\
\hline
\endhead

\hline
\textit{Continua na próxima página}
\hline
\endfoot

\hline
\textit{Finalização da tabela} \\
\hline
\endlastfoot




2014-04-13 & RIO DO SUL & 1 & -27.196051 & -49.629953 \\
2014-05-18 & MORRO DA FUMAÇA & 1 & -28.637696 & -49.242862 \\
2014-06-01 & MASSARANDUBA & 1 & -26.626911 & -48.988192 \\
2014-07-06 & ITAJAÍ & 1 & -26.969013 & -48.753417 \\
2014-12-21 & MARAVILHA & 1 & -26.759974 & -53.198113 \\
2015-02-15 & BLUMENAU & 1 & -26.885767 & -49.097309 \\
2015-03-01 & JOINVILLE & 1 & -26.244282 & -48.951405 \\
2015-03-08 & ARAQUARI & 1 & -26.461708 & -48.757777 \\
2015-03-08 & BALNEÁRIO CAMBORIÚ & 2 & -27.004737 & -48.621741 \\
2015-03-22 & CAÇADOR & 1 & -26.763016 & -51.084665 \\
2015-03-22 & ITAPEMA & 4 & -27.108583 & -48.634427 \\
2015-03-22 & CHAPECÓ & 5 & -27.125144 & -52.650339 \\
2015-04-05 & MAJOR GERCINO & 1 & -27.427926 & -49.052632 \\
2015-04-12 & CORUPÁ & 1 & -26.439934 & -49.326711 \\
2015-04-19 & BOMBINHAS & 1 & -27.173256 & -48.517487 \\
2015-04-19 & TUBARÃO & 1 & -28.481581 & -49.037249 \\
2015-04-26 & CORDILHEIRA ALTA & 1 & -26.975707 & -52.641660 \\
2015-04-26 & SÃO MIGUEL DO OESTE & 1 & -26.727237 & -53.512121 \\
2015-05-24 & CANOINHAS & 1 & -26.249950 & -50.533371 \\
2015-06-07 & GUARACIABA & 1 & -26.580106 & -53.571076 \\
2015-11-29 & BOM JESUS DO OESTE & 1 & -26.684021 & -53.095288 \\
2015-11-29 & PINHALZINHO & 2 & -26.829373 & -52.976632 \\
2015-12-06 & MODELO & 1 & -26.774777 & -53.051096 \\
2015-12-27 & GAROPABA & 1 & -28.046734 & -48.658936 \\
2015-12-27 & INDAIAL & 1 & -26.994979 & -49.224178 \\
2016-01-10 & BOM JESUS & 1 & -26.739042 & -52.389991 \\
2016-01-17 & CORONEL FREITAS & 2 & -26.883161 & -52.725052 \\
2016-01-17 & SÃO JOSÉ DO CEDRO & 1 & -26.480851 & -53.533736 \\
2016-01-24 & ITAPOÁ & 1 & -26.082477 & -48.652354 \\
2016-01-24 & SÃO LOURENÇO DO OESTE & 1 & -26.465488 & -52.859773 \\
2016-01-24 & CAIBI & 1 & -27.025176 & -53.263058 \\
2016-01-24 & SERRA ALTA & 1 & -26.692621 & -53.025238 \\
2016-01-31 & DESCANSO & 2 & -26.858414 & -53.479301 \\
2016-02-07 & FLORIANÓPOLIS & 1 & -27.577834 & -48.508198 \\
2016-02-14 & SAUDADES & 2 & -26.897207 & -53.040014 \\
2016-02-21 & XANXERÊ & 1 & -26.872279 & -52.409754 \\
2016-03-06 & GUATAMBÚ & 1 & -27.116474 & -52.783390 \\
2016-03-13 & BRUSQUE & 2 & -27.125000 & -48.909743 \\
2016-03-13 & PALMITOS & 3 & -27.092622 & -53.179829 \\
2016-04-03 & UNIÃO DO OESTE & 2 & -26.786666 & -52.856010 \\
2016-04-17 & NOVA ITABERABA & 2 & -26.961142 & -52.833166 \\
2016-05-01 & QUILOMBO & 1 & -26.729236 & -52.717982 \\
2016-11-13 & SÃO JOÃO BATISTA & 1 & -27.328928 & -48.859016 \\
2017-12-03 & SOMBRIO & 1 & -29.070191 & -49.655927 \\
2018-05-14 & CAMBORIÚ & 1 & -27.070939 & -48.708989 \\
2018-09-24 & LAGUNA & 1 & -28.486421 & -48.825930 \\
2019-02-03 & CUNHA PORÃ & 1 & -26.879228 & -53.190665 \\
2019-03-03 & PORTO BELO & 1 & -27.174796 & -48.616435 \\
2019-03-17 & NAVEGANTES & 1 & -26.828743 & -48.727396 \\
2019-04-14 & RIQUEZA & 1 & -26.978455 & -53.344352 \\
2019-06-09 & BALNEÁRIO PIÇARRAS & 1 & -26.759492 & -48.733290 \\
2020-01-12 & JARAGUÁ DO SUL & 1 & -26.481908 & -49.159974 \\
2020-02-02 & GASPAR & 1 & -26.931343 & -48.966033 \\
2020-02-09 & ÁGUAS DE CHAPECÓ & 1 & -27.054386 & -52.956735 \\
2020-02-16 & PENHA & 1 & -26.807760 & -48.650765 \\
2020-02-23 & SÃO CARLOS & 1 & -27.030109 & -53.031412 \\
2020-03-01 & TIJUCAS & 2 & -27.247563 & -48.700886 \\
2020-03-01 & SÃO JOSÉ & 2 & -27.578471 & -48.656256 \\
2020-03-08 & ANCHIETA & 1 & -26.532948 & -53.333109 \\
2020-03-08 & DIONÍSIO CERQUEIRA & 1 & -26.329843 & -53.533105 \\
2020-03-22 & SÃO FRANCISCO DO SUL & 1 & -26.261597 & -48.640319 \\
2020-03-29 & PALMA SOLA & 3 & -26.373544 & -53.314529 \\
2020-03-29 & FORMOSA DO SUL & 1 & -26.634189 & -52.793789 \\
2020-03-29 & XAXIM & 2 & -26.970309 & -52.529105 \\
2020-04-05 & NOVA ERECHIM & 1 & -26.908169 & -52.906693 \\
2020-04-05 & IPUAÇU & 3 & -26.678739 & -52.476014 \\
2020-04-12 & ITAPIRANGA & 1 & -27.112222 & -53.711674 \\
2020-04-26 & IRATI & 1 & -26.627592 & -52.890099 \\
2020-04-26 & GUARUJÁ DO SUL & 1 & -26.398336 & -53.484021 \\
2020-05-10 & ABELARDO LUZ & 2 & -26.562427 & -52.255719 \\
2020-05-24 & ENTRE RIOS & 1 & -26.739710 & -52.578368 \\
2020-06-07 & MONDAÍ & 3 & -27.099293 & -53.446497 \\
2020-06-21 & SANTA TEREZINHA DO PROGRESSO & 1 & -26.592749 & -53.169925 \\
2020-11-29 & CAMPO ERÊ & 1 & -26.450419 & -53.130290 \\
2021-01-31 & CONCÓRDIA & 1 & -27.239127 & -52.007382 \\
2021-02-14 & IPORÃ DO OESTE & 1 & -27.000432 & -53.489414 \\
2021-02-14 & BALNEÁRIO BARRA DO SUL & 1 & -26.446419 & -48.652569 \\
2021-03-14 & SEARA & 2 & -27.147833 & -52.340694 \\
2021-03-14 & SANTA HELENA & 1 & -26.921956 & -53.619652 \\
2021-03-14 & CAMPO ALEGRE & 1 & -26.120556 & -49.217539 \\
2021-03-28 & BARRA VELHA & 1 & -26.662441 & -48.727386 \\
2021-04-04 & PALHOÇA & 1 & -27.771821 & -48.661708 \\
2021-04-04 & TUNÁPOLIS & 2 & -26.988117 & -53.649949 \\
2021-04-18 & GARUVA & 3 & -26.057112 & -48.867328 \\
2021-04-18 & FLOR DO SERTÃO & 1 & -26.763781 & -53.337342 \\
2021-04-18 & ITÁ & 1 & -27.244015 & -52.332158 \\
2021-04-25 & ILHOTA & 1 & -26.859041 & -48.857351 \\
2021-05-09 & GUARAMIRIM & 1 & -26.476551 & -48.944882 \\
2021-05-09 & BELMONTE & 1 & -26.857082 & -53.618060 \\
2021-06-27 & SÃO BERNARDINO & 1 & -26.487695 & -52.987611 \\
2022-01-30 & ROMELÂNDIA & 1 & -26.648390 & -53.317124 \\
2022-02-06 & IPUMIRIM & 1 & -27.040309 & -52.152891 \\
2022-02-20 & SANTIAGO DO SUL & 1 & -26.630874 & -52.697312 \\
2022-02-20 & ALTO BELA VISTA & 1 & -27.421535 & -51.925948 \\
2022-02-27 & PRINCESA & 1 & -26.432945 & -53.611776 \\
2022-02-27 & PLANALTO ALEGRE & 1 & -27.051964 & -52.868983 \\
2022-02-27 & IRACEMINHA & 3 & -26.845317 & -53.324456 \\
2022-02-27 & SÃO MIGUEL DA BOA VISTA & 2 & -26.689643 & -53.239962 \\
2022-02-27 & CAXAMBU DO SUL & 1 & -27.146711 & -52.920211 \\
2022-02-27 & GUABIRUBA & 1 & -27.103338 & -49.020750 \\
2022-03-06 & ASCURRA & 3 & -26.973634 & -49.395966 \\
2022-03-06 & SÃO JOÃO DO OESTE & 4 & -27.091607 & -53.591664 \\
2022-03-06 & LINDÓIA DO SUL & 1 & -27.031507 & -52.052466 \\
2022-03-06 & FAXINAL DOS GUEDES & 1 & -26.846669 & -52.244323 \\
2022-03-06 & TIGRINHOS & 1 & -26.676067 & -53.156546 \\
2022-03-06 & VARGEÃO & 1 & -26.778693 & -52.124176 \\
2022-03-06 & PERITIBA & 1 & -27.351361 & -51.875059 \\
2022-03-13 & SALTINHO & 1 & -26.590522 & -53.022279 \\
2022-03-13 & PONTE SERRADA & 1 & -26.861769 & -51.928214 \\
2022-03-13 & XAVANTINA & 1 & -27.021214 & -52.321890 \\
2022-03-20 & PIRATUBA & 3 & -27.461219 & -51.772820 \\
2022-03-20 & CUNHATAÍ & 2 & -26.975087 & -53.100870 \\
2022-03-20 & PARAÍSO & 2 & -26.662800 & -53.678304 \\
2022-03-20 & RODEIO & 2 & -26.895097 & -49.351036 \\
2022-03-27 & NOVA TRENTO & 1 & -27.313268 & -49.041367 \\
2022-03-27 & LAJEADO GRANDE & 1 & -26.854941 & -52.552085 \\
2022-03-27 & IMBITUBA & 1 & -28.194274 & -48.702663 \\
2022-03-27 & BARRA BONITA & 3 & -26.668358 & -53.425055 \\
2022-03-27 & OURO VERDE & 1 & -26.711533 & -52.278758 \\
2022-03-27 & MAREMA & 1 & -26.812383 & -52.630070 \\
2022-03-27 & SÃO DOMINGOS & 1 & -26.539387 & -52.554602 \\
2022-03-27 & ÁGUAS FRIAS & 5 & -26.851639 & -52.852311 \\
2022-03-27 & TIMBÓ & 1 & -26.808192 & -49.268848 \\
2022-04-03 & SUL BRASIL & 1 & -26.695342 & -52.946800 \\
2022-04-03 & IPIRA & 1 & -27.370819 & -51.798708 \\
2022-04-10 & BRAÇO DO NORTE & 1 & -28.240945 & -49.142365 \\
2022-04-10 & IRANI & 1 & -27.025235 & -51.917948 \\
2022-04-10 & JOAÇABA & 1 & -27.151796 & -51.592935 \\
2022-04-10 & POMERODE & 1 & -26.728186 & -49.173304 \\
2022-04-10 & GALVÃO & 2 & -26.449409 & -52.659011 \\
2022-04-10 & IÇARA & 2 & -28.751615 & -49.277346 \\
2022-04-10 & CRICIÚMA & 1 & -28.715695 & -49.379716 \\
2022-04-10 & BIGUAÇU & 1 & -27.433530 & -48.693947 \\
2022-04-17 & BANDEIRANTE & 1 & -26.768596 & -53.648168 \\
2022-04-17 & APIÚNA & 1 & -27.124502 & -49.364245 \\
2022-04-24 & CANELINHA & 1 & -27.234604 & -48.802310 \\
2022-05-01 & NOVA VENEZA & 1 & -28.684718 & -49.584912 \\
2022-05-01 & BOTUVERÁ & 1 & -27.218576 & -49.123526 \\
2022-05-01 & URUBICI & 1 & -28.038876 & -49.577458 \\
2022-05-01 & LAURENTINO & 2 & -27.207982 & -49.733614 \\
2022-05-01 & PAIAL & 1 & -27.213003 & -52.487869 \\
2022-05-08 & JARDINÓPOLIS & 1 & -26.717195 & -52.871306 \\
2022-05-08 & URUSSANGA & 1 & -28.492699 & -49.328277 \\
2022-05-08 & ARABUTÃ & 1 & -27.140868 & -52.181545 \\
2022-05-08 & POUSO REDONDO & 1 & -27.286166 & -49.977272 \\
2022-05-08 & GOVERNADOR CELSO RAMOS & 1 & -27.376617 & -48.577230 \\
2022-05-15 & COCAL DO SUL & 1 & -28.598697 & -49.333930 \\
2022-05-22 & SCHROEDER & 1 & -26.367952 & -49.060456 \\
2022-05-29 & IBICARÉ & 1 & -27.092470 & -51.370040 \\
2022-06-12 & CATANDUVAS & 1 & -27.044998 & -51.696411 \\
2022-12-18 & RANCHO QUEIMADO & 1 & -27.673932 & -49.089881 \\
2023-01-22 & DOUTOR PEDRINHO & 1 & -26.705567 & -49.553839 \\
2023-02-19 & RIO DOS CEDROS & 1 & -26.617927 & -49.368385 \\
2023-02-26 & PASSO DE TORRES & 1 & -29.267248 & -49.721932 \\
2023-03-05 & SÃO LUDGERO & 1 & -28.353135 & -49.168162 \\
2023-03-05 & SANTO AMARO DA IMPERATRIZ & 3 & -27.741766 & -48.796550 \\
2023-03-12 & LUZERNA & 1 & -27.090413 & -51.505633 \\
2023-03-19 & PAULO LOPES & 1 & -27.964855 & -48.760163 \\
2023-03-26 & SÃO JOÃO DO ITAPERIÚ & 1 & -26.591052 & -48.797733 \\
2023-03-26 & ÁGUAS MORNAS & 1 & -27.733481 & -48.936027 \\
2023-04-02 & LAURO MÜLLER & 1 & -28.384585 & -49.451932 \\
2023-04-09 & ANTÔNIO CARLOS & 1 & -27.497474 & -48.830982 \\
2023-04-09 & FORQUILHINHA & 1 & -28.781693 & -49.493283 \\
2023-04-09 & MAFRA & 1 & -26.200812 & -49.889559 \\
2023-04-16 & SANTA TEREZINHA & 1 & -26.667603 & -50.005296 \\
2023-04-16 & GRAVATAL & 1 & -28.354020 & -49.019524 \\
2023-04-16 & PAPANDUVA & 1 & -26.502424 & -50.172362 \\
2023-04-23 & HERVAL D'OESTE & 1 & -27.195376 & -51.410665 \\
2023-04-30 & SÃO PEDRO DE ALCÂNTARA & 1 & -27.591567 & -48.837131 \\
2023-04-30 & PRESIDENTE GETÚLIO & 2 & -27.062584 & -49.714430 \\
2023-04-30 & ANITÁPOLIS & 1 & -27.879776 & -49.138750 \\
2023-05-07 & LEOBERTO LEAL & 1 & -27.472140 & -49.257222 \\
2023-05-14 & ALFREDO WAGNER & 1 & -27.705947 & -49.330751 \\
2023-05-21 & ITUPORANGA & 1 & -27.453450 & -49.538870 \\
2023-06-04 & SÃO BENTO DO SUL & 1 & -26.294902 & -49.349982 \\
2023-06-11 & SALETE & 1 & -26.969623 & -50.010685 \\
2023-07-09 & ORLEANS & 1 & -28.279938 & -49.371860 \\
2023-11-05 & SÃO BONIFÁCIO & 1 & -27.956554 & -48.939098 \\
\end{longtable}

 % \end{apendicesenv}

% ----------------------------------------------------------
% Anexos
% ----------------------------------------------------------
 % \begin{anexosenv}
 %     \partanexos
 %     \include{tex/Anexos}
 % \end{anexosenv}

%---------------------------------------------------------------------
% INDICE REMISSIVO
%---------------------------------------------------------------------
%\phantompart
%\printindex
%---------------------------------------------------------------------

\end{document}

