\chapter{Apresentação dos Resultados}

\section{Análise e discussão dos resultados}

\section{Análise Estatística Descritiva \textcolor{red}{Climatologia}}

\Para melhor compreensão da descrição, analise a tabela (ou quadro?) (tabela \ref{tab:valores_climato}):

\begin{table}
    \centering
    \caption{Valores Estaduais de Variáveis Climatológicas}
    \begin{tabular}{|r|cccc|}
    \hline
        Valores Estaduais & Mínima & Média & Máxima & Desvio Padrão\\
    \hline
    Temperatura Mínima Diária & -7,98 C & 14,75 C& 30,26 C & 4,88 C\\
    Temperatura Mínima Semanal & -2,77 C & 14,75 C & 26,15 C & 4,11 C\\
    Temperatura Média Diária & -4,39 C & 18,66 C & 36,14 C & 4,78 C\\
    Temperatura Média Semanal & 1,68 C & 18,66 C & 33,67 C & 4,19 C\\
    Temperatura Máxima Diária & -1,21 C & 24,31 C & 40,72 C & 5,41 C\\
    Temperatura Máxima Semanal & 7,94 C & 24,31 C & 37,91 C & 4,52 C\\ 
    Precipitação Diária & 0,0 mm & 4,3 mm & 261,25 mm & 12,68 mm\\
    Precipitação Semanal & 0,0 mm & 30,07 mm & 478,25 mm & 43,05 mm\\
    \hline
    \end{tabular}
    \label{tab:valores_climato}
\end{table}

\indent Sobre a temperatura mínima diária, o menor valor que a série histórica do Estado de \acrshort{SC} apresentou foi de -7,98 C, sendo -2,77 C quando agrupado semanalmente. O valor máximo da temperatura mínima foi de 30,26 C, dados diários, e 26,15 C, dados semanais. A média diária da temperatura mínima para o Estado foi de 14,75 C, sendo o mesmo valor quando agrupado em semana, e o maior desvio padrão de 4,88 C.

\indent Em se tratando de temperatura média diária, o menor valor que a série histórica do Estado de \acrshort{SC} apresentou foi de -4,39 C, sendo 1,68 C para dados semanais. O valor máximo da temperatura média foi de 36,14 C, dados diários, e 26,15 C, dados semanais. A média diária da temperatura mínima para o Estado foi de 14,75 C, sendo o mesmo valor quando agrupado em semana, e o maior desvio padrão de 4,88 C. 


O mesmo ocorre para a temperatura média diária (-


TMIN ESTADUAL
Mínima: -7.98 C
Média: 14.75 C
Desvio Padrão Máximo: 4.88 C
Máxima: 30.26 C
TMED ESTADUAL
Mínima: -4.39 C
Média: 18.66 C
Desvio Padrão Máximo: 4.78 C
Máxima: 36.14 C
TMAX ESTADUAL
Mínima: -1.21 C
Média: 24.31 C
Desvio Padrão Máximo: 5.41 C
Máxima: 40.72 C
PREC ESTADUAL
Mínima: 0.0 mm
Média: 4.3 mm
Desvio Padrão Máximo: 12.68 mm
Máxima: 261.25 mm

TMIN ESTADUAL SEMANAL
Mínima: -2.77 C
Média: 14.75 C
Desvio Padrão Máximo: 4.11 C
Máxima: 26.15 C
TMED ESTADUAL SEMANAL
Mínima: 1.68 C
Média: 18.66 C
Desvio Padrão Máximo: 4.19 C
Máxima: 33.67 C
TMAX ESTADUAL SEMANAL
Mínima: 7.94 C
Média: 24.31 C
Desvio Padrão Máximo: 4.52 C
Máxima: 37.91 C
PREC ESTADUAL SEMANAL
Mínima: 0.0 mm
Média: 30.07 mm
Desvio Padrão Máximo: 43.05 mm
Máxima: 478.25 mm

\section{Modelagem Preditiva}

\subsection{Pré-processamento}

\subsection{Processamento}

\indent Citar correlações.

\indent A rede neural apresentou baixo ajuste frente ao comportamento dos fenômenos.

\indent \textcolor{red}{explicar como evitar "double dipping".}

\subsection{Pós-processamento}

\indent Os dados de casos de dengue, quando comparados entre as fontes (\acrshort{Dive} e \acrshort{DataSUS}), há diferença de atualização. Sendo os dados da própria \acrshort{Dive} mais atualizados, sendo eleitos para treinamento do modelo.

\section{Validação dos Modelos}

\section{Estudo de Caso}

\section{Produto Técnico Tecnológico (PTT)}



%https://cdn.embedly.com/widgets/media.html?src=https%3A%2F%2Fwww.youtube.com%2Fembed%2F3tiuRHuzST4&display_name=YouTube&url=https%3A%2F%2Fwww.youtube.com%2Fwatch%3Fv%3D3tiuRHuzST4&image=http%3A%2F%2Fi.ytimg.com%2Fvi%2F3tiuRHuzST4%2Fhqdefault.jpg&key=40cb30655a7f4a46adaaf18efb05db21&type=text%2Fhtml&schema=youtube
