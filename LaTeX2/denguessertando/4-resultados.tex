\chapter{Apresentação dos Resultados}

\section{Análise e discussão dos resultados}

\indent Como forma de otimizar a leitura, os resultados serão apresentados e discutidos na sequência. Também como facilitação, a ordem e numeração apresentados seguirão os mesmos adotados na metodologia.

\section{Análise Estatística Descritiva \textcolor{red}{Climatologia}}

\indent Para melhor compreensão da descrição climatológica, acompanhe a tabela \textcolor{red}{(ou quadro?)} (tabela \ref{tab:valores_climato}) abaixo:

\begin{table}[htbp]
    \centering
    \caption{Valores estaduais de variáveis climatológicas}
    {\rowcolors{1}{lightgray}{white}
    \begin{tabular}{r|cccc}
    \hline
    \rowcolor{darkgray} \textcolor{white}{Valores Estaduais} & \textcolor{white}{Mínima} & \textcolor{white}{Média} & \textcolor{white}{Máxima} & \textcolor{white}{Desvio Padrão}\\
    \hline
    Temperatura Mínima Diária & -7,98 C & 14,75 C& 30,26 C & 4,88 C\\
    Temperatura Mínima Semanal & -2,77 C & 14,75 C & 26,15 C & 4,11 C\\
    Temperatura Média Diária & -4,39 C & 18,66 C & 36,14 C & 4,78 C\\
    Temperatura Média Semanal & 1,68 C & 18,66 C & 33,67 C & 4,19 C\\
    Temperatura Máxima Diária & -1,21 C & 24,31 C & 40,72 C & 5,41 C\\
    Temperatura Máxima Semanal & 7,94 C & 24,31 C & 37,91 C & 4,52 C\\ 
    Precipitação Diária & 0,0 mm & 4,3 mm & 261,25 mm & 12,68 mm\\
    Precipitação Semanal & 0,0 mm & 30,07 mm & 478,25 mm & 43,05 mm\\
    \hline
    \end{tabular}}
    \label{tab:valores_climato}
\end{table}

\indent Sobre a temperatura mínima diária, o menor valor que a série histórica do Estado de \acrlong{SC} apresentou foi de -7,98 C, sendo -2,77 C quando agrupado semanalmente. O valor máximo da temperatura mínima foi de 30,26 C, dados diários, e 26,15 C, dados semanais. A média diária da temperatura mínima para o Estado foi de 14,75 C, sendo o mesmo valor quando agrupado em semana, e o maior desvio padrão de 4,88 C para dados diários. O desvio padrão semanal foi de 4,11 C.

\indent Em se tratando de temperatura média diária, o menor valor que a série histórica do Estado de \acrlong{SC} apresentou foi de -4,39 C, sendo 1,68 C para dados semanais. O valor máximo da temperatura média foi de 36,14 C, dados diários, e 36,14 C, dados semanais. A média diária da temperatura mínima para o Estado foi de 18,66 C, tanto para valores diários quanto semanais, e o desvio padrão foi de 4,78 C e 4,19 C, dados diários e semanais, respectivamente. 

\indent Para a temperatura máxima diária, o menor valor foi de -1,39 C, sendo 7,94 C para dados semanais. O valor máximo da temperatura máxima foi de 40,72 C, dados diários, e 37,91 C, dados semanais. A média diária da temperatura máxima para o Estado de \acrlong{SC} foi de 24,31 C, tanto para valores diários quanto semanais, e o desvio padrão foi de 5,41 C e 4,52 C, dados diários e semanais, respectivamente.

\indent Todas as temperaturas apresentaram comportamento semelhante, mínima, média e máxima, por compartilharem o mesmo método de agrupamento semanal, baseado na média dos dias da semana. Logo, tem-se valores de máximas e mínimas próximos a média quando agrupados semanalmente e os valores de médias diárias e semanais são semelhantes, por conta do próprio método. Essa suavização por agrupamento também é verificada no desvio padrão, sendo menor em dados semanais. 

\indent A precipitação na série histórica de \acrlong{SC} obteve seu maior valor em 261,25 mm diários e 478,25 mm semanais. O menor valor foi zero (0) mm e igual em ambos os períodos analisados, diários ou semanais. A média de precipitação para o Estado é de 4,3 mm diários e 30,07 mm semanais, tendo 12,68 mm e 43,05 mm como desvio padrão para dados diários e semanais, respectivamente.

\indent O comportamento da precipitação é menos suave aos dados agrupados semanalmente, diferente das temperaturas. Porém, assim como as temperaturas, esse comportamento é por conta do método de agrupamento, baseado no somatório dos dias da semana. Logo, os valores semanais de precipitação são maiores, sejam eles médio, máximo ou desvio padrão. O valor mínimo diário e semanal é semelhante, zero (0) mm, pois há períodos em que não precipita em \acrlong{SC} por mais de sete (7) dias. Esse valor também diz respeito a característica do dado, sendo uma variável quantitativa de razão, em contraposição às temperaturas, que são variáveis quantitativas intervalares.

\indent \textcolor{red}{Interessante Amplitude Térmica? (Debater com o Mário)}

\section{Modelagem Preditiva}

\subsection{Pré-processamento}

\subsection{Processamento}

\indent Citar correlações. \citeonline{StatsDummies} \acrfull{r} \acrlong{r} \acrshort{r}
\indent \citeonline{espurioCorr}

\indent A rede neural apresentou baixo ajuste frente ao comportamento dos fenômenos.

\indent \textcolor{red}{explicar como evitar "double dipping".}

\subsection{Pós-processamento}

\indent Os dados de casos de dengue, quando comparados entre as fontes (\acrshort{Dive} e \acrshort{DataSUS}), há diferença de atualização. Sendo os dados da própria \acrshort{Dive} mais atualizados, sendo eleitos para treinamento do modelo.

\section{Validação dos Modelos}

\section{Estudo de Caso}

\section{Produto Técnico Tecnológico (PTT)}



%https://cdn.embedly.com/widgets/media.html?src=https%3A%2F%2Fwww.youtube.com%2Fembed%2F3tiuRHuzST4&display_name=YouTube&url=https%3A%2F%2Fwww.youtube.com%2Fwatch%3Fv%3D3tiuRHuzST4&image=http%3A%2F%2Fi.ytimg.com%2Fvi%2F3tiuRHuzST4%2Fhqdefault.jpg&key=40cb30655a7f4a46adaaf18efb05db21&type=text%2Fhtml&schema=youtube
