\chapter{Apresentação dos Resultados}

TMIN ESTADUAL
Mínima: -7.98 C
Média: 14.75 C
Desvio Padrão Máximo: 4.88 C
Máxima: 30.26 C
TMED ESTADUAL
Mínima: -4.39 C
Média: 18.66 C
Desvio Padrão Máximo: 4.78 C
Máxima: 36.14 C
TMAX ESTADUAL
Mínima: -1.21 C
Média: 24.31 C
Desvio Padrão Máximo: 5.41 C
Máxima: 40.72 C
PREC ESTADUAL
Mínima: 0.0 mm
Média: 4.3 mm
Desvio Padrão Máximo: 12.68 mm
Máxima: 261.25 mm

TMIN ESTADUAL SEMANAL
Mínima: -2.77 C
Média: 14.75 C
Desvio Padrão Máximo: 4.11 C
Máxima: 26.15 C
TMED ESTADUAL SEMANAL
Mínima: 1.68 C
Média: 18.66 C
Desvio Padrão Máximo: 4.19 C
Máxima: 33.67 C
TMAX ESTADUAL SEMANAL
Mínima: 7.94 C
Média: 24.31 C
Desvio Padrão Máximo: 4.52 C
Máxima: 37.91 C
PREC ESTADUAL SEMANAL
Mínima: 0.0 mm
Média: 30.07 mm
Desvio Padrão Máximo: 43.05 mm
Máxima: 478.25 mm


\indent Os dados de casos de dengue, quando comparados entre as fontes (\acrshort{Dive} e \acrshort{DataSUS}), há diferença de atualização. Sendo os dados da própria \acrshort{Dive} mais atualizados, sendo eleitos para treinamento do modelo.

\indent A rede neural apresentou baixo ajuste frente ao comportamento dos fenômenos.

Texto texto texto texto texto texto texto texto texto texto texto texto texto texto texto texto texto texto texto texto texto texto texto texto texto texto texto texto texto texto texto texto texto texto texto texto texto texto texto.

\section{Análise e discussão dos resultados}

Texto texto texto texto texto texto texto texto texto texto texto texto texto texto texto texto texto texto texto texto texto texto texto texto texto texto texto texto texto texto texto texto texto texto texto texto texto texto texto.

\indent \textcolor{red}{explicar como evitar "double dipping".}


%https://cdn.embedly.com/widgets/media.html?src=https%3A%2F%2Fwww.youtube.com%2Fembed%2F3tiuRHuzST4&display_name=YouTube&url=https%3A%2F%2Fwww.youtube.com%2Fwatch%3Fv%3D3tiuRHuzST4&image=http%3A%2F%2Fi.ytimg.com%2Fvi%2F3tiuRHuzST4%2Fhqdefault.jpg&key=40cb30655a7f4a46adaaf18efb05db21&type=text%2Fhtml&schema=youtube
