\chapter{Introdução}
%Máximo 2pg:

% Contextualizar o Clima de SC;
\indent O clima é a composição do tempo meterológico ao longo de um período, pelo menosn trinta anos, em um determinado território. (\acrlong{SC}, \citeyear{AtlasSCnatureza}, pg-75 ).\\
\indent O estado de \acrfull{SC} está localizado no sul do Brasil, região onde ocorrem totais pluviométricos elevados e precipitação bem distribuída ao longo do ano. Parte desse comportamento é devido a atuação sistemas meteorológicos como: \acrfull{FF}, Sistemas Frontais, Clicones Extratropicais, Vórtices Clicônicos em Altos Níveis, Bloqueios Atmosféricos e \acrfull{CCM}, além a influência indireta da \acrfull{ZCAS} \cite{reboita2010}.\\
\begin{center}
\textcolor{red}{Incluir \cite{Guerra2023Regionalizacao}}
\end{center}
\textcolor{red}{CONTEXTUALIZAR TEMPERATURA/CLIMA EM SC}\\
\textcolor{red}{CONSIDERAÇÕES AQUI!}\\ 

%Contextualizar Dengue;
\indent Assim, sendo expresso no Guia de Vigilância em Saúde, o mosquito \latim{Aedes aegypti} pode comumente transmitir algumas arboviroses, como: Dengue, Zika e Chikungunya \cite{GuiaVigSaúde22}; o \latim{A. aegypti} também é responsável pela transmissão de Febre Amarela em ciclo urbano. Ainda informado pelo \citeonline{GuiaVigSaúde22}, a dengue possui como agente etiológico o \acrfull{DENV}, sendo agrupado em quatro sorotipos distintos (\acrshort{DENV}-1, \acrshort{DENV}-2, \acrshort{DENV}-3 e \acrshort{DENV}-4), e é uma das zoonoses vetorizadas por artrópodes mais relevantes nas Américas.\\
\textcolor{red}{Relacionar Clima e Dengue (Geral)}\\
\indent Em estudo realizado por \citeonline{Drumond2020Dinamica}, entre 2007 e 2017 no \acrfull{DF}, o número de casos foi maior nos primeiros semestres, particularmente entre o final do verão e o início do outono. Este período  coincide com o final do período chuvoso. Os autores também observaram que a dengue se manteve hiperendêmica no \acrlong{DF}, tendo casos registrados durante todos os meses do ano e circulação dos quatro sorotipos do vírus.\\
\begin{center}
\textcolor{red}{CONSIDERAÇÕES AQUI!}\\ 
\end{center}
\textcolor{red}{INCLUIR OUTROS ESTUDOS}
\textcolor{red}{Relacionar Clima e Dengue (SC)}\\
\textcolor{red}{INCLUIR OUTROS ESTUDOS}\\
\textcolor{red}{Relacionar Clima e Dengue (SC)}\\
{\color{red} \rule{\linewidth}{0.5mm}}\color{red} \\ INÍCIO DE BLOCO DE CONSIDERAÇÃO\\
\indent O Estado catarinense apresentou em 2014 apenas três casos autóctones de dengue (infectados dentro do próprio estado), no município de Itajaí, e 66 casos importados. Naquele mesmo período, a região serrana e planalto não apresentavam focos \cite{Matiola2020Dissertação}. De acordo com a \citeonline{Informe5DiveSE9/23}, para este momento (final de março de 2023 - décima segunda semana epidemiológica), há focos de dengue distribuídos por todas as regiões catarinenses.\\
FINAL DE BLOCO DE CONSIDERAÇÃO \\ {\color{red} \rule{\linewidth}{0.5mm}}\color{black}
\indent Segundo \citeonline{Matiola2020Dissertação}, em \acrlong{SC}, há tendência de resposta à sazonalidade nos meses com temperaturas mais elevadas, observando-se o crescimento do número de focos do \latim{Aedes} sp.,  principalmente quando retroagidos por um período de 30 dias, além da diminuição nos meses mais frios.\\ 
\indent Portanto, a temperatura é um fator crítico para o desenvolvimento do mosquito no estado catarinense. com relação a previsibilidade das ocorrências, \citeonline{Matiola2020Dissertação} também sugerem que a precipitação do mês que antecede o registro do foco influencia no desenvolvimento do mosquito, corroborando com correlações maiores quando retroativas.\\

\begin{center}
\textcolor{red}{CONSIDERAÇÕES AQUI!}\\ 
\end{center}
\textcolor{red}{INCLUIR INFODENGUE COMO MODELO PREDITIVO}\\
\textcolor{red}{Contextualizar estudos com modelos Dengue/A.a}
%%%%%%%%%%%%%%%%%%%%%%%%%%%%%%%%%%%%%%%%%%%%%%%%%%%%%%%%%%%%%%%%%%%%%%%%%%%%
\newpage
\section{Justificativa}
%1parágrafo
\indent Pelo exposto, a dengue é uma zoonose emergente no estado de \acrlong{SC} e se faz necessário relacionar a nova dinâmica climatológica, a ocorrência de \latim{Aedes aegypti} e os fatores sociais frente a esse atual comportamento.
\begin{center}
\textcolor{red}{CONSIDERAÇÕES AQUI!}\\ 
\end{center}
\textcolor{red}{Necessidade de reconhecer relação das variáveis climáticas com a reprodução do mosquito, mudança no cenário da doença no estado associado a disseminação do mosquito...}\\
 
%%%%%%%%%%%%%%%%%%%%%%%%%%%%%%%%%%%%%%%%%%%%%%%%%%%%%%%%%%%%%%%%%%%%%%%%%%%%%
 \section{Objetivo Geral}
 Avaliar a relação entre os focos de \latim{Aedes} sp. e os casos de dengue e a condição climatológica predominante sobre o Estado de \acrlong{SC}, visando elaborar um modelo preditivo de proliferação do vetor biológico e da doença.

\begin{center}
\textcolor{red}{CONSIDERAÇÕES AQUI!}\\ 
\end{center}

 % Elaborar um modelo preditivo para a proliferação do mosquito \latim{Aedes aegypti}, a partir de análise das condições meteorológicas do Estado de \acrlong{SC}.

 % Analisar a relação entre a reprodução do mosquito \latim{Aedes aegypti} e as condições meteorológicas no Estado de \acrlong{SC}.

 
\section{Objetivos Específicos}
As estratégias para atingir o objetivo geral são descritas como objetivos específicos abaixo:

\begin{alineas}
\item Analisar estatisticamente padrões das séries temporais, a partir das regiões climatológicas determinadas por \citeonline{Guerra2023Regionalizacao}, para modelagens futuras; \textcolor{red}{REFINAR!}
\item Elaborar modelo preditivo para a proliferação dos mosquitos do gênero \latim{Aedes} sp. e os casos confirmados de dengue, a partir de análise das condições meteorológicas do Estado de \acrlong{SC}.
\item Analisar estatisticamente as modelagens para validação das ferramentas;
\item Realizar um estudo de caso para melhor entender o fenômeno a partir de um ciclo anual;
\item Sintetizar o \acrfull{PTT}:
\subitem Sistema Computacional;
\subitem Visualização Cartográfica desses resultados.
    %Último deve relacionar o produto
\end{alineas}

