\chapter{Introdução}
%Máximo 2pg:

% Contextualizar o Clima de SC;
\indent O clima é a composição do tempo meterológico ao longo de um período, pelo menos trinta anos, em um determinado território. (\acrlong{SC}, \citeyear{AtlasSCnatureza}, pg-75).

\indent O estado de \acrfull{SC} está localizado no sul do Brasil, região onde ocorrem totais pluviométricos elevados e precipitação bem distribuída ao longo do ano. Parte desse comportamento é devido a atuação sistemas meteorológicos como: \acrfull{FF}, Sistemas Frontais, Clicones Extratropicais, Vórtices Clicônicos em Altos Níveis, Bloqueios Atmosféricos e \acrfull{CCM}, além a influência indireta da \acrfull{ZCAS} \cite{reboita2010}.

\indent Atualmente, \acrlong{SC} apresenta seis (6) mesorregiões climáticas, por similaridade no comportamento de temperatura, precipitação, umidade relativa do ar, intensidade e direção dos ventos, e são as seguintes: Oeste; Planalto Sul e Meio Oeste; Planalto Norte e Vale do Itajaí; Planalto Serrano; Litoral Norte; e Litoral Sul \cite{Guerra2023Regionalizacao}.

\textcolor{red}{(?) CONTEXTUALIZAR TEMPERATURA/CLIMA EM SC (?)}

%Contextualizar Dengue;
\indent Antes de mais nada, para a \acrfull{Cobrade} os desastres podem ser divididos em naturais ou tecnológicos. Ainda para a \acrshort{Cobrade}, as epidemias são classificadas como desatres naturais biológicos, podendo ser divididas quanto ao agente etiológico: viral, bacteriano, parasítico ou fúngico \cite{GIRD}.

\indent Sobretudo, sendo expresso pelo \citeonline{GuiaVigSaúde22} no Guia de Vigilância em Saúde, o mosquito \latim{Aedes aegypti} (Linnaeus, 1762) pode comumente transmitir algumas arboviroses, como: Dengue, Zika e Chikungunya. A dengue possui como agente etiológico o vírus da dengue (\acrfull{DENV}), sendo agrupado em quatro sorotipos distintos (\acrshort{DENV}-1, \acrshort{DENV}-2, \acrshort{DENV}-3 e \acrshort{DENV}-4), e é uma das zoonoses vetorizadas por artrópodes mais relevantes nas Américas \cite{GuiaVigSaúde22}.

\indent Para \citeonline{ClimateDengue2019} a expansão global da dengue segue a expansão global dos vetores (gênero \latim{Aedes} sp.), que geograficamente se distribuem conforme condições climáticas, em especial pela temperatura. Para esses autores, não somente a expansão geográfica se deva ao incremento de temperatura, como a própria alteração no tempo do ciclo de vida (fases larvar, pupa e adulto) e dos períodos de incubação do vírus (extrínsico e intrínsico).

\indent Em estudo realizado por \citeonline{Drumond2020Dinamica}, entre 2007 e 2017 no \acrfull{DF}, o número de casos foi maior nos primeiros semestres, particularmente entre o final do verão e o início do outono. Este período  coincide com o final do período chuvoso. Os autores também observaram que a dengue se manteve hiperendêmica no \acrlong{DF}, tendo casos registrados durante todos os meses do ano e circulação dos quatro sorotipos do vírus.

\indent Há forte evidência de correlação entre epidemias de dengue e clima em diferentes capitais (Manaus, São Luís, Recife, Aracajú, Salvador, Belo Horizonte e Rio de Janeiro), principalmente utilizando temperatura média e frequência de precipitação \cite{ForecastingDengueBrazil2019}.

\indent \citeonline{PinhalzinhoDengue2020} apontaram que, em análise temporal, os casos de dengue apresentaram um padrão sazonal de ocorrência em Pinhalzinho/\acrshort{SC}. Foi observado nesse estudo que os casos tiveram início nos meses de dezembro e janeiro, alcançando
maior incidência em fevereiro e março, quando as temperaturas são mais elevadas, e decréscimo no mês de maio.

\indent Segundo \citeonline{Matiola2020Dissertação}, em \acrlong{SC}, há tendência de resposta à sazonalidade nos meses com temperaturas mais elevadas, observando-se o crescimento do número de focos do \latim{Aedes} sp.,  principalmente quando retroagidos por um período de 30 dias, além da diminuição nos meses mais frios.

\indent Portanto, a temperatura é um fator crítico para o desenvolvimento do mosquito no estado catarinense. com relação a previsibilidade das ocorrências, \citeonline{Matiola2020Dissertação} também sugerem que a precipitação do mês que antecede o registro do foco influencia no desenvolvimento do mosquito, corroborando com correlações maiores quando retroativas.

\indent Sendo assim, \citeonline{ForecastingDengueBrazil2019}, em seus estudos, perceberam que a temperatura média e a frequência de precipitação são variáveis com forte poder preditivo durante a estação de inverno em algumas capitais brasileiras (Aracajú, Belo Horizonte, Manaus e Rio de Janeiro).

\indent \textcolor{red}{CITAR INFODENGUE}

\indent Estudos recentes de (relatorio23infodengue) concluem que, para o ano de 2024, as modelagens indicam alta atividade de dengue no território nacional, em ambos os casos de ocorrência ou não de \latim{El Niño}. Esse mesmo relatório também indica risco de atividade aumentada de dengue para Santa Catarina.

Como comentado por (fastdengue), a atenção deve ser redobrada para a expansão da dengue em latitudes maiores, assim como regiões fronteiriças, entre Argentina, Paraguai e Uruguai.


\indent Por simplificação, o projeto tratará ambos os pernilongos \latim{Aedes aegypti} (Linnaeus, 1762) e \latim{Aedes albopictus} (Skuse, 1895) apenas como gênero \latim{Aedes} sp. (Meigen, 1818), da família \latim{Culicidae} (Meigen, 1818). Nomenclatura científica seguindo taxonomia citada e arquivada pelo Museu Nacional de História Natural, Instituto Smithsonian (\citeyear{ITIS}), através do Sistema Integrado de Informação Taxonômica (\acrfull{ITIS}).


\newpage

\begin{center}
\textcolor{red}{CONSIDERAÇÕES AQUI!}
\end{center}

{\color{red} \rule{\linewidth}{0.5mm}}

\color{red} INÍCIO DE BLOCO DE CONSIDERAÇÃO

\indent O Estado catarinense apresentou em 2014 apenas três casos autóctones de dengue (infectados dentro do próprio estado), no município de Itajaí, e 66 casos importados. Naquele mesmo período, a região serrana e planalto não apresentavam focos \cite{Matiola2020Dissertação}. De acordo com a \citeonline{Informe5DiveSE9/23}, para este momento (final de março de 2023 - décima segunda semana epidemiológica), há focos de dengue distribuídos por todas as regiões catarinenses.

FINAL DE BLOCO DE CONSIDERAÇÃO

{\color{red} \rule{\linewidth}{0.5mm}}

\color{black}


\textcolor{red}{TALVEZ JOGAR TUDO ISSO NA JUSTIFICATIVA}



%%%%%%%%%%%%%%%%%%%%%%%%%%%%%%%%%%%%%%%%%%%%%%%%%%%%%%%%%%%%%%%%%%%%%%%%%%%%
\newpage
\section{Justificativa}
%1parágrafo
\indent Pelo exposto, a dengue é uma zoonose emergente no estado de \acrlong{SC} e se faz necessário relacionar a nova dinâmica climatológica aos comportamentos de distribuição de \latim{Aedes} sp. e ocorrência de dengue.

\begin{center}
\textcolor{red}{CONSIDERAÇÕES AQUI!}
\end{center}
\textcolor{red}{Necessidade de reconhecer relação das variáveis climáticas com a reprodução do mosquito, mudança no cenário da doença no estado associado a disseminação do mosquito...}

\indent DESCRITIVO > PREDITIVO > PRESCRITIVO
 
%%%%%%%%%%%%%%%%%%%%%%%%%%%%%%%%%%%%%%%%%%%%%%%%%%%%%%%%%%%%%%%%%%%%%%%%%%%%%
 \section{Objetivo Geral}
 Avaliar a relação entre os focos de \latim{Aedes} sp. e os casos de dengue e a condição climatológica predominante sobre o Estado de \acrlong{SC}, visando elaborar um modelo preditivo de proliferação do vetor biológico e da doença.

\begin{center}
\textcolor{red}{CONSIDERAÇÕES AQUI!}\\ 
\end{center}

 % Elaborar um modelo preditivo para a proliferação do mosquito \latim{Aedes aegypti}, a partir de análise das condições meteorológicas do Estado de \acrlong{SC}.

 % Analisar a relação entre a reprodução do mosquito \latim{Aedes aegypti} e as condições meteorológicas no Estado de \acrlong{SC}.

 
\section{Objetivos Específicos}
As estratégias para atingir o objetivo geral são descritas como objetivos específicos abaixo:

\begin{alineas}
\item Analisar estatisticamente padrões das séries temporais, a partir das regiões climatológicas determinadas por \citeonline{Guerra2023Regionalizacao}, para modelagens futuras; \textcolor{red}{REFINAR!}
\item Elaborar modelo preditivo para a proliferação dos mosquitos do gênero \latim{Aedes} sp. e os casos confirmados de dengue, a partir de análise das condições meteorológicas do Estado de \acrlong{SC}.
\item Analisar estatisticamente as modelagens para validação das ferramentas;
\item Realizar um estudo de caso para melhor entender o fenômeno a partir de um ciclo anual;
\item Sintetizar o \acrfull{PTT}:
\subitem Sistema Computacional;
\subitem Visualização Cartográfica desses resultados.
    %Último deve relacionar o produto
\end{alineas}
